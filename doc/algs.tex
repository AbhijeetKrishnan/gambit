\chapter{Solution Algorithms}\label{SolutionAlgorithms}

\section{EnumMixed}
Finds all Nash equilibria for a two person game.  More
precisely, it finds the set of extreme points of the components
of the set of Nash equilibria.  The procedure is to enumerate the set
of complementary basic feasible solutions.  

{\em Limitations:}  Only works for two-person normal form games. 

The following parameters can be set:

\begin{description}
\item[nequilib:] Specifies the maximum number of equilibria to find.  The
default is zero, which means that all equilibria are found.  To check if
there is a unique Nash equilibrium, one could set this parameter to 2.

\item[time:] Returns the elapsed time for the operation.

\item[trace:] Sets the print level.  Higher values generate more output.
The default value is 0.
\end{description}


\section{Gobit}
Computes a branch of the logistic {\em quantal response equilibrium}
correspondence for n-person normal form games (as described in
\cite{McKPal:95a}  and n-person extensive form games (as described
in \cite{McKPal:95b}.) The branch is computed
for values of $\lambda$ between $minLam$ and $maxLam.$  The algorithm
starts at $\lambda(0) = minLam$ if $delLam>0,$ or $\lambda(0) = maxLam$ if
$delLam<0$. It then increments according to the formula $$
\lambda(t+1) = \lambda(t) +delLam \lambda(t)^a.
$$ 
where $minLam,$ $maxLam,$ $delLam,$ and $a$ are
parameters described below. In the computation for the first value of
$\lambda(0)$, the algorithm begins its search for a solution at the
starting point determined by the parameter $start.$'  At each
successive value of $\lambda(t)t,$ the algorithm begins it's search at
the point found in step $t - 1.$  

If the starting point is set to the centroid of the game (this is the
default), and $delLam > 0,$ then this algorithm computes the {\em principal
branch} of the logistic Quantal response equilibrium.  In this case
taking the limit, as $\lambda$ goes to 0, the quantal response equilibrium
defines a unique selection from the set of Nash equilibrium for generic
normal form games.  Similarly, for generic extensive form games, it
defines a unique selection from the set of sequential equilibria.
Therefore, in extensive form games, this algorithm can be used to compute
approximations to a sequential equilibrium.

The following parameters can be set:

\begin{description}
\item[fullGraph:] Determines whether to return the full guantal response 
graph or just the final value  of equilibria to find for each value of
$\lambda$.  Has a default value of 1.

\item[minLam:] Sets the minimum value of $\lambda.$
Default is $lambda = 0.01$.

\item[maxLam:] Sets the maximum value of $\lambda.$  Default is
$\lambda = 30.0.$

\item[delLam:]  The constant, $\delta,$ used in incrementing.   Default is
$\delta = .01.$

\item[plotType:] Determines the exponent, $a,$ used in incrementing
$\lambda.$  {\em Linear} corresponds to setting $a = 0,$ and {\em
logarithmic} corresponds to setting $a = 1.$ Default is logarithmic.

\item[start:] Sets the starting point of the search for the initial value of
$\lambda.$  Default is the centroid, where all strategies are chosen
with equal probability (not implemented yet.)

\item[maxits 1D:] Sets the maximum number of iterations to the
n-dimensional optimization routine.  Default is 20.

\item[maxits nD:] Sets the maximum number of iterations in the
1-dimensional line search.  Default is 100.

\item[tol 1D:] Sets the tolerance for the n-dimensional optimization
routine.  Default is 1.0e-10.

\item[tol nD:] Sets the tolerance for the 1-dimensional line search.
Default is 2.0e-10.

\item[pxifile:] Can be used to generate an output file compatible for
input to pxi, a program for graphical viewing and display of the output.
\end{description}

\section{GobitAll}
Performs a grid search to compute the complete logistic
quantal response correspondence (as described in \cite{McKPal:95a}
for a {\em small} two-person normal form game.  

The algorithm computes approximate fixed points of the correspondence for 
the correspondence for values of $\lambda$ between $minLam$
and $maxLam.$  Starting at $\lambda = minLam,$ $\lambda$ is incremented
according to the formula $$ \lambda(t+1) = \lambda(t) +delLam \lambda(t)^a.
$$ 
where $minLam,$ $maxLam,$ $delLam,$ and $a$ are parameters described
below. For each value of $\lambda$, a grid search is done over all values
of the probabilities for player $1$, evaluated on a grid of mesh 'del p.'
Points are evaluated in terms of the value of an objective function that
measures the distance between the original point, and the best response to
the best response (under the logistic best response function.)  Points
with a value of the objective function less than $Tolerance$ are
approximate fixed points, and are kept, others are discarded.

Limitations:  This algorithm is only implemented for two-person normal
form games, and is very computationally intensive.
  
The following parameters can be set for Grid Solve.

\begin{description}
\item[minLam:] Sets the minimum value of $\lambda.$
Default is $\lambda = .01.$
\item[maxLam:]  Sets the maximum value of $\lambda.$
Default is $\lambda = 3.$ 
\item[delLam:] Specifies the rate at which the value of Lambda changes.
Has a default value of .1.
\item[Plot Type:] Specifies whether to have geometric or linear incrementing.
Default is 0, resulting in $a = 1,$ or !geometric incrementing.
\item[del p:] Grid size for search over probability space.  

\item[pxifile:] Can be used to generate an output file compatible for
input to pxi, a program for graphical viewing and display of the output.
\end{description}

\section{LCP}
For a normal form game, this algorithm searches for equilibria of the
specified normal form game using the Lemke-Howson algorithm, as described
in \cite{LemHow:64}. Eaves \cite{Eav:71} lexicographic rule for
linear complementarity problems is used to avoid cycling.

In the Lemke Howson algorithm equilibria are found by following paths of
``almost'' equilibria, where one relaxes at most one constaint.
Equilibria are thus inter-connected by networks of paths that result when
different of the constraints are relaxed.  One can find the set of
``accessible''  equilibria in such methods by starting at the extraneous
solution and then tracing out this entire network.  See, e. g., Shapley,
xxx.  However, the set of accessible equilibria is not necessarily all
Nash equilibria.

For extensive form games, this algorithm implements Lemke's algorithm on
the ``sequence form'' of the game, as described in \cite{KolMegSte:94}.

Limitations:  This algorithm is fast, but currently only works for two
person games.  \cite{Wilson:1971} and \cite{Rosenmuller:1971} have
suggested ways in which the Lemke-Howson Algorithm can be extended to
general $n$-player games, but these extensions require methods of tracing
the solution to a set of non linear simultaneous equations, and have not
yet been implemented in GAMBIT.

The following parameters can be specified:

\begin{description}
\item[n equi:] Specifies the number of equilibria to find.  If not specified,
the default value is zero, which means that all equilibria reachable by
the algorithm are to be found.
\item[max depth]
Specifies the maximum depth of search.
\end{description}

\section{Liap}
Finds Nash equilibria via the Lyapunov function method
described in \cite{McK:91}.  Works on either the extensive or normal
form.  This algorithm casts the problem as a function minimization
problem by use of a Lyapunov function for Nash equilibria.  This is a
continuously differentiable non negative function whose zeros coincide
with the set of Nash equilibria of the game.  A standard descent
algorithm is used to find a constrained local minimum of the function
from any given starting location.  Since a local minimum need not be a
global minimum (with value 0,) the algorithm is not guaranteed to find
a Nash equilibrium from any fixed starting point.  The algorithm thus
incorporates the capability of restarting.  The algorithm starts from
the initial starting point determined by the parameter 'start'.  If a
Nash equilibrium is not found, it will keep searching from new
randomly chosen starting points until a Nash equilibrium has been
found or the maximum number of tries (parameter 'ntries') is exceeded,
whichever comes first.  For an extensive form game, if the algorithm
converges, it converges to a sequential equilibrium (Andrew Solnick,
personal communication).

Limitations: The disadvantages of this method are that it is generally
slower than any of the above methods, and also,  there can be local minima
to the Liapunov function which are not zeros of the function.  Thus the
algorithm can potentially converge to a non Nash point.  However,
inspection of the objective function can determine if this problem has
occurred.  If the objective function is zero, a Nash equilibrium has been
found. If it is greater than zero, the point is not Nash.  The algorithm
will automatically check this.  If the objective function is larger than the
tolerance, then the point is discarded.

The following parameters can be set;

\begin{description}
\item[start:] Sets the starting profile for the descent algorithm.  The
default is the centroid.
\item[trace:] Sets the print level.  Higher values generate more output.
The default value is 0.
\item[ntries:] Sets the maximum number of attempts at finding each
equilibrium. Default is 10
\item[nequilib:] Sets the number of equilibria to find.  Has a default
value of 1.  
\item[maxits nD:] Sets the maximum number of iterations to the
n-dimensional optimization routine.  Default is 200.
\item[maxits 1D:] Sets the maximum number of iterations in the
1-dimensional line search.  Default is 100.
\item[tol nD:] Sets the tolerance for the n-dimensional optimization
routine.  Default is 1.0e-10.
\item[tol 1D:] Sets the tolerance for the 1-dimensional line search.
Default is 2.0e-10.
\end{description}

\section{EnumPure}
Computation of pure strategy Nash equilibria is done by simple enumeration.
All pure strategies are checked to see if they are Nash equilibria.  

The following parameters can be set. 

\begin{description}
\item[num:] Allows the user to set the maximum number of Nash equilibria to
find.  Has a default value of 1. (not implemented yet) 
\end{description}

\section{SimpDiv}
Computes a Nash equilibrium to a normal form game based
on a simplicial subdivision algorithm.  The algorithm implemented is
that of \cite{VTH:1987}.  The algorithm is a simplicial subdivision
algorithm which can start at any point in the simplex.  The algorithm
starts with a given grid size, follows a path of almost completely labeled
subsimplexes, and converges to a completely labeled sub-simplex that
approximates the solution.  Additional accuracy is obtained by refining
the grid size and restarting from the previously found point.  The idea is
that by restarting at a close approximation to the solution, each
successive increase in accuracy will yield a short path, and hence be
quick.

In its pure form, the algorithm is guaranteed to find at least one
mixed strategy equilibrium to any n-person game.  Experience shows
that the algorithm works well, and is acceptably fast for many
moderate size problems.  But in some examples it can be quite slow.
The reason for this is that sometimes after restarting with a refined
grid size, even though the starting point is a good approximation to
the solution, the algorithm will go to the boundary of the simplex
before converging back to a point that is close to the original
starting point.  When this occurs, each halving of the grid size will
take twice as long to converge as the previous grid size.  If a high
degree of accuracy is required, or if the normal form is large, this
can result in the algorithm taking a long time to converge.

In order to combat the above difficulty, a parameter 'leash' has been
added to the algorithm which places a limit on the distance which the
algorithm can wander from the restart point. (Setting this parameter
to 0 results in no limit, and gives the pure form of the algorithm.)
With this parameter set to a non trivial value, the algorithm is no
longer guaranteed to converge, and setting small values of the
parameter will sometimes yield lack of convergence.  However,
experience shows that values of the parameter on the order of 10
generally do not destroy convergence, and yield much faster
convergence.

Parameters:

\begin{description}
\item[n equilib:] Maximum number of equilibria to find. Default is 1.  
\item[n Restarts:] Number of restarts.  At each restart the mesh of the
triangulation is halved.  So this parameter determines the final mesh
by the formula ${1/2}^{ndivs}$.
\item[Leash:] Sets the leashlength. Default is 0, which results in no
constraint, or no leash.  
\end{description}


