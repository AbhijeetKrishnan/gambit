%
% $Header$
%
% Description: Introduction chapter for GCL manual

\chapter{Introduction}

This document describes the Gambit Command Language (GCL).  Gambit is
a library of computer program that allows one to build, manipulate,
and solve finite extensive and normal form games. The GCL provides a
method of directing the operation of Gambit that is analagous to that
of a high level general purpose programming language.

The purpose of the GCL is to provide a simple, but powerful and
flexible language by which one can perform complicated or repetitive
operations and procedures on games in extensive or normal form.  The
language has facilities for building and editing an extensive or
normal form game, converting back and forth between the extensive and
normal form representations, and solving the resulting games for
various equilibria of interest.  Standard arithmetic, logic, text and
input-output operations are provided, as well as vectorizable
functions which support many vector and matrix operations.  Flow
control statements allow for repetitive operations (such as
investigating games as one changes various parameters) or conditional
operations.  Thus the GCL is also suitable for certain types of
econometric analysis of games.


\section{\rindex{Installation}}
All of the gambit files can be found at the Gambit World Wide Web site
at

\begin{verbatim}
http://www.hss.caltech.edu/~gambit/Gambit.html
\end{verbatim} 

\noindent Instructions on downoading and installation as well as a
list of platforms supported can be obtained there.  Source code is
also available at the same location. 

After you have installed the GCL, you can start it on unix systems by
typing the name of the GCL executable file (usually \verb+gcl+) at the
command line prompt.  On Windows systems, you can start it by double
clicking on its icon.

When the GCL starts, it first looks for the initialization file
\verb+gclini.gcl+, which is a file containing GCL commands, including
commands to load the standard user defined functions that are
documented in this manual.  If the file cannot be found, a warning
will be issued.  Control is then turned over to you, and you will
receive the GCL prompt

\begin{verbatim}
GCL1:= << 
\end{verbatim}

\noindent The interpreter is ready for your first command.  Note that
your prompt may be different from the above depending on settings in
your \verb+gclini.gcl+ file.  

\subsection{Command line editing}\index{Command line!Editing}

You can edit the command line by using the arrow keys. The left and
right arrow key move the cursor, the back space or delete key can be
used to delete characters, and the typing characters will enter the
characters in insert mode. The up and down arrow keys can be used to
recall previous lines.  

\subsection{Meaning of the prompt}\index{Prompt}
\index{Command line!Prompt}

The GCL prompt is of the form \verb+GCLnn:= <<+.  The prompt is
actually part of the GCL command, and can be deleted or modified by
the command line editor if so desired.  The \verb+<<+ part of the
command is the short form of the \verb+Print+ command, which causes
the evaluation of the command line to be echoed to the
console. Backspacing over this part of the command will supress
output.  The \verb+GCLnn:=+ part of the command saves the evaluation
of the command line to the variable \verb+GCLnn+.  Backspacing over
this will prevent saving of the output into a variable.

\section{\rindex{Technical support} and \rindex{bug reports}}

User feedback is an important part of Gambit's development cycle.  Our
design choices have been motivated in large part by our experiences
and by some of the potential uses of the software we have imagined.
But this is by no means complete, and so we are eager to hear from
you, Gambit's users, to find out what features are particularly
useful, and in what areas future releases might be improved.

The authors may be contacted via email at {\tt
gambit@hss.caltech.edu}.  This address will forward a copy of your
message to the development team.  While a personal reply may not
always be possible, we will read all correspondence and apply as many
suggestions as possible to our future releases.

Although we have made every effort to ensure the reliability and
correctness of the code, it is certain that errors of varying degrees
of severity exist.   If you experience problems with the software,
send us email at {\tt gambit@hss.caltech.edu} describing your problem.

When reporting a bug, it is important to include the following
information:

\begin{itemize}
\item the version number
\item the platform(s) on which the problem occurred
\item as complete a description of the circumstances of the problem as
possible, including a sequence of steps which reproduces the problem
\end{itemize}
 
\noindent 
Without this information, it will be difficult for us to
identify the source of the problem to correct it.

At this time, no formal technical support mechanism exists for Gambit.
As time permits, we will make every effort to answer any and all
questions pertaining to the program an its documentation.

We hope you will find Gambit a useful tool for research and
instruction.

