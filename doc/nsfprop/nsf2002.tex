\documentclass[12pt]{article}
\topmargin -.5in
\textheight 9in
\renewcommand{\baselinestretch}{.90}
\begin{document}
\pagenumbering{roman}
\begin{center}

\section{\bf  PROJECT SUMMARY} 

\bigskip
Computer Software for Solution of Extensive Form Games\\ 

\bigskip
Richard D. McKelvey \\
Andrew McLennan \\
Ted Turocy \\
\end{center}

Under this grant, we will update the Gambit software to include
several improvements and enhancements that are necessary to update the
software from the previous NSF support of the project.  We will
change the existing quantal response equilibrium (QRE) algorithm to be
based on a true homotopy algorithm, and introduce new homotopy based
algorithms for computation of proper equilibrium and the tracing
procedure.  We will add support for symmetric games, which will allow
for solution of substantially larger games in gambit, and we will
improve the graphics user interface and other graphics capabilities of
gambit.

\newpage
\tableofcontents

\newpage
\pagenumbering{arabic}
\begin{center}
\section{{\bf  PROJECT DESCRIPTION}}
\end{center}

\subsection{Introduction}

This grant proposes to update the Gambit software to include a number
of improvements and enhancements that have been accumulating since the
completion of the previous NSF support of the Gambit Project.  The
Gambit Project is a project for development of public domain computer
software for solution and analysis of extensive and normal form games.
The Gambit Project was funded for about five years by NSF under grants
\#SBR-9208863 and \#SBR-9617854 to the California Institute of
Technology, and \#SBR-9308862 to the University of Minnesota. During
this time, we extensively developed this software.  The software
(source code, as well as executable versions of the code for a number
of popular platforms), is available on the World Wide Web at the
following URL:

\bigskip
\noindent
\verb+        http://www.hss.caltech.edu/gambit+

\bigskip
\noindent
Referees of this proposal are encouraged to download a copy of Gambit
and try it out.  

We are requesting one year of funding, during which time we hope to
bring the project up to date, by improving some of the existing
algorithms and adding some new features.

\subsection{Results From Prior NSF Support} 

We begin by describing our accomplishments under the prior grant, NSF
Grant \#SBR-9617854, ``Computer Software for Solution of Extensive
Form Games,'' Richard McKelvey in collaboration with Andrew McLennan.

\bigskip
Under this grant, which was a continuation of grants \#SBR-9208863,
and \#SBR-9308862, we developed software for the solution of extensive
and normal form games. Unlike most NSF grants, the main output of this
project was computer software rather than publications. The software
is referred to as the ``Gambit'' software and consists of two parts, a
Graphics User Interface (GUI), and the Gambit Command Language
(GCL). The GUI is an interactive graphics based program for building
and solving finite extensive and normal form games. The GCL is a
Mathematica style language which supports user written programs and is
suitable for econometric and computationally intensive applications.

The Gambit software is public domain, and the source code and executables
are freely available at the Gambit web site\footnote{{\tt
www.hss.caltech.edu/gambit}}. The Gambit software is widely
used around the world and has become the standard game theory software used
in teaching and research. The most recently released version of the software
now implements the polynomial solver (developed by McLennan) for the
computation of all Nash equilibria of (moderately sized) extensive and
normal form games.

As part of the gambit project, we have implemented algorithms for
computation of the QRE correspondence for arbitrary extensive and normal
form games. Also, the GCL supports user programming for doing econometric
analysis of data gathered from experiments for those games. 

\bigskip
\centerline{ PUBLICATIONS AND WORKING PAPERS}
\begin{description}
\item[1] McKelvey, R. D., and Andrew McLennan, ``The maximal number of
regular totally mixed equilibria,'' \textit{Journal of Economic
Theory}, 72:411-425, 1997.

\item[2] McKelvey, R. D., and Andrew McLennan, ``Computation of
equilibria in finite games,'' in \textit{Handbook of Computational
Economics}, Amman, H. M, D. A. Kendrick, and J. Rust, eds, (Amsterdam:
Elevier, 1996)

\item[3] McKelvey, R. D., ``A Liapunov function for computation of
Nash equilibria'' \textit{Caltech Working paper \#} (1998).

\item[4] McKelvey, R. D., A. McLennan, and T. Turocy, ``Gambit
Command Language,'' version 0.96.3, \textit{postscript document}
distributed with software at the Gambit web site$^*$, (1997-2000).

\item[5] McKelvey, R. D., A. McLennan, and T. Turocy, ``Gambit
Graphics User Interface,'' version 0.96.3, \textit{interactive online
manual} distributed with software at the Gambit web site$^*$, (2000).

\item[6] McLennan, A, ``The Expected Number of Nash Equilibria of a
Normal Form Game,'' \textit{working paper}, University of Minnesota
(March 1999)
\end{description}


\subsubsection{Current status of Gambit Project}

The Gambit Project software consists of a library of computer
programs, written in $C^{++}$, for performing basic operations on
extensive and normal form games, together with two user interfaces to
the underlying programs --- the Gambit Graphics User Interface (GUI),
and the Gambit Control Language (GCL).

\bigskip
\noindent
{\em The GUI}

\smallskip
\noindent
The Gambit Graphics User Interface (GUI) is a menu driven program for
interactive building and solution of extensive form games.  A manual
as well as an executable version can be down-loaded from the Gambit
Web site.

The Gambit GUI displays on the computer screen a graphics representation
of any extensive form game.  The user is able to navigate around the
extensive form using commands from the keyboard.  A series of editing
options allows one to alter the current game tree by adding, inserting,
changing, or deleting portions of the extensive form.  After the extensive
form has been built, the program can solve the game for various
equilibria, and display them on the original graphical representation of
the game.  

\bigskip
\noindent
{\em The GCL}

\smallskip
\noindent
The Gambit Control Language (GCL) is a high level computer language for
building and solving games in extensive or normal form.   A manual as
well as an executable version of the GCL can be down-loaded from the
Gambit Web site.  

The GCL is an implicitly typed language (meaning that data types are
inferred from context), which has data types corresponding to various
parts of extensive and normal form games.  One can define variables to
refer to and change various parts of a game.  The language has flow
control statements and user defined functions which can be used to
perform repetitive or recursive operations on games.  In addition, it
supports vector and list operations. The language is especially
suitable for applications, such as econometric estimation on games, or
investigation of the functional form of game theoretic solutions,
where one must repetitively solve similar games.  It also provides a
language which can be used to define and build multi-stage extensive
form games.  The language is suitable for use either in an interactive
or a batch mode.

\subsection{Proposed Software Development}

Since the completion of the previous NSF grant, we have received a
number of suggestions from users of possible improvements to the
software.  In addition, we have discovered bugs in the current code,
and have a number of enhancements that we want to make to the code.
Consequently, we plan to modify the existing code and add several new
features to bring it up to date.  The particualr projects which we
would like to complete follow:

\subsubsection{Quantal Response Algorithms} 

To get the code up and running, we took shortcuts in various places.
Consequently, some of the algorithms that are currently used in our
code are unsatisfactory for various reasons. One of the most glaring
examples of this is in our code to compute the logistic quantal
response equilibirum QRE corresponcence.

The logistic QRE has become one of the most heavily used parts of the
Gambit code, as it is used as an econometric method to analyse data
from game theory experiments.  Unfortunately, the current
implementation of the QRE algorithm has problems.  The QRE
correspondence is generically a one dimensional manifold in the
cartesian product of the space of behavior profiles with the real line
(representing the precision, $\lambda$, of the error).  Starting from
a point $(p_0, \lambda_0)$ on the manifold, we currently trace the
correspondence by incrementing $\lambda_0$ to $\lambda_1$, and using
the current profile, $p_0$ as the starting point for a standard
minimization algorithm to find a new point $(p_1,\lambda_1)$ on the
manifold.  This procedure is not the correct way to do this, and will
fail if the QRE correspondence ``loops back''.  Our code contains
tests for whether the computed equilibirum really is a valid QRE
solution.  So the above does not result in incorrect solutions.
However, it does result in the Gambit code not being able to compute
solutions in cases where one exists.  

A second problem with the QRE code is that it suffers from numerical
instability for large values of $\lambda$.  This is because formulae
for the logistic QRE involve taking exponents of expected utility
differences weighted by $\lambda$.  So when $\lambda$ gets moderately
high, this computation results in floating point overflow.  The values
of $\lambda$ that lead to numerical problems are sometimes in the
range of the values of $\lambda$ that would be estimated from
experimental data.  So in this case, the logistic QRE algorithm is not
able to correctly compute the QRE estimates for the required values of
the parameters in the data.

The correct way to fix the first problem above is to re-implement the
path following using a true homotopy algorithm in the product space
(of $\lambda$ and $p$).  This will allow the algorithm to turn
corners, and backtrack in $\lambda$ space.  Recently Ted Turocy
[2001]. has discovered a way to also avoid the second problem of
numerical instability.  He has already experimented with the revised,
homotopy based algorithm that incorporates both of the above changes,
and found that it solves the problems of our current implementation,
allowing for computation of the QRE correspondence to values in the
millions, as opposed to values below one hundred (the current
capability).  So we plan to implement both of these changes to the the
computation of the logistic QRE correspondence.  We will implement
similar changes for the extensive form (AQRE) version of the QRE.

\subsubsection{Other Homotopy Based Algorithms}
As part of the work to resolve the problems with the QRE algorithm, we
need to perfect a homotopy algorithm.  There are several other
problems that we will be able to solve once we have a good homotopy
based algorithm.  

\begin{itemize} 
\item
We currently have no means in Gambit of finding {\it proper
equilibria}, as defined by Myerson [1978].  Yamomoto [1993] proposes a
homotopy based algorithm for finding proper equilibria, and we plan to
introduce code to implement this algorithm.
\item
Harasany and Selten propose the {\it tracing procedure} as a method
for finding a unique selection from the set of Nash equilibria.  This
procedure is based on following a homotopy.  We plan to implement an
algorithm to compute the tracing procedure.
\item
Govindan and Wilson have proposed algorithms for finding multiple
equilibria which offer very good performance in terms of speed.  These
algorithms also require a homotopy to implement, and we plan to
implement this algorithm as well.
\item
Finally, in the Gambit code developed by McLennan for finding all
equilibria of $n$-person games, we have been investigating a homotopy
based approach.  This method involves starting with a system where we
can find the maximum number of solutions for a system of polynomial
equations that is guaranteed by Bernstein's theorem.  We then
transform it to the game of interest, following all of the solutions,
but just keeping those that are real and satisfy the required
constraints.  We have attempted implementation of such an algorithm,
but have had problems in finding suitable homotopy code that will work
within gambit.  Once we perfect our own homotopy code, we anticipate
that we will be able to successfully implement this algorithm as well.
\end{itemize}

We do not currently have any means of finding either proper equilibria
or the tracing procedure in Gambit.  So adding these features will add
increased functionality to Gambit.  The second two items will
hopefully lead to more efficient and hence faster algorithms for
finding multiple equilibria of $n$-person games.

\subsubsection{Symmetric and Anonymous Games}

One class of games that is of particular interest in experimental work
is the class of symmetric and anonymous games.  For example, much
experimental literature on auctions, voting, and public goods
provision deals with this class of games.

Since a symmetric game is just a game with restrictions on the set of
possible strategies or payoffs that can arise, symmetric and anonymous
games can currently be represented and solved by Gambit.  However the
computational complexity of solving such games grows exponentially
with the number of players and strategies.  Since games of
experimental interest frequently have either many players or many
strategies, the strategy set is frequently much too large to be
computationally feasible with our current algorithms.

We plan to develop data structures and algorithms that will take
explicit account of symmetries in the game to represent and solve them
more efficiently.  There are two types of savings that are possible.
First, the amount of space and time to retrieve the payoff functions
can be improved substantially if the underlying data structures take
explicit account of symmetries of the game.  Second, the algorithms
can take advantage of symmetries in the game in that not all profiles
need be investigated to determine if a profile is an equilibrium.
Finally, if one is only concerned with finding symmetric solutions to
a game, then the algorithms can take further account of the symmetries
to drastically reduce the dimensionality of the problem.  For example,
to determine if a profile is a symmetric Nash equilibrium for a
symmetric game, one can start at a symmetric profile, and only
investigate deviations by one player.  So if each of $n$ players each
have $k$ strategies available, a problem that starts out being $n k$
dimensional when one does not take account of the symmetry becomes
$2k$ dimensional (note independence from $n$) when one takes account
of the symmetry.  Some of the solution algorithms, in particular those
based on function minimization (QRE and Liapunov) could be relatively
easily modified to take advantage of the lower dimensionality.  Others
(the linear complementarity based algorithms) would first require
theoretical advances (which we plan to pursue) to take advantage of
lower dimensionality.

We anticipate that incorporating the above capabilities to deal with
symmetric games will enable Gambit to solve games of the size that are
encountered in experimental work, and hence make the code practical
for use in such settings.  

\subsubsection{Improved Graphics user Interface}

We plan to make improvements to the graphics user interface in Gambit
to make it more intuitive, and to provide better means for visualizing
and displaying various aspects of solutions.  Towards these goals, we
plan the following changes

\begin{itemize}
\item
Integrate the PXI into Gambit. The {\it PXI} is an external plotting
program that allows for viewing and plotting of output from certain
algorithms in gambit.  It is currently most useful in conjunction with
the QRE algorithm, as it allows for the plotting of the entire QRE
correspondence, and also allows for the plotting of data from
experiments on the same graph as that containing the QRE
correspondence.  We plan to considerably enhance the capabilities of
the PXI program and to integrate it into Gambit, so that it is more
generally useful for to visualize solutions from any algorithm, and
for viewing experimental data corresponding to the games.  We want to
add the capability of ploting other items of game theoretic interest,
such as best reply correspondences.
\item
We plan to add support for evolutionary games and adaptive learning
models.  Here, we plan to allow the capability to view the gradient
field corresponding to the replicator dynamics, or other learning
dynamics, and to be able to view the orbits that are traced out by the
dynamic, or to view an animation of the dynamics by viewing the
motion, over time, of a grid of points in the simplotpe. This will
require additional features in the PXI to enable these displays.
\item
We plan to improve profile viewing and editing capabilities, Rather
than having the profile displayed as a vector, it will be displayed as
a tree, which has the same form as the tree of information sets in the
extensive form game.  Thus, it will be much more intuitive to see
which branch is associated with which probability. This will make it
easier to compare and edit profiles.
\item
Finally, we plan to modernize the graphics user interface to make it
consistent with current standards.  This will involve having context
dependent menus and help, intuitive drag and drop capabilities, and
navigation of the extensive form through scroll bars and highlighting
of the relevant element rather than via a cursor.
\end{itemize}

\subsection{Theoretical work}

While the main thrust of our proposal is to complete the software
enhancements to Gambit described above, in the course of our efforts
theoretical issues arise.  For example, since one of our main goals
under the previous grant was to develop algorithms to find {\it all}
Nash equilibria, we were naturally led to the question of the
computational complexity of this effort, and to the closely related
question of how many Nash equilibria there might be.  We solved this
problem in the papers (McKelvey, McLennan, 1995, and McLennan, 1999).
There we find an exact formula for the maximal number of totally mixed
equilibria, and for the average number of Nash equilibria for a game,
and show that the number of such equilibria is exponential both in the
number of players and the size of strategy sets.  Berg and McLennan
[2001] applies the fomula from McLennan [1999] to two player games,
finding that the mean number of equilibria grows exponentially as the
size of both players' pure strategy sets goes to infinity.  Similarly, in
investigating the problems with the numerical instability of the QRE
algorithm, one of us (Turocy [2001]) has been led to discover numerically
stable ways of computation of the correspondence, and to find other
properties of the correspondence, linking it to the replicator
dynamics.  We plan to continue to pursue theoretical questions such
as these as they arise.

\newpage
\begin{center}
\section{\bf  REFERENCES }
\end{center}

\begin{description}
\item
Berg, J., and A. McLennan, ``The Asymptotic Expected Number of Nash
Equilibria of Two Player Normal  Form Games,''  {\it mimeo},
University of Minnesota, 2001.

\item
Govindan, S., and R. Wilson, ``A Global Newton Method to Compute Nash
Equilibria,'' {\it mimeo}, Stanford University, 2000.  

\item
Govindan, S., and R. Wilson, ``Computing Nash Equilibria by Iterated
Polymatrix Approximation,'' {\it mimeo}, Stanford University, 2000.

\item
Harsanyi, J. C., and R. Selten, {\it A General Theory of Equilibrium
Selection in Games}, (MIT press: Cambridge, 1988).

\item
McKelvey, R. D., and A. McLennan, ``Computation of Equilibria in
Finite Games,'' Chapter 2 in {\it Handbook of Computational
Economics}, Edited by H. Amman, D. A. Kendrick and J. Rust,
(Amsterdam: Elsevier, 1996).

\item
McKelvey, R. D., and A. McLennan, ``The Maximal Number of Regular
Totally Mixed Nash Equilibria,'' {\it Journal of Economic Theory},
forthcoming, 1996b.

\item
McKelvey, R. D., and T. R. Palfrey, ``Quantal Response Equilibria for
Normal form Games,'' {\it Games and Economic Behavior}, 10 (1995):
6-38.

\item
McKelvey, R. D., and T. R. Palfrey, ``Quantal Response Equilibria for
Extensive form Games,'' {\it Experimental Economics}, 1 (1998): 9-41.

\item 
McLennan, A., ``The Maximal Generic Number of Pure Nash Equilibria,''
{\it Journal of Economic Theory}, forthcoming, 1996.

\item
McLennan, A, ``The Expected Number of Nash Equilibria of a
Normal Form Game,'' {\it working paper}, University of Minnesota
(March 1999)

\item
Myerson, R. B., ``Refinements of the Nash equilibrium concept,'' {\it
International Journal of Game Theory}, 7 (1978): 73-80.

\item
Turocy, T. L., ``Computing the Logistic Quantal Response Equilibrium
Correspondence,'' {\it working paper}, Department of Economics, Texas
A \& M University, 2001.

\item
Van Damme, E., {\it Refinements of the Nash Equilibrium Concept},
(Springer Verlag, 1983).

\item
Yamomoto, Y., ``A Path-Following Procedure to Find a Proper
Equilibrium of Finite Games'', {\it International Journal of Game
Theory}, 22 (1993): 249-259.
\end{description}

\newpage
\begin{center}
\section{BIOGRAPHICAL SKETCHES}
\end{center}
\subsection{RICHARD D. McKELVEY}

\noindent
{\bf PERSONAL} 

\ \\
\begin{tabular}{ll}
Date of birth:&April 27, 1944\\
Marital Status:&Married, three children.\\
Office address:&Division of the Humanities and Social Sciences \\
&California Institute of Technology \\
&Pasadena, California 91125 \\
&(818) 356-4091 \\
\end{tabular}

\ \\
\noindent
{\bf EDUCATION}
\begin{description}
\item
B.A., (Mathematics) Oberlin College, 1966.
\item
M.A., (Mathematics) Washington University in St. Louis, 1967.
\item
M.A., (Political Science) University of Rochester, 1970.
\item
Ph.D.,(Political Science) University of Rochester, 1971.
\end{description}

\noindent
{\bf ACADEMIC APPOINTMENTS}
\begin{description}
\item
Professor, California Institute of Technology; 1979-present 
\item
Sherman Fairchild Distinguished Scholar, California Institute of
Technology; 1978-79 
\item 
Associate Professor, Carnegie-Mellon University; 1977-79
\item
Assistant Professor, Carnegie-Mellon University; 1974-77
\item
Assistant Professor, University of Rochester; 1972-74 
\item
Instructor, University of Rochester; 1970-72
\end{description}

\noindent
{\bf RESEARCH GRANTS} Principal investigator on the following NSF Grants.
\begin{description}
\item
(1973-1974) --- ``Mathematical Analyses of Voting 
Procedures,'' with Richard Niemi
\item
(1974-1975) --- ``Non-equilibrium Allocations in 
Spatial Models of Policy Formation''
\item
(1977-1978) --- ``Agenda Design in Formal Models of 
Voting''
\item
(1977-1982) --- ``A Theoretical and Experimental 
Analysis of Committee Coalition Processes,'' with Peter C. Ordeshook
\item
(1982-1985) --- ``Information in Elections:  
Theory and Experiments''
\item
(1984-1989) --- ``Collaborative Research on
Information and Political Processes,'' with Peter C. Ordeshook
\item
(1985-1988) --- ``Common Knowledge in Models of
Information Aggregation and Learning,'' with R. Talbot Page
\item
(1990-1993) --- ``Strategic Learning in Games of
Incomplete Information,'' with Thomas R. Palfrey.
\item
(1990-1993) --- ``Full Information Dynamic Models of Policy Formation
in Legislative Systems,'' with Raymond Riezman.
\end{description}

\noindent
{\bf PUBLICATIONS:} Over 60 publications in refereed journals or volumes
in economics ({\it Econometrica, JET, JME, RES}), political science ({\it
APSR, AJPS, JOP}),  social choice ({\it Social Choice and Welfare, Public
Choice}), and applied math and game theory ({\it International Journal of
Game Theory, JASA, SIAM, Math in Operations Research, Management Science}).

\ \\
\noindent
{\bf Five publications most closely related to the project}
\begin{description}

\item
``An Experimental Study of the Centipede Game,'' with Thomas R. Palfrey,
{\it Econometrica}, 60, (1992): 803-836.
\item
``The Maximal Number of Regular Totally Mixed Equilibria,'' with
A. McLennan, {\it Journal of Economic Theory}, 72:411-425, 1997.
\item
``Quantal Response Equilibria for Normal Form Games.''  {\em Games and
Economic Behavior} 10, (1995):6-38.
\item
McKelvey, R. D., and A. McLennan, ``Computation of Equilibria in
Finite Games,'' in {\it Handbook of Computational
Economics}, Edited by H. Amman, D. A. Kendrick and J. Rust,
(Amsterdam: Elsevier, 1996).
\item
``Quantal Response Equilibria for Extensive form Games,'' with
T. Palfrey, {\it Experimental Economics}, 1 (1998): 9-41.
\end{description}

\noindent
{\bf Five other significant publications}
\begin{description}
\item
``General Conditions for Global Intransitivities in Formal Voting 
Models.''  {\em Econometrica} 47 (1979):1085-1112.
\item
``Covering, Dominance, and Institution-Free Properties of Social 
Choice.''  {\em American Journal of Political Science} 30 (1986):283-314.
\item
``Generalized Symmetry Conditions at a Core Point,'' with Norman 
Schofield.  {\em Econometrica}, 55 (1987):923-934.
\item
``Elections with Limited Information:  A Fulfilled Expectations Model 
using Contemporaneous Poll and Endorsement Data as Information 
Sources,'' with Peter C. Ordeshook.  {\em Journal of Economic Theory} 
36 (1985):55-85.
\item
``Common Knowledge, Consensus, and Aggregate Information,'' with
Talbot R. Page, {\it Econometrica}, 54 (1986): 109-127.

\end{description}



\noindent
{\bf Collaborators within the last 48 months:} Serena Guarnaschelli,
Andrew McLennan, Thomas Palfrey, John Patty.

\ \\
\noindent
{\bf Graduate Advisor:} Peter Ordeshook

\ \\
\noindent
{\bf Postdoctoral Advisor:} none

\newpage
\subsection{ANDREW McLENNAN}

\noindent
{\bf PERSONAL} 

\ \\
\begin{tabular}{ll}
Date of birth:&March 7, 1954\\
Marital Status:&Divorced, one child.\\
Office address:&Department of Economics \\
&University of Minnesota \\
&1035 Management and Economics \\
&271 19th Avenue South \\
&Minneapolis, MN 55455 \\
&(612) 625-7504 \\
\end{tabular}

\ \\
\noindent
{\bf EDUCATION}
\begin{description}
\item
B.A., (Mathematics) University of Chicago, 1976.
\item
Ph.D.,(Economics) Princeton University, 1982.
\end{description}

\noindent
{\bf ACADEMIC APPOINTMENTS}
\begin{description}
\item
Professor, University of Minnesota; 2000 - present. 
\item
Associate Professor, University of Minnesota; 1987 - 2000. 
\item
Postdoctoral Fellow, Mathematical Science Research Institute, January -
June 1986.
\item
Assistant Professor, Cornell University, Department of
Economics, 1985 - 1987.
\item
Visiting Assistant Professor, Cornell University,
Department of Economics, 1984-85 academic year.
\item
Visiting Assistant Professor, University of Minnesota,
Department of               Economics, 1981-82 academic year.
\item
Assistant Professor, University of Toronto, Department
of Political Economy (later the Department of Economics), 
1980 - 1985 (on leave 1981-82 and 1984-85 academic years).
\item
Research Assistant, Resources for the Future, 
Washington, D.C., Summer of 1977.
\end{description}

\noindent
{\bf RESEARCH GRANTS} 
\begin{description}
\item
Social Sciences and Humanities Research Council of Canada
Postdoctoral Fellowship, 1982 - 1983, renewed for 1983 - 1984.
\item
Postdoctoral Fellowship, Mathematical Sciences Research Institute,
funded by NSF Grants MCS-8120790 and SES-8420114.
\item
Single Quarter Leave, Fall Quarter 1991.
\end{description}


\ \\
\noindent
{\bf Five publications most closely related to the project}
\begin{description}

\item
``The Maximal Number of Regular Totally Mixed Equilibria,'' with
R. McKelvey, {\it Journal of Economic Theory}, 72:411-425, 1997.

\item
McKelvey, R. D., and A. McLennan, ``Computation of Equilibria in
Finite Games,'' in {\it Handbook of Computational
Economics}, Edited by H. Amman, D. A. Kendrick and J. Rust,
(Amsterdam: Elsevier, 1996).

\item
``Consistent Conditional Systems in
Non-cooperative Game Theory'', {\it International Journal of Game
Theory}, {\bf 18} (1989), 141-174.

\item
``Generic Finiteness of Equilibrium Outcome Distributions in Game
Forms,'' in {\it Econometrica}, {\bf 69} (2001), 455-471. (Joint with
Srihari Govindan)

\item
``The Mean Number of Real Roots of a Multihomogeneous System of
Polynomial Equations'' forthcoming in the {\it American Journal of
Mathematics} (2002).




\end{description}

\noindent
{\bf Five other significant publications}
\begin{description}
\item
``Fixed Points of Contractible Valued
Correspondences'', {\it International Journal of game Theory},
{\bf 18} (1989), 175-184.

\item
``Sequential Bargaining as a Non-cooperative
Foundation for Walrasian Equilibrium,'' \break {\it Econometrica},
{\bf 59}, (1991), 1395--1424.  (Joint with Hugo Sonnenschein)


\item
``Justifiable Beliefs in Sequential Equilibrium,'' {\it
Econometrica}, {\bf 53} (1985), 889-904.

\item
``Stationary Markov Equilibria,'' {\it Econometrica}, {\bf 62},
(1994), 745-781.  (Joint with D.~Duffie, J.~Geanakoplos, and
A.~Mas-Colell)

\item
``Price Diversity and Incomplete Learning in the Long Run,'' {\it Journal
of Economic Dynamics and Control}, {\bf 7} (1984), 331-347.


\end{description}



\noindent
{\bf Collaborators within the last 48 months} 

Srihari Govindan, Richard McKelvey, In-Uck Park, and Hulya Eraslan

\ \\
\noindent
{\bf Graduate Advisor:} Hugo Sonnenschein

\ \\
\noindent
{\bf Postdoctoral Advisor:} none

\newpage
\subsection{THEODORE TUROCY}

\noindent
{\bf PERSONAL} 

\ \\
\begin{tabular}{ll}
Office address:&Department of Economics \\
&Texas A \& M \\
&College Station, TX 77843 \\
\end{tabular}

\ \\
\noindent
{\bf EDUCATION}
\begin{description}
\item
B.S. with honors (Engeneering and Applied Science/Economics)
California Institute of Technology, 1994.
\item
Ph.D.,(Economics) Northwestern University (MEDS), August 2001.
\end{description}

\noindent
{\bf ACADEMIC APPOINTMENTS}
\begin{description}
\item
Assistant Professor, Texas A \& M; 2001 - present. 
\item
Lecturer, Nortwestern University, Kellogg Graduate School of
Management, Fall 1998-99, Fall 1999-2000, Winter 2000-01.
\end{description}

\ \\
\noindent
{\bf Five publications most closely related to the project}
\begin{description}
\item
``Computing the Logistic Quantal Response Equilibrium
Correspondence,'' in {\it working paper}, Department of Economics,
Texas A \& M, College Station, TX (2001).
\item
``Game Theory,'' with Bernhard von Stengel, {\it Encyclopedia of
Information Systems}, Academic Press, forthcoming.
\item 
``Gambit: A Computer System for Constructing and Analysing Finite
Extensive and Normal Form Games,'' with R. D. McKelvey, A. McLennan
{\it et. al.}  Available at {\tt www.hss.caltech.edu/gambit} 1997-2000.
\item 
``Gambit Command Language, version 0.96.3'', with R. D. McKelvey and
A. McLennan {\it postscript document} distributed with software at the
Gambit web site: {\tt www.hss.caltech.edu/gambit}, 2000.
\item 
``Gambit Graphics User Interface, version 0.96.3,'' with
R. D. McKelvey and A. McLennan {\it interactive online manual}
distributed with software at the Gambit web site: {\tt
www.hss.caltech.edu/gambit}, 2000.
\end{description}

\noindent
{\bf Five other significant publications}
\begin{description}
\item 
``Intertemporal Speculation Under Uncertain Future Demand:
Experimental Results,'' with Charles Plott, in {\it Understanding
Strategic Interaction: Essays in Honor of Rienhard Selten}, Albers
{\it et. al.,} editors (Berlin: Springewr Verlag, 1996)
\item
``Implications of Approximate Solution Concepts in Sealed-Bid
Auctions,'' {\it working paper}, Northwestern University, 2001.  
\item
``Game Theory at the Ballpark: An Analysis of Attempting Stolen Bases
in Baseball,'' {\it working paper,} Northwestern University, July
2000.
\item
``Markets Online: Practical Experience and Directions for Research,''
with U. Mukherjea {\it working paper,}  Northwestern University, 2000.
\end{description}



\noindent
{\bf Collaborators within the last 48 months} 

Richard McKelvey, Andrew McLennan, Urmi Mukherjea, Charles Plott,
Bernhard von Stengel.

\ \\
\noindent
{\bf Graduate Advisor:} Mark Satterthwaite

\ \\
\noindent
{\bf Postdoctoral Advisor:} none

\newpage
\begin{center}
\section{BUDGET}
\end{center}
\newpage
\addtocounter{page}{3}
\subsection{Budget Justification}

\newpage
\begin{center}
\section{CURRENT AND PENDING SUPPORT}
\end{center}

\end{document}

