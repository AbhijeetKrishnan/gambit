%
% FILE: guiman.tex -- source for GUI manual
%
% $Id$
%
\documentstyle[a4,makeindex,verbatim,texhelp,fancyhea,mysober,mytitle]{report}%
\input psbox.tex
%\newcommand{\bd}{\begin{description}}
%\newcommand{\ed}{\end{description}}
\parskip=10pt%
\parindent=0pt%
\itemsep{0pt}
\helpfontsize{11}
\title{Gambit Graphics User Interface:\\
$$\image{1cm;0cm}{gamblogo}$$\\
\centerline{An Interactive Extensive Form Game Program}}
\author{Developed by: Richard D. McKelvey\\
Main Programmer: Theodore Turocy\\
Front End: Eugene Grayver\\
 \\
California Institute of Technology\\ \today.\\
Version 0.94\\
 \\
Part of the Gambit Project:\\
Richard D. McKelvey and Andrew McLennan, PI's\\
Funding provided by the National Science Foundation\\
 \\}

\makeindex%
\begin{document}%
\maketitle%

%\pagestyle{fancyplain}
\bibliographystyle{plain}
%\pagenumbering{roman}
%\setheader{{\it CONTENTS}}{}{}{}{}{{\it CONTENTS}}
%\setfooter{\thepage}{}{}{}{}{\thepage}
\tableofcontents%

\chapter*{Copyright notice}%
%\setheader{{\it COPYRIGHT}}{}{}{}{}{{\it COPYRIGHT}}%
%\setfooter{\thepage}{}{}{}{}{\thepage}

Copyright (c) 1995, The Gambit Project, at California Institute of
Technology and University of Minnesota.  

Permission to use, copy, modify, and distribute this software and its
documentation for any purpose is hereby granted without fee, provided that the
above copyright notice and this permission notice appear in
all copies of this software and related documentation.

THE SOFTWARE IS PROVIDED ``AS-IS'' AND WITHOUT WARRANTY OF ANY KIND, EXPRESS, IMPLIED OR 
OTHERWISE, INCLUDING WITHOUT LIMITATION, ANY WARRANTY OF MERCHANTABILITY OR FITNESS FOR A 
PARTICULAR PURPOSE.
 
IN NO EVENT SHALL THE GAMBIT PROJECT, THE CALIFORNIA INSTITUTE OF TECHNOLOGY, THE UNIVERSITY 
OF MINNESOTA, OR ANYONE ASSOCIATED WITH THE DEVELOPMENT OF COMPUTER SOFTWARE UNDER THE GAMBIT 
PROJECT, BE LIABLE FOR ANY SPECIAL, INCIDENTAL, INDIRECT OR CONSEQUENTIAL DAMAGES OF ANY KIND, 
OR ANY DAMAGES WHATSOEVER RESULTING FROM LOSS OF USE, DATA OR PROFITS, WHETHER OR NOT ADVISED 
OF THE POSSIBILITY OF DAMAGE, AND ON ANY THEORY OF LIABILITY, ARISING OUT OF OR IN CONNECTION 
WITH THE USE OR PERFORMANCE OF THIS SOFTWARE.


\chapter*{Acknowledgments}%
%\setheader{{\it ACKNOWLEDGEMENTS}}{}{}{}{}{{\it ACKNOWLEDGEMENTS}}%
%\setfooter{\thepage}{}{}{}{}{\thepage}

The Gambit Project is a project for the development of computer code for the 
solution of extensive and normal form games.  The software developed under 
the project is public domain.  

The Gambit Project is funded in part by 
National Science Foundation grants SBR-9308637 to the California Institute 
of Technology and SBR-9308862 to the University of Minnesota.  

Numerous students at Caltech and the University of Minnesota have contributed
 to the Gambit Project:  Among these are Bruce Bell,  Anand Chelian, Nelson 
Escobar, Todd Kaplan, Brian Trotter, and Gary Wu. 

\chapter{Introduction} 
Gambit is a library of computer programs, written
in \verb$C++$, for building, analyzing, and solving n-person games, in
either extensive or normal form.

The Gambit Graphics User Interface (\popref{Gambit GUI}{guigloss}) is an
interactive, menu driven program for accessing this library of programs.
The Gambit GUI allows for the interactive building and solving of
extensive and normal form games.  It consists of two separate modules, one
for extensive and the other for normal form games.   In each module, you
can build, save load and solve a game, and convert back and forth from one
form of game to the other.

Despite it's ease of use, the Gambit GUI is not suitable for repetitive or
computer intensive  operations.  A separate program, the Gambit Command
Language (\popref{GCL}{gclgloss}) is designed to be used for such operations 
on games.  The
GCL is a program which allows access to the functionality of the gambit
library of programs through a high level programming language.  The
Gambit GUI and GCL  are compatible with each other, in the sense that they
each generate files that can be read by the other, and they call the same
underlying library of functions.


\chapter{Hardware and Installation}
\section{Platforms}
The current version of gambit \popref{GUI}{guigloss}\ relies on the wxWin 
library.  The following platforms are currently supported by wxWin:
\begin{itemize}
\item IBM PC and compatible.  Gambit will run on any IBM PC compatible machine
equipped with MS Windows 3.1 and above.   A port to Win32 is in the works.
\item Sun workstations.  Gambit is supported on most Sun workstations running
either SUN-OS or Solaris.  Either MOTIF or XView toolkits are required.
\item IBM RS/6000. Gambit is supported on IBM RS/6000 machines running motif.
\item Hewlett Packard workstations.  Supported using the Motif1.2 toolkit.
\item Linux.  A version supporting Linux Motif will be available shortly. 
Contact the authors for additional information.
\item Macintosh. A Macintosh version of the wxWin library is in the works.  As
soon as it becomes available, GAMBIT will be ported to that platform as well.
\item OS/2.  An OS/2 version of the wxWin library is in the works.  As
soon as it becomes available, GAMBIT will be ported to that platform as well.
Until then, Gambit GUI will run in a windows box under OS/2.
\item Others.  wxWin library should work on most UNIX-based platforms.  Contact
the authors for additional information.
\end{itemize}
\section{Installation}
All of the gambit files can be found at the Gambit World Wide Web site
at 

\begin{verbatim}
http://hss.caltech.edu/~gambit/Gambit.html
\end{verbatim}

You will need a Web browser such as Netscape to download the files.  Follow the 
instructions there for installation.  
 
\chapter{Representation of Games in GAMBIT}

This chapter describes some notation and concepts from game theory
that are necessary to understand how Gambit represents and solves
games.  This chapter assumes a basic understanding of game
theory.  Definitions of the main terms are given intuitively in the
Glossary.

\section{The \index{extensive form}{Extensive Form}}\label{extformsec}

The extensive form is a detailed description of a sequence of decisions 
that must be made by a group of individuals. To illustrate the basic ideas,
 Figure~\ref{fig_samp1} 
shows how the extensive form of a simple 
\popref{two player poker game}{pokergloss} (from \cite{Mye:91} is represented
 in the Gambit GUI.  

\begin{figure}\label{fig_samp1}
$$\image{1cm;0cm}{poker_ef}$$
\caption{GAMBIT Display of Extensive Form of
		a Simple Two Player Poker Game}\label{fig_samp1}
\end{figure}

In Gambit, the extensive form \popref{game tree}{toptreegloss}, is drawn with the 
\popref{root node}{rootnodegloss} being the node furthest to the left,
branches going from left to right, and successive 
nodes to the right of 
their predecessors.  So nodes further to the right represent decisions 
that occur later in time.  Each player, (including chance) is represented by a color.  

A \popref{node}{nodegloss} is represented as a 
horizontal line.  Every non-terminal node is a \popref{decision node}{decnodegloss} 
and is represented by the color of the player who makes a decision at that node.  
Nodes are connected by \popref{branches}{branchgloss}, which are represented by two 
connected line segments, the ``fork'' and the ``tine''.  The fork is used in 
the graphics display to indicate graphically the probabilities of taking each 
branch.  The tine is used as a location at which to display textual information 
about the branch.  In gambit \popref{Outcomes}{outcomegloss} can 
be attached to either non terminal or terminal nodes and can 
be displayed either by label or by payoff next to the node to 
which they are attached.  

\popref{Information sets}{infosetgloss} in Gambit are identified by a 
pair of numbers, the first representing the player number and the 
second representing the information set number.  This information 
can be displayed on the extensive form next to each node to identify 
which nodes are in which information sets.  
Whenever possible information sets are also represented 
by vertical lines connecting the nodes that are members of the same information set.  
It is only possible to do this when nodes 
in the same information set are at the same level of the tree.  

\subsection{Labeling}
There are three locations at which each node can be labeled, (above, 
below, and to the right) and two at which each branch can be 
\index{label}{labeled} (above and below.) You can choose which information to display 
in the various positions by selecting the information in the \helpref{Display Legend}{displaymenu}  menu.  

\subsection{Numbering}\label{infosetnumbering}
In order  to associate strategy profiles in the 
normal form with the corresponding behavioral strategies in the extensive 
form, (and vice versa) you will need to understand how the branches and 
information sets are numbered.  The numbering of branches and information 
sets in the extensive form  is used to number pure strategies in the normal form.  

The \index{numbering}{numbering} of the \index{branches}{branches} is always
from top to bottom.  Thus, the highest branch at any node is always branch
1, the next branch is branch 2, etc. An \popref{action}{actiongloss} consists of 
the set of branches in an information set with the same branch number.  

In Gambit, each \popref{information set}{infosetgloss} has a unique 
information set ID, consisting of the player followed by the information 
set number.  Information sets for each player are numbered consecutively, 
in \popref{indexed traversal order}{indextravgloss}.  
The information set ID can be displayed at the decision nodes by selecting 
it in the Options Legend menu.  

\section{\index{strategies}{Strategies} and The \index{normal form}{Normal Form}}

A \index{pure strategy}{pure strategy} for a player is a function that 
associates with each of 
its information sets, a chosen action at that information set.  A pure 
strategy $n$-tuple is a vector of strategies, one for each player.  
The normal form for a game is a mapping which associates with each pure 
strategy $n$--tuple, an expected payoff to each player from 
playing that strategy. 

In the extensive form of the game in GAMBIT, the information sets for a 
player are numbered from \verb+1+ to \verb+J+, where \verb+J+ is the number of information 
sets that player \verb+i+ has.  We denote the pure strategy in which player \verb+i+ 
adopts strategy \verb+a+ in its first information set, \verb+b+ in its second, and 
\verb+c+ in its third \verb+abc+.  So if player \verb+i+ has three information sets, where 
the first two information sets contain three branches and the last contains 
two branches, then player \verb+i+ has a total of \verb+18+ strategies, 
and \verb+312+ indicates the strategy where the player chooses the third 
branch at its first information set, the first branch at its second 
information set, and the second branch at its third information set.  

In the extensive form of \helpref{Figure~\ref{fig_samp1}}{extformsec}, 
Player 1 has two information sets, each with two choices, so it has a total of four 
strategies.  They are {\tt 11, 12, 21 } and {\tt 22}.  Player 2 has one 
information set, with two branches, so they are labeled {\tt 1} and {\tt 2}.  
The strategy {\tt 21} for player 1 represents the strategy of
choosing FOLD with a Red card, and RAISE with a Black card.  
 
\subsection{Wildcard notation}
To represent a collection of strategies which differ only in the choice at
a particular information set, we use a ``\index{wildcard}{wildcard}'' 
notation.  For example,
if a player has three choices at its second information set, then the 
notation {\tt 3*2} is used to represent the collection of strategies, 
{\tt \{312, 322, 332\}}.  

The {\tt *} notation in GAMBIT serves the same purpose as the ``wildcard'' 
character in DOS file specifications.  To get a good appreciation for how 
the wildcard works, you should exit GAMBIT and WINDOWS, go to the root 
directory of your hard drive, and type \verb+del *.*[RET]+.

\subsection{\index{normal form}{Normal Form}}\label{normformsec}
The normal form of a game is a mapping which associates with 
each \popref{strategy profile}{stratprofgloss} a vector of
 payoffs for the players.  The normal form will thus be an $n$ dimensional 
matrix, with each cell containing a vector of length $n$.

Every extensive form game has an associated normal form.  The payoff for 
a given player under a particular strategy is computed as the sum of the 
\popref{realization probabilities}{realprobgloss} of each node times the 
value to the player of any outcome at that node.  The payoffs in the normal 
form are simply the expected payoff to each player from the given strategy
 profile.  

\begin{figure}
$$\image{1cm;0cm}{poker_nf}$$
\caption{GAMBIT Graphics Display of Normal Form of
		Simple Two Player Poker Game}\label{fig_samp2}
\end{figure}

Figure~\ref{fig_samp2} gives the normal form for the extensive form game 
of poker illustrated in 
\helpref{Figure~\ref{fig_samp1}}{extformsec}. 

In the game of  \helpref{Figure~\ref{fig_samp1}}{extformsec}, with the 
strategy profile {\tt (12, 1)}, the realization probability of the terminal 
node on the path RED, RAISE, MEET, with a payoff of $(\$2.00, -\$2.00)$ 
is $1/2$, and the realization probability of terminal node on the path 
BLACK, FOLD, with a payoff of  $(-\$1.00, \$1.00)$ is $1/2$.    All other 
terminal nodes have realization probability of $0$ at this strategy profile.  
Taking expected values, this gives a payoff of  
$(\$0.50, -\$0.50)$, which is the entry in the normal form for this cell.  

\section{Types of Games}
\subsection{\index{Perfect Recall}{perfect recall}}\label{perfrecallsec}
A game of perfect recall is a game in which individuals do not forget what
they have known or done earlier in the game.  This requires that
information sets must refine earlier information sets, and if there is
an information set, one of whose nodes follows a choice by the same player
at a previous information set, then all of the members of that information
set must follow the same choice.

GAMBIT does not enforce perfect recall on the extensive form games that
you build.  In fact, GAMBIT will solve games without perfect recall for
 optimal
mixed strategies.  The problem comes in converting the mixed strategy back
to a behavioral strategy.  If the game has perfect recall, then by Kuhn's
theorem (see eg., \cite{vanDamme:1983}), any mixed strategy can be
converted back to a behavioral strategy which is realization equivalent.  
If the game does not have perfect recall, this is not always possible.  If
a game without perfect recall is solved, and there is no realization
equivalent behavioral strategy, you will be warned that there is a
problem. [Note -- this option is not yet implemented].  
 

\subsection{Games of \index{Incomplete Information}{incomplete information}}\label{incinfsec}
Games of incomplete information are games in which different players have 
different beliefs about the underlying parameters (such as the utility 
functions, strategy sets, or number of players) of the game.  The standard 
way of treating such games is given in \cite{Harsanyi:1967}, where it
is shown that 
 such games were equivalent to games in which players have some common
prior distribution over the joint distribution of characteristics, and
 then individuals observe their own type, but not that of the other players 
in the game.  

Games of incomplete information can be modeled in GAMBIT having an
initial chance move determine the distribution of types, and then
defining information sets so each individual observes only its own
type.

\chapter{Using GAMBIT GUI}

\begin{figure}
$$\image{3cm;3cm}{biggamb}$$
\caption{Gambit GUI showing normal and extensive
forms}\label{fig_biggamb}
\end{figure}

The Gambit GUI consists of two self-contained modules for representing,
displaying and manipulating games.  A game can be viewed in either the
extensive or normal forms.  Although independent, the modules are
seamlessly integrated and it is possible to solve the game in the normal
form while viewing it in the extensive.  It is also possible to go from
the extensive form representation to the normal form (but not vice versa
at the present).   You can select the module to be used in the File menu,
in one of two ways.

\begin{itemize}
\item Build a new game by selecting File, New , and then choosing either 
Extensive or Normal.  The toolbar can be used to speed up frequently used
commands. 
\item  Load an existing game by selecting File, Open, and then selecting 
a file that contains a previously saved game in either extensive or normal form.
\end{itemize}


\section{File menu}
You can load a new game or load a previously saved game from the file 
menu.  When you save an extensive or normal form game in Gambit, it is 
written in a standard format..  Files containing extensive form games are
given the default extension of \verb+.efg+, and files containing normal
form games are given the default extension \verb+.nfg+ .  When you select
File Open, you will get a file loading dialog box from which you can
either directly enter the name of the file, or browse the directory system
for files with the extensions \verb+.efg+ or \verb+.nfg+.  If you select a
normal form file, you will bring up the normal form module of the GUI.  If
you select an extensive form file, you will bring up the extensive form
module of the GUI.

\section{Data Types}
Gambit allows you to do computations in either floating point (double) or
exact arithmetic (rational).  You must select the data type whenever you
load a game from an external file or create a new game.  If you want to
change to a different data type, use the {\tt Solve->DataType} menu.  There
are two main reasons for changing data types: 1) storing all games as
rationals to preserve {\em exact} payoff values and 2) changing to doubles
to execute some algorithms that are not implemented for rationals.
However, note that saving a rational file and
reloading it as double could lose information.  (For example, 1/3 would
be converted to its nearest decimal representation.)

\begin{description}
\item[Double:]  If you select Double, the game will be represented in double precision floating
 point numbers, which on most machines results in about 15 digits of accuracy.  The benefit of 
floating point calculations is speed, since floating point numbers are designed to fit in a fixed 
amount of storage, and arithmetic operations are coded in hardware.  However,  in some games, 
floating point calculations can result in roundoff errors that will lead to either incorrect 
solutions, or failure to find solutions that exist.  
\item[Rational:]  If you select Rational, the game will be represented in rational numbers, 
which are represented internally as the quotient of two arbitrary precision integers.  Doing 
calculations in rationals will guarantee that the answer is exact, at least for those algorithms 
that support rational calculations.  However calculations in rationals are slower than double 
calculations, by one or two orders of magnitude.  Currently the only algorithms supporting 
calculations in rationals are the following two person algorithms: LpSolve, LcpSolve, and 
EnumPure and EnumMixed.  
\end{description}

\section{Normal Form GUI}

\begin{figure}
$$\image{0cm;0cm}{nfg}$$
\caption{A 2 player normal form}\label{fig_nfg}
\end{figure}

\begin{figure}
$$\image{6cm;2cm}{nfgtools}$$
\caption{NF GUI toolbar}\label{fig_nfgtools}
\end{figure}

In the normal form representation, the game is viewed as a 2-dimensional
window into an N-dimensional matrix.  Each 2D window shows a table of
payoffs for each player as a function of the strategies for two of the
players, holding the strategies of all of the other players (if there are
any) fixed.  Note that this organization requires each game to have at least
two players.  Each cell contains the payoff vector with one entry per
player for the strategy profile determined by this cell.  The profile
itself is set by using a combination of settings for the row and column
players plus the strategy settings for the rest of the players.  {\em The
payoff vector can be edited by double-clicking on the cell and entering
new values in the dialog. To enter an entire game matrix, an accelerator
key may be useful.  Pressing the {\tt TAB} key will call up the payoff
edit dialog and automatically move the cursor to the next cell}

\subsection{Row and Column Players} The row and column player choice boxes
determine which players' strategies get displayed on the horizontal and
vertical axis of the matrix.  Each box can contain a value from 1 to the
total number of players in the game.  The dimensions of the matrix are
determined by the number of strategies for the row and column players.
Note that it is meaningless to select the same player for both the row and
the column.  If the game has more than two players, a warning will be
issued and no action will be taken.  In the case of a two player game, the
row and the column players will be switched.

\subsection{Strategy Profile}
The array of choice boxes labeled "Profile" at all times reflects the
strategies picked by each player to achieve the payoffs shown in the
highlighted cell.  The nth choice box can contain a value from 1 to the
total number of strategies the nth player has.  When one of the choice
boxes is changed, one of two things can happen:
\begin{enumerate}
\item If the choice box number is not equal to either the row or
the column player, the entire matrix will be updated with new values to
reflect the new 2D view into the matrix.
\item If the choice box number was either the row or the column player,
the highlighted cell will move to reflect the new strategy.
\end{enumerate}

\subsection{Normal Form GUI Display}
In order to accommodate as many different platforms and tastes as possible,
the display features of the normal form matrix are highly configurable.  A
large number of these configuration features are common to all the 'table'
displays in Gambit.  Refer to \helprefn{Table Window}{TableWindow} section
of this manual for a detailed explanation of those features.  Another set
of display options can be accessed through the 
\helpref{Display->Features}{nffeaturessec} menu.

\subsubsection{Normal Form Features}\label{nffeaturessec}
The features dialog allows the choice to display or not to display extra
normal form data.  This data is obtained from running various solution
algorithms.  Three data sets may be available depending on the solutions
run on the game.  Any of them can be turned off through this dialog.
\begin{enumerate}
\item Strategy probability data.  Many normal form solution
algorithms return solutions in the form of mixed or pure strategy
equilibria.  If such an algorithm was run, an extra row and column will be
added to the matrix to show the calculated probability of row and column
players choosing the respective strategies.  The value in the lower left
cell is the probability of players other than row and column choosing
their strategies to achieve this profile.
\item Value to player data.  This displays for each strategy the expected 
payoff for that strategy under the current solution.  
\item Strategy dominance data.  This data is generated by running
the Eliminate Dominant strategies algorithm.  For each strategy of row and
column players, it will show by which, if any, strategy, that strategy is
dominated.  For more information see the discussion of the ElimDom
algorithm.
\end{enumerate}

\subsection{Normal Form Solutions}\label{NormalFormSolutions}
A large number of different \helpref{algorithms}{SolutionAlgorithms} are 
available for solving normal form
games.  In order to cater to experienced users familiar with the 
properties of individual algorithms, almost every relevant parameter can
be edited through a variety of solution related dialog boxes.  To allow 
simpler access to the basic features provided by the algorithms, a set of
commonly used {\em standard} solution types has been included with Gambit.
The user need only determine the general features of the desired equilibria
and Gambit will take care of all the details.

All of
the settings are saved to a defaults file every time they are changed. There
is no need to enter the custom parameters before each recalculation, even
through multiple sessions, unless a change is required.
After selecting the solution type/parameters, the algorithm must be started
by selecting the Solve->Solve menu or pressing the 
\begin{figure}
\image{.5cm;.5cm}{solvtool}
\end{figure}
 button on the toolbar.

If the current game was generated from an extensive form game, and if no
changes have been made to either game, there will exist a link between the
two forms.  In this case, the {\em Extensive form} checkbox will be
enabled.  Checking this box will cause all generated solutions (if
applicable to the solution type) to be converted back into behavior
profiles and projected back to the extensive form window.  The solutions
can then be examined in the extensive form display.

\subsubsection{NFG Standard Solutions}\label{NFGStandardSolutions}
The standard solutions are selected by the Solve->Standard menu.  The
particular settings are then picked in the dialog box.
The following standard solutions are provided with Gambit.

\begin{tabular}{|l|p{2cm}|r|}
\row{Type & Number & Internal Settings}\hline
\row{Nash & One & LCP for 2 person, SimpDiv for n-person; Maximum of 1 solutions;
		Eliminate all weakly dominated strategies before solving.}
\row{Nash & Two & EnumMixed for 2 person, Liap for n-person; Maximum of 2 solutions;
		Eliminate all strongly dominated strategies before solving.}
\row{Nash & All & EnumMixed for 2 person, Liap for n-person {\em not guaranteed};
		No maximum number of solutions,
		Eliminate all strongly dominated strategies before solving.}
\row{Perfect & One & LCP for 2 person, Not implemented for n-person; Maximum of 1 solutions;
		Eliminate all strongly dominated strategies before solving. }
\row{Perfect & Two & Same as above with maximum of 2 solutions}
\row{Perfect & All & Same as above with no maximum number of solutions} \hline
\end{tabular}

To have individual control over each of the parameters mentioned above, see
the \helpref{Custom Solutions}{NFGCustomSolutions} section.

\subsubsection{NFG Custom Solutions}\label{NFGCustomSolutions}
Only a user familiar with these algorithms should use the custom settings
since the program makes no attempt to check for the validity of some of
the parameters.  For a detailed discussion of the 
\helpref{algorithms}{SolutionAlgorithms} see the appropriate sections.

All of the custom features are selected in the Solve->Custom submenu.  

Each solution is controlled by two groups of settings
\begin{description}
\item[Algorithm] The actual algorithm used to calculate the equilibria.  Not all algorithms
are applicable to all games.  If an algorithm is not available for the
currently loaded normal form, it will be grayed out in the dialog.  After an
algorithm has been selected, the selection may be saved by pressing 
the {\tt Save} button, or an additional level or parameters can be accessed by
pressing the {\tt Params} button.
\item[Dominance] Prior to running the algorithm, dominated strategies may be eliminated, thus
speeding up the consequent calculations.  The feature is toggled by 
the {\tt ElimDom before solve} checkbox.  Either a single step of elimination,
or a iterative elimination is selected by the {\tt Iterative Eliminate} 
checkbox.  The dominance type (Strong/Weak) and method (Pure/Mixed) are
 controlled by the radioboxes.
\end{description}

Note that some algorithms require a starting point.  Depending on the
options selected in the custom parameters dialog for these algorithms,
a starting point dialog may pop up before the algorithm is run.  The
default (centroid) is usually the best place to start.  Double clicking on
the first profile in the window and pressing {\tt Ok} will select the
default and allow the algorithm to continue. Advanced users may choose to
use values obtained by solving the game using a different algorithm as a
starting point.

\subsubsection{Normal Solutions Inspect}\label{NormalSolutionInspect}
\begin{figure}
$$\image{3cm;3cm}{nfgsoln}$$
\caption{Normal Form Solution Inspect Window}\label{fig_nfgsoln}
\end{figure}


The solution inspection window is a very powerful and complex part of the GUI.
The basic functionality consists of displaying the MixedProfiles that contain the
probability data for the solutions.  The entries are arranged with each 
player on a separate row, and strategies in consecutive columns.  
Each cell consists of the strategy number (or name), followed by a colon, and followed
by the probability of that strategy.
If the {\em display zero prob} option is not turned on, strategies with zero probability
will not be displayed. 

A variety of data dependent on the generating algorithm are available for most
 MixedProfiles. Not every category is relevant for every profile.  If a
certain data member is either not relevant or not available, it will be 
displayed as {\tt ----}.  The display of each data member is toggled in the
Options dialog.  Note that the ID data member is always displayed (it is
an increasing (not necessarily consecutive) index used to identify each
solution)

The number of the solution that is currently selected will be highlighted in the first
column.  The currently selected solution will also be displayed in the NF window.  
To change the current solution (and thus change the display in the 
corresponding NF window), double click (control-click in XView) on the \# of the
desired solution.  To quickly browse the solutions, the 
{\em Update Solutions Dynamically} 
option may be used.  With this option on, the solution will change automatically once
the cursor moves to a different solution \#.  Double clicking on the first
row deselects the current solution.

Solutions may be removed by pressing the {\tt Remove} button.  A solution may
be edited, or a new one created by pressing the corresponding button.  Note 
that an edited or a newly created solution will have {\em User} as the
Creator and will have most data members undefined.  Double clicking on any
of the strategy values will automatically call up the solution edit dialog.
To facilitate working with a large number of solutions, 
the \helpref{Sort/Filter}{SolnSortFilt} functionality is available.

If the underlying Normal Form game was generated from an Extensive Form game, 
and if the link between the two is still 
valid, the {\em NF->EF} button will be enabled.  To project the solution to the 
extensive form, and immediately display it there, press this button.

Note that all solution windows will be deleted if any changes are 
made to the underlying NF.

\normalbox{For configuration and output features of this dialog see
the generic \helprefn{Table Window}{TableWindow} description.}

\subsubsection{Sorting and Filtering Solutions}\label{SolnSortFilt}
$$\image{1cm;0cm}{sortfilt}$$
To facilitate work with a large number of solutions, Gambit implements solution
sorting and filtering.  These features are accessed by the {\tt Sort/Filter} button
in the solution inspect window.
[Double clicking on the first row will sort the solutions by the data member
that is displayed in the column clicked in].  Deselecting an item in the
{\tt Filter By} box will cause the solutions that satisfy that item not to
be displayed.  (I.e. to filter out all NASH equilibria, deselect the {\tt Yes} item
in the {\tt Nash} listbox; to display just solutions generated by the {\tt Gobit}
 algorithm, deselect all items except for Gobit in the {\tt Creator} listbox.
Note that the filter option may lead to some confusion if newly generated 
solutions are automatically filtered out.


\subsubsection{Elimination of Dominated Strategies}\label{ElimDom}
\begin{figure}
$$\image{3cm;3cm}{elimdomd}$$
\caption{Dominance elimination dialog}\label{fig_elimdomd}
\end{figure}

The Supports->ElimDom menu can be used to compute dominated 
strategies for a normal or extensive form game, and return a new support 
consisting of undominated strategies.  
The following parameters can 
be selected:
\begin{description}
\item[DomType:]  You can select whether to eliminate weakly dominated 
strategies or strongly dominated strategies.
\item[Method:] You can select whether to eliminate only strategies that are
dominated by a single (pure) strategy or by a combination of some strategies
(mixed).
\item[Iterative Eliminate:] Selecting this option will iteratively eliminate dominated strategies. The 
result will be a nested sequence of supports, with the last one consisting of 
strategies that are undominated in the final support.  
\item[Players:] You may select which players to eliminate dominated strategies for. The 
default is all players, meaning that dominated strategies for each player are computed 
over the initial support, and then eliminated simultaneously to create the new support.  
\item[Compress:] Selects whether to set the current support to the undominated 
strategies returned by ElimDom.  
\end{description}

\subsubsection{Examining NF Supports}\label{ElimDomInspect}
For a detailed explanation of a {\em Support} see the appropriate theory section.
Supports can be generated either by the 
\helpref{elimination of dominated strategies}{ElimDom}  algorithm, 
or explicitly by pressing the \helpref{New Support}{NewSupport} button 
in this dialog.  Two supports are defined for every NF window:
\begin{enumerate}
\item The displayed support determines the dimensions of the NF that is actually
displayed in the window.
\item The current support determines what support will be used for all the solution
computation and elimination of dominated strategies. 
\end{enumerate}
By default, these supports are identical.  However, an advanced user may wish to
display one support while working on another.  Solution display may become 
confusing in this case, as strategy probabilities will no longer add to one.  

The dimensionality of the supports is displayed in the text box above the selection
choicebox.

\subsubsection{Creating NF Supports}\label{NewSupport}
A new support starts with the default of the {\em full support}.  That is, for each player,
all the strategies are included in the support.  To deselect a player's strategy, click on it
in the appropriate listbox.  The strategy will no longer be hilighted.  At least one strategy
must remain selected in each listbox.  Upon returning to the 
\helpref{examination}{ElimDomInspect} dialog, the newly created support will be added
to the support list.

\subsection{Default Accelerator Keys}\label{NormFormDefAccl}
To speed up some of the more commonly used commands in the GUI, a set of
{\em accelerator keys} may be defined.  Once defined, a combination of keys
can be used to activate most of the menus and buttons.  Accelerator key
definitions are accessed through the Prefs->Accels menu.  Accelerator key
definitions are saved in the main Gambit defaults file (gambit.ini).  Gambit
is distributed with the following keys defined.
\begin{description}
\item[TAB] The TAB key can be used to rapidly enter or modify a normal form.
Pressing TAB will call up the payoff values dialog, and automatically move
the cursor to the next cell once the dialog is dismissed.  Thus the entire
game can be entered by repeatedly pressing TAB and entering the values --
no mouse operations are required.
\end{description}

%---------------------------------- EFG -------------------------
\section{Extensive Form GUI}

\begin{figure}
$$\image{0cm;0cm}{efg}$$
\caption{A 3 player extensive form}\label{fig_efg}
\end{figure}

\begin{figure}
$$\image{6cm;2cm}{efgtools}$$
\caption{EF GUI toolbar}\label{fig_efgtools}
\end{figure}

In the extensive form, the game is represented as a topological tree.  The 
section on representation of the \helpref{Extensive Form}{extformsec} 
has a more detailed explanation of this form.
Compared to the Normal Form, the Extensive Form interface is much richer
and thus considerably more complex.  A major portion of the functionality
is devoted to the tree building.  Another set of functions deals with
customizing the display, and yet another set of functions takes care of
the solutions and their display.  

\subsection{Navigating the Extensive Form}
The cursor is indicated on the display of the normal form by a 
solid or flashing dark line above 
one of the nodes or by a dark triangle around a subgame icon. 
The cursor position is used by many of the tree building functions.  
You can move the cursor around the extensive form game 
either by use of the arrow keys or by using the mouse.  

When the extensive form 
game becomes large, only a window into the extensive form game is displayed.  
As you move the cursor around the game tree, the window will redraw 
the game tree when necessary to keep the cursor visible. 

For large games, the whole game tree will not be visible at one time.  If you 
want to see more of the game tree, you can change the magnification by 
using the Display->Zoom menu.  You can also use the \verb+-+ and 
\verb&+& keys to change the magnification by constant increments. When 
editing a large game, if the magnification level is too small, you may 
not be able to read the textual information displayed on the game tree.  You 
can use the {\em Zoom Window}, which can be toggled on and off in the Display 
menu to get a magnified view of the tree at the cursor location.  

\subsection{Tree Building Functions}
This section assumes a working knowledge of the conventions used in the
GAMBIT extensive form display (discussed in the 
\helpref{Extensive Form}{extformsec} section). Note that any changes to
the tree structure will invalidate all existing solutions.  The solutions
list will be cleared and solutions removed.

All of the tree editing functions are grouped under the {\tt Build} top
level menu.  The {\tt Build} menu may also be called up by clicking the
{\em right} mouse button causing the menu to pop up at
the current mouse cursor position.
\begin{figure}
\image{0cm;0cm}{popup}
\caption{Popup build menu}
\end{figure}

Many of the editing functions can be executed
by drag-and-dropping with the mouse (read the following sections to see
which functions have this).
\begin{figure}
\image{0cm;0cm}{dragdrop}
\caption{Sample Drag n' Drop operations}
\end{figure}

Most of the labeling information can be
quickly changed by double-clicking on the text of the label and entering
the data in the dialog that pops up.

\subsubsection{Node Menu}
\begin{description}
\item[Add:] This is the most commonly used tree building
function.  Each node is determined by the player it belongs to and the
infoset it belongs to.  There are two ways to add a node: either by
choosing the number of branches the node is to have and thus create a new
infoset, or by choosing an existing infoset this node will belong to.  To
add a node you must first select a player.  The player can be either
chance or an existing player, or a "New Player."  If New Player is chosen,
a new player will be created and given a default name "Player \#." After
selecting the player, you must decide which of the two methods described
above is to be used.  If a new infoset is to be created, just enter the
number of branches desired.  If this node is to belong to an existing
infoset, choose the desired infoset from the Iset choicebox.  If the node
created was a CHANCE node, you will then be prompted for the probabilities
associated with each branch.
\item[Delete:]  When a node is deleted, one of its children (if any
exist) will replace it.  You will be asked to select a branch.  This determines which 
node replaces it.  The other children, and all of their descendants will be destroyed.  
Note that you can not delete the ROOT node.
\item[Insert:] This is analogous to Add Node, except that it can be used at a 
non terminal node.  The node is inserted into the tree at that point, and the portion 
the tree that was previously at that node is moved to the first child 
of the inserted node. 
\item[Label:]  Each node can have a label associated with it. 
This function allows the entering or modification of this label.  The
display of these labels is controlled in the Legends Dialog.
\item[Set Mark:]  The GUI allows you to mark (memorize) one
node for later use.  The marked node is required in some tree operations. 
The marked node indicated by a small circular token 
on the game tree.  Note that drag-and-drop mouse operations make the
node-mark unnecessary.
\item[Goto Mark:]  Moves the cursor to the mark.  This can be useful in a 
very large game to quickly move from one part of the game to another. 
\end{description}

\subsubsection{Actions Menu}
\begin{description}
\item[Delete:]  The selected action will be removed from all information sets 
that it belongs to, all descendants of the nodes at these branches are also removed. 
\item[Insert:]  Inserts a new action at the selected information set.  An branch 
will be added to every node that belongs to the same information set as the 
selected node. 
\item[Label:]  Each action in an infoset can have a label. 
This allows the setting or changing of this label. 
\item[Probs:]  This option is only valid when the cursor is on a
CHANCE node.  It allows the explicit setting of each action probability.
\end{description}

\subsubsection{Tree Menu}
\begin{description}
\item[Copy] Copies the part of the tree following the mark to the 
position of the cursor.  Note that the cursor must be at a terminal node.  No 
new information sets are created by this operation.  Thus, you may have to edit the 
information sets after this operation to achieve the desired change. 
{\em Drag-And-Drop:} The copy function can be achieved by clicking on and
dragging the source node to the target node while holding down the {\tt Control}
 key.  The cursor will change to a small 'tree' icon with the word 'copy'
to indicate that a copy operation is in progress.  Releasing the mouse while
the cursor is not on any terminal node will cancel the operation.
\item[Move] Moves the part of the game tree following the mark to the position of 
the cursor.  
{\em Drag-And-Drop:} The move function can be achieved by clicking on and
dragging the source node to the target node. The cursor will change to a small
 'tree' icon with the word 'move'
to indicate that a move operation is in progress.  Releasing the mouse while
the cursor is not on any terminal node will cancel the operation.
\item[Delete]  The node at the cursor and all its descendants will be deleted.
\item[Label]  The label pertains to the game as a whole and will be
displayed on the titlebar of the window.
\item[Players]  Each player must have a name.  Although not required, it
is highly recommended that these names be unique.  When a player is first
created, it is given a default name "Player \#."  This dialog allows you to
change these default names to something appropriate to the game.
\end{description}

\subsubsection{Infoset Menu}
\begin{description}
\item[Merge:] Merges the information set at the cursor into the information set at 
the mark.  Both information sets must have the same number of actions.  Note
that you can not merge infosets across different subgames (unmark subgames first).
{\em Drag-And-Drop:} The merge operation can be achieved by clicking on an
infoset marker (small circle) at the base of a node and dragging it to the
infoset marker of the target node.  
\item[Break:] Removes the node at the cursor from its information set, creating a new 
information set with just that node in it. 
\item[Join:] Removes the node at the cursor from its information set, and puts it into 
the information set at the mark.  
\item[Split:] Splits the infoset with all of the nodes above the cursor node
remaining in the same infoset, while all of the nodes below the cursor node
are moved to a newly created infoset.
{\em Drag-And-Drop:} The split operation can be achieved by moving the mouse
cursor on top of the line connecting the nodes in the infoset while holding
down the {\tt Shift} key.  When the mouse cursor is on top of the infoset
line, it will change to 'scissors.'  Pressing the mouse button while the cursor
is 'scissors' will split the infoset.
\item[Label:] Assigns a textual name to the information set
\item[Player:]  This allows changing the player that has the choice
at this node.  You can choose any player except the one that is currently
selected.  If player names are being displayed on the tree, double clicking
on the name will also activate this operation.
\item[Reveal:] Reveals the information set to selected players.
\end{description}

\subsubsection{Outcomes GUI}
\begin{figure}
$$\image{0cm;0cm}{outcomes}$$
\caption{Sample outcome windows}\label{fig_outcomes}
\end{figure}

The outcomes dialog arranges the payoff data for the game in one of two
formats: a compact matrix form with each row containing an entire outcome 
vector with one entry per player, or a long entry form with each player
payoff taking up an entire row.  The first form is better suited for games
with simple outcomes (constants) while the second form works better for games
with polynomial outcomes.  The display form is changed through the {\tt Opt}
 button.

If an entry is blank, it is assumed to be 0.  The last column
contains the name of the outcome.  If no name is given, a default of
"Outcome \#" will be assigned.  When first started, the dialog will consist
of one blank row.  As the greatest row is filled in, a new empty row will
be created below it.  You can thus create any number of outcomes.  An
entry can be modified by moving the 'cursor' to it and typing in the
desired number.

An outcome is automatically saved/modified every time the cursor moves to
a different row.  Thus, you can use an outcome right after entering it and
moving the cursor the next row.

To attach an outcome to a node, position the 'cursor' in the outcomes
dialog on any cell in the row that contains the desired outcome vector.
Position the cursor in the extensive form display window on the node an
outcome is to be assigned to.  Press the {\em Attach} button in the
outcomes dialog.  The extensive form display will update to reflect the
changes (assuming outcomes are being displayed).  Outcomes attached to
non-terminal nodes get added to all of the terminal note outcomes descendant
from that node.

An outcome may be detached from a node by pressing the {\em Detach} button
while the cursor is on the desired node.  

An outcome may be deleted entirely by pressing the {\em Delete} button.  Note
that that outcome is automatically detached from all of the nodes it was
attached to previously.


\subsection{Subgames Menu}\label{SubgamesMenu}
Gambit implements and supports the concept of a game theoretical
subgame. The 
conditions necessary for a subtree to be qualified as a subgame, see
the 
theory section of this document.  When an extensive form is first
created or 
read in from a file, no subgames (except for the default ROOT subgame) are 
defined.  If a node is a subgame root, a small triangle is drawn at
its base. 
If a subgame is collapsed, all of the nodes it contains are not displayed but 
replaced by a subgame icon.  Shift-clicking
(holding down the {\em SHIFT} key while clicking the left mouse button) on a
subgame root node will toggle the state of that subgame (expanded vs. collapsed).
Refer to the following picture for an example of subgame structures. 


\begin{figure}
$$\image{0cm;0cm}{subgame}$$
\caption{Simple efg with nontrivial subgames}\label{fig_subgames}
\end{figure}

All of the solution algorithms (with the current exception of the 
Gobit algorithms) make use of the marked subgames.  When non-trivial 
subgames are defined, then the solution algorithms will solve the 
extensive form game by recursion through the marked subgames.  Thus, a 
subgame is solved only when all subgames following it have been 
solved.  

If all subgames are marked, then any Nash equilibrium found will be a 
subgame perfect Nash equilibrium.  
Note that the solution algorithms only 
respect marked subgames, and hence if you want the solution 
algorithms to make use of subgames, you must mark the subgames.  
When an algorithm encounters a subgame, the subgame is solved by itself and
the rest of the tree is resolved for each of the subgame solutions.  Clearly,
the total number of solutions grows exponentially with the number of subgames.
To control the total number of solutions, two parameters are available for
all solution algorithms: maximum number of solutions per subgame, and the
maximum number of total solutions. For advanced
users, the Solve->Custom->Algorithm dialog provides the option to select
which solutions are used for each subgame.  This feature lets the advanced user
concentrate on solutions of interest while keeping the total number of solutions
small.

The following list describes the commands for working with subgames. 
\begin{description}
\item[Mark All:] Uses a built in algorithm to scan the entire tree and mark
those nodes that are root nodes of subgames.  Note that the subgame roots
now become marked by small rectangles.
\item[Mark:] Check whether the cursor node is a subgame root, and mark 
it if it is.  If the cursor node is 
not a root of a valid subgame, a warning is issued. 
\item[Un-mark All:] Unmarks all subgames, consequently expanding all
subgames.
\item[Un-mark:] If the cursor node is a marked subgame, that subgame is removed
from the list of subgames and is no longer considered either for display or
solution purposes.  If the cursor node is not a subgame root, a warning is
issued.
\item[Collapse Level:] If the cursor node is a subgame root, that subgame is 
collapsed and all of the nodes contained in it are replaced with a subgame 
icon.  The cursor changes shape to reflect this change.
\item[Collapse All:] Applies {\em Collapse} to every node that is marked as a 
subgame root, including the ROOT node.
\item[Expand Level:] If the cursor node is a subgame root, that subgame is expanded
and all of the nodes, up to the first descendant collapsed subgame,
are displayed.
\item[Expand Branch:] If the cursor node is a subgame root, all of the
descendant subgames are expanded.
\item[Expand All:] Applies {\em Expand} to every node that is marked as a 
subgame root.
\end{description}


\subsubsection{Prefs Menu}\label{displaymenu}
The display menu allows you to load, set and save various parameters 
which affect the extensive form display.  
\begin{description}
\item[Zoom:]  Sets the magnification level.
\item[Zoom Window:]  Toggles on and off a window in which you can view the 
portion of the game tree near the cursor without any magnification.  
\item[Display:] This allows you to change parameters affecting the way in which the 
game tree is drawn, such as the length of the nodes, branches, and 
vertical spacing between nodes, and also allows for setting whether to display 
information sets (by connecting the nodes) and whether to have a flashing cursor. 
\item[Legend:] Allows you to specify what information will be displayed on the 
extensive form display next to each node and next to each branch.  
There are two positions where information can be displayed for each branch 
(above and below.)   In each of these positions, you can select to display 
information about the game tree (such as branch numbers or action names), 
or information about the currently selected solution (such as action  
probabilities or action values).  
Similarly, there are three positions at which information can be displayed 
for a node (above, below, and to the right).  In the positions above and below 
the node, you can select to display 
information about the game tree (such as infoset name or ID), 
or information about the currently selected solution (such as realization 
probabilities, node values, belief probabilities). The position to the right of the node 
is reserved for information relating to outcomes, and here you can select to display 
the outcome name or outcome vector.  

{\em Note that double clicking on any of the textual information selected 
above will call up a dialog to modify the particular entry}
\item[Colors:] You can change the default colors assigned to the players in this 
menu. 
\item[Accels:]  You can define and save accelerator keys for use in the 
extensive form. A more complete discussion of this topic is given in the section 
on \helpref{Accelerator Keys}{Accelerators}
\end{description}

\subsection{Command language logging}
GAMBIT can also be used without the GUI, by running the command line
version of the program.  However, it is often convenient to build an
extensive form tree in the GUI, and then operate on it in the command
language.  To facilitate this, the GAMBIT GUI can log all the commands
executed through it to a file in the command language format.  To start
logging, select the {\em File->Logging} menu.  All the commands executed
following this will be saved to a new file or appended to an existing one.
To stop logging, select the {\em File->Logging} menu again.  The resulting
file can now be loaded into the gcl (gambit command language) to produce
results identical to those achieved in the GUI.


\normalbox{For configuration and output features of this dialog see
the generic \helprefn{Table Window}{TableWindow} description.}

\subsection{Extensive Form Solutions}
A large number of different \helpref{algorithms}{SolutionAlgorithms} are 
available for solving extensive form
games.  In order to cater to experienced users familiar with the 
properties of individual algorithms, almost every relevant parameter can
be edited through a variety of solution related dialog boxes.  To allow 
simpler access to the basic features provided by the algorithms, a set of
commonly used {\em standard} solution types has been included with Gambit.
The user need only determine the general features of the desired equilibria
and Gambit will take care of all the details.

All of
the settings are saved to a defaults file every time they are changed. There
is no need to enter the custom parameters before each recalculation, even
through multiple sessions, unless a change is required.
After selecting the solution type/parameters, the algorithm must be started
by selecting the Solve->Solve menu or pressing the button on the toolbar.

\subsubsection{EFG Standard Solutions}\label{EFGStandardSolutions}
The standard solutions are selected by the Solve->Standard menu.  The
particular settings are then picked in the dialog box.
The following standard solutions are provided with Gambit.

\begin{tabular}{|l|l|l|}
\ruledrow{Type & Num & Internal Settings} \hline
\row{Nash & One &  LCP for 2-person, SimpDiv for n-person;Maximum of 1 solutions;
		Eliminate all weakly dominated strategies before solving;
		Mark subgames before solving.}
\row{Nash & Two & EnumMixed for 2-person, Liap for n-person;Maximum of 2 solutions;
		Eliminate all strongly dominated strategies before solving;
		Do not mark subgames before solving.}
\row{Nash & All & EnumMixed for 2-person, Liap for n-person;No maximum number of solutions;
		Eliminate all strongly dominated strategies before solving;
		Do not mark subgames before solving.}
\row{Subgame Perfect & One & LCP for 2-person, SimpDiv for n-person;Maximum of 1 solutions;
		Eliminate all weakly dominated strategies before solving;
		Mark subgames before solving.}
\row{Subgame Perfect & Two & EnumMixed for 2-person, Liap(NF) for n-person;Maximum of 2 solutions;
		Eliminate all strongly dominated strategies before solving;
		Mark subgames before solving.}
\row{Subgame Perfect & All & EnumMixed for 2-person, Liap(EF) for n-person;No maximum number of solutions;
		Eliminate all strongly dominated strategies before solving;
		Mark subgames before solving.}
\row{Sequential & One & Gobit; Maximum of 1 solution; Do not eliminate
		dominated strategies; Do not mark subgames before solving.}
\row{Sequential & Two & Liap; Maximum of 2 solutions {\em not guaranteed};Do not eliminate
		dominated strategies; Do not mark subgames before solving.}
\row{Sequential & All & Liap; Maximum of 2 solutions {\em not guaranteed};Do not eliminate
		dominated strategies; Do not mark subgames before solving.} \hline
\end{tabular}

\subsubsection{EFG Custom Solutions}\label{EFGCustomSolutions}
Only a user familiar with these algorithms should use the custom settings
since the program makes no attempt to check for the validity of some of
the parameters.  For a detailed discussion of the 
\helpref{algorithms}{SolutionAlgorithms} see the appropriate sections.

All of the custom features are selected in the Solve->Custom submenu.  

Each solution is controlled by three groups of settings

\begin{description}
\item[Algorithm] The actual algorithm used to calculate the equilibria.  Not all algorithms
are applicable to all games.  If an algorithm is not available for the
currently loaded normal form, it will be grayed out in the dialog.  After an
algorithm has been selected, the selection may be saved by pressing 
the {\tt Save} button, or an additional level or parameters can be accessed by
pressing the {\tt Params} button.

It is possible to use many of the normal form algorithms to solve
extensive form games by first converting them to normal form.  If one of
these algorithms is required, checking the {\em Use NF} checkbox will
enable the {\em Nfg Algorithms} choicebox.  A normal form representation
will be automatically created, the algorithm run and the solutions either
output to a file or converted back to behavioral strategy profiles
(depending on the algorithm type).
\item[Dominance] Dominance settings are used only if a NF base algorithm is
used.  In that case prior to running the algorithm, 
dominated strategies are eliminated from the temporary NF , thus
speeding up the consequent calculations.  The feature is toggled by 
the {\tt ElimDom before solve} checkbox.  Either a single step of elimination,
or a iterative elimination is selected by the {\tt Iterative Eliminate} 
checkbox.  The dominance type (strong/weak) and method (pure/mixed) are controlled
by the respective radioboxes.
\item[Subgames] \helpref{Subgames}{SubgamesMenu} may be marked prior to running
any algorithm.  In most cases marking the subgames will result in faster 
calculations.
\end{description}

Note that some algorithms require a starting point.  Depending on the
options selected in the custom parameters dialog for these algorithms,
a starting point dialog may pop up before the algorithm is run.  The
default (centroid) is usually the best place to start.  Double clicking on
the first profile in the window and pressing {\tt Ok} will select the
default and allow the algorithm to continue.  Advanced users may choose to
use values obtained by solving the game using a different algorithm as a
starting point.

\subsubsection{EFG Solutions and Subgames}\label{EFGSolnSubg}
If the extensive form contains more than one subgame, and if the subgames are
either marked explicitly or implicitly by selecting the {\tt Mark Subgames}
option, an additional feature becomes enabled.  Gambit will apply the 
selected algorithm to each subgame in reverse tree traversal order.  For each
solution obtained at a subgame node, a full solution will exist.  If there
are a lot of subgames, and each subgame produces multiple solutions, the 
total number of solutions grows exponentially.  Two parameters are available
to assure a reasonable number of total solutions.
\begin{description}
\item[Max Solns] The total maximum number of solutions to return.  Once this
number has been reached, no further solutions will be computed and the
algorithm will exit.
\item[Max Solns Per Subgame] No more than this number of solutions will be
computed for any subgame.  Note that this does not in general limit the
total number of solutions since the number of subgames can be large and the
number of solutions grows exponentially with the number of subgames.
\end{description}

To provide the user with finer control over which subgame solutions will be
used to computer the full solutions, the subgame solution picking dialog has
been created.  This option is enabled from the main Custom solutions dialog.
When enabled, for each subgame solved by Gambit, a solution display window
will be created and the user must select the subgame solutions to be used
for further calculations.  Solutions are selected/deselected by {\tt double-clicking} 
or {\tt Ctrl-clicking} on the corresponding soln number.  At least one solution
must be selected for each subgame.

\begin{figure}
$$\image{0cm;0cm}{efgpick}$$
\caption{Subgame solution picking dialog w/ all solns selected}\label{fig_efgpick}
\end{figure}

To speed up solution selection two shortcuts exist:

\begin{description}
\item[Select/unselect all of the solutions at once.]  The dialog starts up with
all of the solutions in the same state (selected or unselected) depending on
the current default (see next item).  If all of the solutions are selected,
a {\tt None} button will exist.  Pressing the button will unselect all of the
solutions.  The button will then change to {\tt All}.  Pressing the {\tt All} 
button will select all of the solutions.
\item[Default selection parameter.]  The startup state of the solution picking
dialog can be controlled through the {\tt Opt} button.  The {\tt Pick All Solutions} 
item is saved to a defaults file and is used for all future dialogs.
\end{description}

\subsubsection{Extensive Solutions Inspect}\label{ExtensiveSolutionInspect}
\begin{figure}
$$\image{3cm;3cm}{efgsoln}$$
\caption{Extensive Form Solution Inspect Window}\label{fig_efgsoln}
\end{figure}

The solution inspection window is a very powerful and complex part of the GUI.
It is accessed through the Inspect->Inspect menu.
The basic functionality consists of displaying the BehaviorProfiles that contain the
probability data for the solutions.  Each player's infoset is displayed on a separate
line.  The {\em Iset} column gives the infoset ID in the form of (player \#,iset \#).
The next cell contains a vector of probabilities in which each entry corresponds to
an action that can be taken at that infoset.  If a player has more than one infoset,
additional lines are used.  This format is repeated for each player in the game.

A variety of data dependent on the generating algorithm are available for most
 Behavior Profiles. Not every category is relevant for every profile.  If a
certain data member is either not relevant or not available, it will be 
displayed as {\tt ----}.  The display of each data member is toggled in the
Options dialog.  Note that the ID data member is always displayed (it is
an increasing (not necessarily consecutive) index used to identify each
solution)

Solutions may be removed by pressing the {\tt Remove} button.  A solution may
be edited, or a new one created by pressing the corresponding button.  Note 
that an edited or a newly created solution will have {\em User} as the
Creator and will have most data members undefined.  Double clicking on any
of the strategy values will automatically call up the solution edit dialog.
To facilitate working with a large number of solutions, 
the \helpref{Sort/Filter}{SolnSortFilt} functionality is available.

The number of the solution that is currently selected will be hilighted in the first
column.  To change the current solution (and thus change the display in the 
corresponding EF window), double click (control-click in UNIX) on the \# of the
desired solution.  To quickly browse the solutions, the {\em Update Solutions Dynamically}
option may be used.  With this option on, the solution will change automatically once
the cursor moves to a different solution \#.  To remove solution display from the NF window,
double click on the first row.

A useful feature when navigating large games is {\em Hilight Infosets}.  It is enabled
from the EF window {\em Solve->Features}.  When enabled, double clicking on an
infoset in the EF window will hilight the corresponding infoset in the Solution inspect
window.  Double clicking on an infoset (Iset column) in the Solution inspection window
will hilight the corresponding infoset in the EF window.

If the underlying EF was generated from an NF, and if the link between the two is still
valid, the {\em EF->NF} button will be enabled.  To project the solution to the NF and
immediately display it there, press this button.  NOT IMPLEMENTED.
 
An additional level of solution inspection is provided by the Node Inspect
window.  The window may be turned on/off through the Inspect->Info menu. When
enabled, a small window displays solution information pertaining to the cursor
node.  The window can be dismissed by turning the option off in the same menu.

\subsection{Extensive form supports}
The concept of an extensive form support is very similar to that of a normal form support.
Instead of considering the entire game tree, a subset of all the player actions is considered.
That is, at each decision node the player may have a reduced number of choices from that of
the original tree (full support).  However, at least one action must exist at each node.  Most 
of the extensive form solution algorithms support, or will soon support the supports.

\subsubsection{Creating EF Supports}
A support can be generated in one of two ways--either by dominated strategy elimination
[NOT IMPLEMENTED] or by explicitly deselecting particular actions.  To access the 
Supports dialog, select the Solve->Supports menu item, and press the {\em New} button.
The new dialog box contains a set of list boxes equal to the total number of players in the
game.  Each listbox lists all the infoset and strategy combination for a particular player.
The format is: (infoset #,action #).  Initially all of the actions are selected corresponding
to a full support.  Clicking on any (a,b) line will deselect that particular action in that
infoset.  Note that at least one action must remain selected for each infoset.  If this 
constraint is not satisfied, an error message will be issued and the invalid support discarded.
Once all the desired actions have been deselected, pressing the {\em Ok} button will return
to the previous dialog box.  For information on setting/using the newly created support, see
the \helpref{Examining EF Supports}{efsuptexam} section.

\subsubsection{Examining EF Supports}\label{efsuptexam}
The main supports dialog is accessed by selecting the Solve->Supports menu item.  This dialog
contains two choiceboxes with a list of all available supports (created manually or 
by some solution algorithm).
Two supports are defined for every EF window:
\begin{enumerate}
\item The displayed support determines the subset of the EF that is actually
displayed in the window.
\item The current support determines what support will be used for all the solution
computation and elimination of dominated strategies. 
\end{enumerate}
By default, these supports are identical.  However, an advanced user may wish to
display one support while working on another.  Solution display may become 
confusing in this case, as strategy probabilities will no longer add to one.  

The dimensionality of the supports is displayed in the text box above the selection
choicebox.  Since the dimensionality of an extensive form tree is a two dimensional structure,
the i'th element in the displayed vector corresponds to the {\em sum} of all the strategies
for the i'th player.  Thus, two different supports may have the same dimensionality vector.


\subsection{Default Accelerator Keys}\label{ExtFormDefAccl}
To speed up some of the more commonly used commands in the GUI, a set of
{\em accelerator keys} may be defined.  Once defined, a combination of keys
can be used to activate most of the menus and buttons.  Accelerator key
definitions are accessed through the Prefs->Accels menu.  Accelerator key
definitions are saved in the main Gambit defaults file (gambit.ini).  Gambit
is distributed with the following keys defined.
\begin{description}
\item[Insert:] Calls up the Build->Node->Add dialog
\item[ENTER:] Marks the node (Build->Node->Mark)
\item[SPACE:] Toggles the expand/collapse state of a subgame
\end{description}

\subsection{Output} 
Printed output of an extensive form can be obtained through the File->Output 
menu.  The procedures for obtaining output are common across many of the 
windows, and are discussed in more detail in the section on 
\helpref{Output}{outputsec}. 

\section{Table Window}\label{TableWindow}
The table window is used in many different places throughout GAMBIT.  It 
offers a large number of possible configuration options.  Gambit 
will attempt to choose the most suitable settings for each use, but
Since Gambit runs on many different platforms, the default settings 
may not be suitable for your platform.  You may modify these settings 
through the Display->Options menu or the {\em Config} button, whichever is
present.

\begin{description}
\item[Label Font] Each row and column can be labeled.  This option controls
the size and the appearance of the font used for these labels.
\item[Data Font] This is perhaps the most useful option that allows the 
appearance of the actual data displayed in the table to be changed.  By
reducing the size of the font, more information can be made visible at the
same time.  Note that cell dimensions will usually change with the font.
\item[Cell Width] The program will usually select the cell width to fit the
widest item in the table.  However, if an unusually long name is used,
this width may be insufficient.  The scrollbar controls the width of the
cell, measured either in character width's or pixels.  Character based
sizing is default and is recommended.  If for some reason more precise
dimensioning is desired, or if the cell width is not to change with the
font size, the {\em char} checkbox should be unchecked.  The width can be
changed for a single column by choosing the desired column \# in the {\em
Col} choicebox or for all the columns by choosing All in the choicebox
(default).
\item[Show Labels] It is possible to turn off the row and column labels by 
unchecking their corresponding checkboxes.  However, this is not
recommended since frequently important information is contained in those
labels.
\item[Color Text] Some instances of the table window employ colored text 
for clearer presentation of the data (i.e. the normal form, solution
inspection windows).  The only reason to turn this feature off is to speed
up display on very slow computer or over very slow networks.
\item[Output Precision] The number of digits displayed after the decimal point
in all floating point numbers is controlled by this slider.  Note that this
setting does not in any way effect the actual precision of the calculations.
A change in this setting will take effect the next time a table window is
created/refreshed (i.e. results are not immediate)
\end{description}

All of these features can be saved to a defaults file (gambit.ini).  They
will be taken into account the next time the table window is used, as long
as they are not overridden by the program.  Only the font information is
never overridden.


\section{Accelerator Keys}\label{Accelerators}

Many frequently executed commands are much more efficiently entered
through the keyboard than by using a mouse.  To speed up the use of the
GUI for an experienced user, most of the 'mouseable' commands can also be
done by entering a combination of keys on the keyboard.  Both the normal
and the extensive forms possess this functionality.  GAMBIT comes with a
pre-defined set of accelerator keys described in the corresponding 
\helpref{extensive}{ExtFormDefAccl} and 
\helpref{normal}{NormFormDefAccl} sections.  Any of these default command-key
associations can be edited or removed.  The accelerator key setting is
accessed through the {\em Display->Accel} menu.  In this dialog, the event
to be changed is first selected from the choicebox on the right.  If this
event is associated with a key combo already, the combo will be displayed
in the box on the left, otherwise, the entries in the box will be blank.
This association can be removed by pressing the {\em Delete} button.  A
new key combo can be assigned by selecting the combo in the box and
pressing {\em Set}.  If the desired key is not alphanumeric, its mnemonic
description can be selected from the choicebox, otherwise, it can just be
entered in the textbox.  A combination of Control \& Shift key modifiers
can be selected in the radioboxes below.  (Note that \verb+??+  means that either
state will be accepted).

\section{Output}\label{outputsec}
Many of the windows in Gambit have a menu item which can be 
selected to generate output to a printer or a file.  When you select this 
option, you will get a dialog box which will ask the medium to which the 
output is to be generated.  You can select one of the following 
choices:
\begin{description}
\item[Printer:] This will send the output directly to your printer. 
\item[Postscript:] Writes a postscript file, which can be viewed by a 
postscript viewer (such as ghostscript) sent to a printer 
supporting postscript.  
\item[Clipboard:] 
\item[MetaFile:]
\item[PrintPreview:]  This will generate display on your screen which will 
show the positioning of the output on the page.   
\end{description}

In some cases, there will be a toggle \verb+fit to page+.  If you select 
this option, then the entire output of the window 
will be displayed, even if it is not all 
visible on the your screen.  If you do not select this option, then only 
the portion that is visible on your screen is output.  

\section{Miscelaneous Notes}
\subsection{Mathematical Errors}
Under certain conditions some of the numerical algorithms may attempt to
execute an invalid mathematical operation.  Possible examples include
taking a logarithm of 0, attempting to exponentiate a very large number, and
other overflow or domain violation errors.  When this condition is detected,
the user is presented with an option to:
\begin{description}
\item[Continue] This is the default option.  The user acknowledges the error
and the algorithm continues.
\item[Ignore] The user acknowledges the error and all future errors are
ignored (no notification).
\item[Quit] The user instructs the program to terminate.
\end{description}

Note that if either of the first two options is selected, the results 
returned by the algorithm may be invalid.

\chapter{Solutions of Games}

\section{\index{Supports}{Supports}}\label{supportsec}

A support is the Cartesian product of strategy sets, one for each player, 
where  each component strategy set is a subset of the whole strategy 
set of the game for that player.  

All of the solution algorithms in Gambit take the current support as an 
argument.  When starting analysis of any normal form game,  the current 
support is the full support of the game:  the Cartesian product of the original 
strategy sets for each of the players.  You can change the support by 
selecting a new support in the \helpref{Solve->Support}{NewSupport} menu.  

\section{\index{domination}{Domination}}\label{domsec}

Gambit can eliminate weakly or strongly 
\popref{dominated strategies}{dominationgloss} in a 
normal form game, in the \helpref{Solve->ElimDom}{ElimDom} menu.  

To find undominated Nash equilibria, you can first eliminate dominated 
strategies, then select the support of the undominated strategies, and then 
solve for a Nash equilibria on that support.  

To find sophisticated equilibria, you can successively eliminate 
dominated strategies using ElimDom in the Solve menu.  This will 
create a nested set of supports.  
Select the smallest support found, and solve for Nash equilibria on that 
support.  

\section{Equilibria}

\subsection{\index{Nash equilibrium}{Nash Equilibrium}}\label{nashsec}
Every finite game has at least one \popref{Nash 
equilibrium}{nashequigloss} in either pure or mixed strategies (\cite{Nash:1950}).  

Gambit has a number of algorithms to find Nash equilibria.  
The appropriate algorithm to use 
depends on a number of factors, most importantly, the number of players in the 
game, and the number of equilibria you want to find. 

Before computing equilibria, you should 
\helpref{eliminate dominated strategies}{ElimDom} from 
the support.  The smaller the support, the faster any algorithm will run.  (Note that 
dominance elimination is currently only supported on the normal form.) 
\begin{itemize}
\item 
If you want to find more than one, or all Nash equilibria of a game, then you may first 
successively eliminate strongly dominated strategies.  Any equilibrium of the original 
game will also be an equilibrium of the reduced game. 
\item 
If you just want to find {\em one} Nash equilibrium, you can first successively eliminate 
{\em weakly} dominated strategies.  Elimination of weakly dominated strategies may 
eliminate some Nash equilibria of the original game (so it should not be used if you 
want to find multiple Nash equilibria,) but any Nash equilibrium to the reduced 
game will be an equilibrium to the original game, so it can be used if you only 
want to find one equilibrium .  
\end{itemize}

\subsubsection{Pure Equilibria}\label{purenashsec}
To find pure strategy Nash equilibria, for a normal form game, in the Solve 
menu select \helpref{Solve->EnumPure}{EnumPure}.  
To find pure Nash equilibria for an extensive 
form game,  in the Solve menu, select EnumPure, 
and check {\em Use NF}.  This will find 
all pure strategy Nash equilibria of the associated reduced normal form of 
the game, and convert them to behavior strategies.  

\subsubsection{Two Person Constant Sum Games}\label{csumsec}
For two person constant sum normal form 
games, the minimax theorem applies.  The set of 
Nash equilibria is a convex set, and the problem of finding Nash equilibria can be 
formulated as a linear program.  The \helpref{Lp algorithm}{Lp} will solve a 
constant sum game 
using this approach.  

\subsubsection{Two Person Games}\label{twopersec}
For two person nonzero sum 
games, the problem of finding Nash equilibria can be formulated as 
a linear complementarity problem, and exact solutions can be found as 
long as computations are done in rationals.  The 
\helpref{Lcp algorithm}{Lcp} solves a two person 
game using this approach.  Note that the Lcp algorithm can also be used 
directly on an extensive form game, where it implements the Koller, Megiddo, 
von Stengel Sequence form \cite{KolMegSte:94}.  To find {\em one} Nash equilibrium 
use Lcp with nEquilib set to 1.  

For a two person game the \helpref{EnumMixed algorithm}{EnumMixed} 
will enumerate all of the extreme points of the components of the set of 
Nash equilibria, and hence can be used to find {\em all} Nash equilibria.  Using 
EnumMixed with nEquilib set to 0 will find all Nash equilibria.  To find if 
there is more than one Nash equilibrium use EnumMixed with nEquilib set to 2.  

\subsubsection{N Person Games}\label{npernashsec}
For n-person normal form games, with n greater than two, Gambit does not 
currently have any algorithms 
to find all Nash equilibria.  Since Nash equilibria can be irrational, the algorithms to 
locate one equilibrium will only find approximations (to machine accuracy) of Nash 
equilibria. 

The \helpref{SimpDiv algorithm}{SimpDiv}
 is guaranteed to locate one (approximate) equilibrium 
for an n-person normal form game.  This algorithm can be 
very slow on some games.  Hence two other algorithms are also supported, 
\helpref{Gobit}{Gobit} and 
\helpref{Liap}{Liap}.  
These algorithms can also be used to search for multiple equilibria (with no 
guarantee that all have been found,) and to search for equilibria close to a given 
starting point. 

\subsection{\index{Sequential Equilibrium}{sequential equilibrium}}\label{seqnashsec}
Sequential equilibria are equilibria that prescribe optimal behavior at any 
information set of the extensive form, given a consistent assessment of 
beliefs.  See \cite{KrepsWilson:1982}.  To compute an approximation to a 
sequential equilibrium you can select GobitSolve in the Extensive form Solve 
menu. 

\section{Solution Algorithms}\label{SolutionAlgorithms}

\subsection{EnumMixed}\label{EnumMixed}
Finds all Nash equilibria for a two person game.  More
precisely, it finds the set of extreme points of the components
of the set of Nash equilibria.  The procedure is to enumerate the set
of complementary basic feasible solutions (see eg. /cite{Man:64}.)

{\em Limitations:}  Only works for two-person normal form games. 

The following parameters can be set:

\begin{description}
\item[nequilib:] Specifies the maximum number of equilibria to find.  The
default is zero, which means that all equilibria are found.  To check if
there is a unique Nash equilibrium, one could set this parameter to 2.
\item[time:] Returns the elapsed time for the operation.
\item[trace:] Sets the print level.  Higher values generate more output.
The default value is 0.
\end{description}

\subsection{EnumPure}\label{EnumPure}
Computation of pure strategy Nash equilibria is done by simple enumeration.
All pure strategies are checked to see if they are Nash equilibria.  

The following parameters can be set. 

\begin{description}
\item[num:] Allows the user to set the maximum number of Nash equilibria to
find.  Has a default value of 1. (not implemented yet) 
\end{description}


\subsection{Gobit}\label{Gobit}
Computes a branch of the logistic {\em quantal response equilibrium} 
correspondence for n-person normal form games (as described 
in \cite{McKPal:95a}  and n-person extensive form games (as described 
in \cite{McKPal:95b}.) The branch is computed 
for values of $\lambda$ between $minLam$ and $maxLam$.  The algorithm 
starts at $\lambda(0) = minLam$ if $delLam > 0$, or 
$\lambda(0) = maxLam$ if 
$delLam < 0$. It then increments according to the formula 
$$
\lambda(t+1) = \lambda(t) +delLam \lambda(t)^a.
$$
where $minLam$, $maxLam$, $delLam$, and $a$ are
parameters described below. In the computation for the first value of
$\lambda(0)$, the algorithm begins its search for a solution at the
starting point determined by the parameter \verb+start+.  At each
successive value of $\lambda(t)$, the algorithm begins it's search at
the point found in step \verb+t - 1+.  

This algorithm returns the last point computed.  
If the starting point is set to the centroid of the game (this is the
default), and $delLam > 0$, then this algorithm computes the {\em principal
branch} of the logistic Quantal response equilibrium.  In this case
taking the limit, as $\lambda$ goes to 0, the quantal response equilibrium
defines a unique selection from the set of Nash equilibrium for generic
normal form games.  Similarly, for extensive form games, it
defines a selection from the set of sequential equilibria.
Therefore, in extensive form games, this algorithm can be used to compute
approximations to a sequential equilibrium.

Limitations:  This algorithm currently can become numerically unstable 
on some problems for high values of $\lambda$, and may generate 
singularity errors.  Hence 
on some problems it may not be possible with the current
implementation to obtain a sufficiently close approximation to an
equilibrium. 

The following parameters can be set:

\begin{description}
\item[minLam:] Sets the minimum value of $\lambda.$ 
Default is $\lambda = 0.01$.
\item[maxLam:] Sets the maximum value of $\lambda.$  Default is 
$\lambda = 30.0.$
\item[delLam:]  The constant, $\delta,$ used in incrementing.   Default is 
$\delta = .01.$
\item[plotType:] Determines the exponent, $a,$ used in incrementing 
$\lambda.$  {\em Linear} corresponds to setting $a = 0,$ and {\em 
logarithmic} corresponds to setting $a = 1.$ Default is logarithmic.
\item[start:] Sets the starting point of the search for the initial value of 
$\lambda.$  Default is the centroid, where all strategies are chosen 
with equal probability (not implemented yet.)
\item[maxits 1D:] Sets the maximum number of iterations to the 
n-dimensional optimization routine.  Default is 20.
\item[maxits nD:] Sets the maximum number of iterations in the 
1-dimensional line search.  Default is 100.
\item[tol 1D:] Sets the tolerance for the n-dimensional optimization 
routine.  Default is 1.0e-10.
\item[tol nD:] Sets the tolerance for the 1-dimensional line search. 
Default is 2.0e-10.
\item[pxifile:] Can be used to generate an output file compatible for 
input to pxi, a program for graphical viewing and display of the output. 
\end{description}

\subsection{GobitAll}\label{GobitAll}
Performs a grid search to compute the complete logistic
quantal response correspondence (as described in \cite{McKPal:95a}
for a {\em small} two-person normal form game.

The algorithm computes approximate fixed points of the correspondence for 
the correspondence for values of $\lambda$ between $\verb+minLam+ 
and \verb+maxLam+.  Starting at $\lambda = minLam$, 
$\lambda$ is incremented according to the formula $$ 
\lambda(t+1) = \lambda(t) +delLam \lambda(t)^a.$$
where $minLam$, $maxLam$, $delLam$, 
and $a$ are parameters described
below. For each value of $\lambda$, a grid search is done over all values
of the probabilities for player $1$, evaluated on a grid of mesh 'del p.'
Points are evaluated in terms of the value of an objective function that
measures the distance between the original point, and the best response to
the best response (under the logistic best response function.)  Points
with a value of the objective function less than \verb+Tolerance+ are
approximate fixed points, and are kept, others are discarded.

Limitations:  This algorithm is only implemented for two-person normal
form games, and is very computationally intensive.

The following parameters can be set for GobitAll.

\begin{description}
\item[minLam:] Sets the minimum value of $\lambda.$
Default is $\lambda = .01.$
\item[maxLam:]  Sets the maximum value of $\lambda.$
Default is $\lambda = 3.$ 
\item[delLam:] Specifies the rate at which the value of Lambda changes.
Has a default value of .1.
\item[Plot Type:] Specifies whether to have geometric or linear incrementing.
Default is 0, resulting in $a = 1,$ or !geometric incrementing.
\item[del p:] Grid size for search over probability space.  
\item[Tolerance:] Maximum value of objective function for which to accept solution.  
\item[pxifile:] Can be used to generate an output file compatible for
input to pxi, a program for graphical viewing and display of the output.
\end{description}

\subsection{Lcp}\label{Lcp}
This algorithm formulates and solves the game as a linear complementarity problem.  
For a normal form game, this algorithm searches for equilibria of the
specified normal form game using the Lemke-Howson algorithm, as described
in \cite{LemHow:64}. Eaves \cite{Eav:71} lexicographic rule for
linear complementarity problems is used to avoid cycling.

In the Lemke Howson algorithm equilibria are found by following paths of
``almost'' equilibria, where one relaxes at most one constaint.
Equilibria are thus inter-connected by networks of paths that result when
different of the constraints are relaxed.  One can find the set of
``accessible''  equilibria in such methods by starting at the extraneous
solution and then tracing out this entire network.  See, e. g., Shapley,
\cite{Sha:74}.  However, the set of accessible equilibria is not necessarily all
Nash equilibria.

For extensive form games, this algorithm implements Lemke's algorithm
on the ``sequence form'' of the game, as defined by Koller, Megiddo 
and von Stengel, in \cite{KolMegSte:94}.

Limitations: This algorithm is fast, but only works for two person
games.  \cite{Wilson:1971} and \cite{Rosenmuller:1971} have suggested
ways in which the Lemke-Howson Algorithm can be extended to general
$n$-player games, but these extensions require methods of tracing the
solution to a set of non linear simultaneous equations, and have not
been implemented in GAMBIT.  Also, on some problems with data type of
Double, the current implementation can exhibit numerical instability
which in extreme cases can even lead to incorrect solutions.  Solving
the same problem with Rationals will resolve any such difficulties.
However, the algorithm is much slower when operating on Rationals than
on Doubles.

The following parameters can be specified:

\begin{description}
\item[n equi:] Specifies the number of equilibria to find.  If not specified,
the default value is zero, which means that all equilibria reachable by
the algorithm are to be found.
\item[max depth]
Specifies the maximum depth of search.
\end{description}

\subsection{Liap}\label{Liap}
Finds Nash equilibria via the Lyapunov function method
described in \cite{McK:91}.  Works on either the extensive or normal
form.  This algorithm casts the problem as a function minimization
problem by use of a Lyapunov function for Nash equilibria.  This is a
continuously differentiable non negative function whose zeros coincide
with the set of Nash equilibria of the game.  A standard descent
algorithm is used to find a constrained local minimum of the function
from any given starting location.  Since a local minimum need not be a
global minimum (with value 0,) the algorithm is not guaranteed to find
a Nash equilibrium from any fixed starting point.  The algorithm thus
incorporates the capability of restarting.  The algorithm starts from
the initial starting point determined by the parameter 'start'.  If a
Nash equilibrium is not found, it will keep searching from new
randomly chosen starting points until a Nash equilibrium has been
found or the maximum number of tries (parameter 'ntries') is exceeded,
whichever comes first.  For an extensive form game, if the algorithm
converges, it converges to a sequential equilibrium (Andrew Solnick,
personal communication).

Limitations: The disadvantages of this method are that it is generally
slower than any of the above methods, and also,  there can be local minima
to the Liapunov function which are not zeros of the function.  Thus the
algorithm can potentially converge to a non Nash point.  However,
inspection of the objective function can determine if this problem has
occurred.  If the objective function is zero, a Nash equilibrium has been
found. If it is greater than zero, the point is not Nash.  The algorithm
will automatically check this.  If the objective function is larger than the
tolerance, then the point is discarded.

The following parameters can be set;

\begin{description}
\item[start:] Sets the starting profile for the descent algorithm.  The
default is the centroid.
\item[trace:] Sets the print level.  Higher values generate more output.
The default value is 0.
\item[ntries:] Sets the maximum number of attempts at finding each
equilibrium. Default is 10
\item[nequilib:] Sets the number of equilibria to find.  Has a default
value of 1.  
\item[maxits nD:] Sets the maximum number of iterations to the
n-dimensional optimization routine.  Default is 200.
\item[maxits 1D:] Sets the maximum number of iterations in the
1-dimensional line search.  Default is 100.
\item[tol nD:] Sets the tolerance for the n-dimensional optimization
routine.  Default is 1.0e-10.
\item[tol 1D:] Sets the tolerance for the 1-dimensional line search.
Default is 2.0e-10.
\end{description}

\subsection{Lp}\label{Lp}
This algorithm formulates and solves the game as a linear program, and finds the 
minimax solution of the game.  This algorithm only works for two person, zero sum 
games.    

The following parameters can be specified:

\begin{description}
\item[n equi:] Specifies the number of equilibria to find.  If not specified,
the default value is zero, which means that all equilibria reachable by
the algorithm are to be found.
\end{description}

\subsection{SimpDiv}\label{SimpDiv}
Computes a Nash equilibrium to a normal form game based
on a simplicial subdivision algorithm.  The algorithm implemented is
that of \cite{VTH:1987}.  The algorithm is a simplicial subdivision
algorithm which can start at any point in the simplex.  The algorithm
starts with a given grid size, follows a path of almost completely labeled
subsimplexes, and converges to a completely labeled sub-simplex that
approximates the solution.  Additional accuracy is obtained by refining
the grid size and restarting from the previously found point.  The idea is
that by restarting at a close approximation to the solution, each
successive increase in accuracy will yield a short path, and hence be
quick.

In its pure form, the algorithm is guaranteed to find at least one
mixed strategy equilibrium to any n-person game.  Experience shows
that the algorithm works well, and is acceptably fast for many
moderate size problems.  But in some examples it can be quite slow.
The reason for this is that sometimes after restarting with a refined
grid size, even though the starting point is a good approximation to
the solution, the algorithm will go to the boundary of the simplex
before converging back to a point that is close to the original
starting point.  When this occurs, each halving of the grid size will
take twice as long to converge as the previous grid size.  If a high
degree of accuracy is required, or if the normal form is large, this
can result in the algorithm taking a long time to converge.

In order to combat the above difficulty, a parameter 'leash' has been
added to the algorithm which places a limit on the distance which the
algorithm can wander from the restart point. (Setting this parameter
to 0 results in no limit, and gives the pure form of the algorithm.)
With this parameter set to a non trivial value, the algorithm is no
longer guaranteed to converge, and setting small values of the
parameter will sometimes yield lack of convergence.  However,
experience shows that values of the parameter on the order of 10
generally do not destroy convergence, and yield much faster
convergence.

Parameters:

\begin{description}
\item[n equilib:] Maximum number of equilibria to find. Default is 1.  
\item[n Restarts:] Number of restarts.  At each restart the mesh of the
triangulation is halved.  So this parameter determines the final mesh
by the formula ${1/2}^{ndivs}$.
\item[Leash:] Sets the leashlength. Default is 0, which results in no
constraint, or no leash.  
\end{description}

\chapter{External Programs}
One particularly useful feature of gambit is the ability to generate complete 
logistic quantal response correspondence for normal form games.  These
algorithms generate a very large amount of data that is not easily visualized.
To display this data we created a specialized data plotting program {\em PXI}.


\begin{figure}
$$\image{3cm;3cm}{pxibig}$$
\caption{Sample PXI screenshot}\label{fig_pxibig}
\end{figure}

{\em PXI} was not funded from the same source as gambit itself, but was 
developed largely independently by Eugene Grayver.  PXI is not limited to
plotting data generated by gambit, but can be used to visualize a wide range
of econometric data.  It features four unique chart types, publication quality
graphics output, and an easy to use graphical interface.  PXI is available on
most platforms supported by gambit.  We strongly recommend using PXI if
either \helpref{Gobit}{Gobit} or \helpref{GobitAll}{GobitAll} algorithms are used.

\normalboxd{ PXI is being released as {\em shareware} and its source code is available
upon request.
For more information on PXI 
see \verb+http://www.cco.caltech.edu/~egrayver/pxi.html+}



\bibliography{gambit}
\addcontentsline{toc}{chapter}{Bibliography}
\setheader{{\it REFERENCES}}{}{}{}{}{{\it REFERENCES}}%
\setfooter{\thepage}{}{}{}{}{\thepage}%

\begin{helpglossary}
\setheader{{\it GLOSSARY}}{}{}{}{}{{\it GLOSSARY}}%
\setfooter{\thepage}{}{}{}{}{\thepage}%

\gloss{Action}\label{actiongloss}
An action is a collection of branches, one at each node of a player's information 
set, which are indistinguishable to the player making a decision at that 
information set. 

In Figure~\ref{fig_samp1}, each information set has two actions.  For the 
second player, at her information set, the two branches labeled MEET are one 
action, and the two branches labeled PASS are another action.
 
\gloss{Branch}\label{branchgloss}
A branch is the line connecting two nodes, one of which immediately 
follows the other.  Branches are numbered from 1 to k, where k is the 
number of branches at the node.  Branch 1 is the uppermost branch.  Branch 
2 is the second uppermost branch, etc. 

In Figure~\ref{fig_samp1}, each decision node has two branches, but in
other extensive forms, the number of branches can be different.

\gloss{Contingency}\label{continggloss}
A contingency for player i is a profile of strategies adopted by the other
 n - 1 players.

\gloss{Decision Node}\label{decnodegloss}
A decision node is a node which has at least one node that follows it.  
Decision nodes represent points at which either a player or chance must 
make a decision.  The different choices are represented by branches.

In Figure~\ref{fig_samp1}, there are a total of five decision nodes (including the chance node). 
 The number of the player who makes a choice at a node appears underneath the
node as the first number after the parenthesis.  On the graphics screen,
decision nodes are color coded, with each player represented by a
different color.

\gloss{Domination}\label{dominationgloss}
A strategy s for player i 
{\em weakly dominates} strategy t if, in every contingency, s 
is at least as good as t, and there is some contingency in which it is 
better.  The strategy s {\em strongly dominates} t if it is 
strictly better than t for player i in every contingency.  

In the poker example of \helpref{Figure~\ref{fig_samp1}}{normformsec}, 
for player 1, strategy \verb+21+ is weakly dominated by strategy \verb+12+, 
and strongly dominated by \verb+11+. 

\gloss{Gambit GUI}\label{guigloss}
This Program -- the Gambit Graphics User Interface.

\gloss{GCL}\label{gclgloss}
The Gambit Command Language -- a programming language for manipulation and
solution of extensive and normal form games.  This language is designed
for repetitive and computer intensive operations on games.  

\gloss{Indexed Traversal Order}\label{indextravgloss}
This is the ordering imposed on the nodes of a game tree by a lexicographic 
ordering of the nodes when each node is identified by the sequence of branch 
numbers necessary to reach it.  

\gloss{Information Set}\label{infosetgloss}
An information set is a collection of nodes which are all controlled by
the same player, but which are indistinguishable 
for the player at the point it is making a decision.  Since any two nodes 
in the same information set are indistinguishable, they must have exactly 
the same number of immediately following nodes. 

In Figure~\ref{fig_samp1}, player 1 has two information
sets, labeled (1, 1) and (1, 2), and player 2 has one, labeled (2, 1).
Player 2 has only one information set because Player 2 does not know whether
 Player 1 drew a red card or a black card from the deck.  So 2's decision 
cannot be contingent on that information.

\gloss{Nash Equilibrium}\label{nashequigloss}
A Nash equilibrium is a strategy profile having the property that no player 
can strictly benefit from unilaterally changing its strategy, while all 
other players stay fixed.  

\gloss{Node}\label{nodegloss}
A node is either decision node or a terminal node.  In the graphics 
display, a node is any location to which you can navigate the graphics 
cursor. In Figure~\ref{fig_samp1}, there are a total of eleven nodes -- namely 
five decision nodes (including the chance node), and six terminal nodes.  

\gloss{Outcome}\label{outcomegloss}
Outcomes are payoff vectors that associate with each player a payoff.  
Outcomes can be attached to any node, terminal or non terminal, in the 
extensive form of the game.  When play passes or terminates at a node 
with an outcome attached to it, each player accumulates the payoff which 
is associated to that player by the outcome at that node. 

In Figure~\ref{fig_samp1}, there is an outcome attached to each of the
terminal nodes.  They are indicated by the pair of numbers at each
terminal node.  The first number represents the payoff to player 1 and the
second the payoff to player 2.

\gloss{Poker Description}\label{pokergloss}
\begin{verbatim} 
Each player ante's $1.00 into the pot before starting
Player 1 (RED) draws a card from a deck. Player 1 observes the card, 2 (BLUE) does not. 
Player 1 then decides whether to FOLD or RAISE. 
  If player 1 chooses FOLD, player 2 wins the pot (a net loss of $1.00 to RED).
  If player 1 chooses RAISE, then player 1 throws a dollar in the pot, and player 2 has a move
    If player 2 chooses PASS, then player 1 wins the pot (a net gain of $2.00to RED)
    If player 2 chooses MEET, player 2 throws a dollar in the pot, and 1 must show the card: 
      If the card is RED Player 1 wins the pot (a net gain of $2.00 to RED).
      If the card is BLACK Player 2 wins the pot (a net loss of $2.00 to RED)
\end{verbatim}

\gloss{PureStrategies}\label{purestratgloss} 
A pure strategy for player i is a plan of action for that player for the 
entire game.  Thus, it is a specification of what branch to select at each 
of the player's information sets.  If player i's j th information set has 
k(j) branches.  Then the total number of pure strategies for player i is 
k(1) x k(2) x . . . x k(J).

\gloss{Realization Probability}\label{realprobgloss}
With each strategy profile, there is associated, to each node in the 
extensive form, a realization probability.  This is the probability of
 reaching that node under the given strategy profile.  This probability is 
computed by finding the path from the root node to the given node, and then 
computing the product of the probabilities of selecting each branch along 
the path.  

Note that all nodes, not just terminal nodes have realization probabilities
 attached to them.  The realization probability of the decision node at RED 
RAISE is 0.5.  The realization probability of the Root node is always 1.

\gloss{Reduced Normal Form}\label{rednormgloss}
The reduced normal form is the game that results when all strategies that 
are identical (result in  the same payoffs for all contingencies of the 
other players are represented by a single strategy.  

In many extensive form games, there are cases in which a choice adopted by 
a player early in the play precludes that player from ever reaching other 
of its information sets.  In this situation, the decisions that the player 
makes at the unreached information sets can not possibly affect the  payoff
 to that player(or to any other player).  Two strategies for a given player
 that differ only in what branch is selected at an unreachable'  information
 set will generate rows (or columns, as the case may be) in the normal form 
that are identical.  The reduced normal form eliminates such duplicated 
strategies, representing each duplicated strategy by its wildcard notation.  

\gloss{Root Node}\label{rootnodegloss}
The root node of a tree is the node with no parent. In the GAMBIT graphics
representation, the Root node is always the node that is furthest to the
left.  It can be reached by successively pressing {\bf $\leftarrow$}. 

In Figure~\ref{fig_samp1}, the Root node is the unlabeled node, furthest to 
the left.

\gloss{Strategy Profile}\label{stratprofgloss}
A strategy profile is an n-tuple of strategies, one for each player. The 
total number of strategy profiles is equal to the product of the number of 
strategies for each player. 
 
In the game of  Figure~\ref{fig_samp1}, the total number of strategy profiles
 is 8. 

\gloss{Subgame}\label{subgamegloss}
A subgame is a subtree with the property that for every information 
set containing members in the subtree, all members of the information 
set are elements of the subtree.  In other words, every information set 
is either contained in or has empty intersection with the subtree. 

In Figure~\ref{fig_samp1}, there are no proper subtrees of the extensive
form, because every subtree except that starting at the Root node breaks
up Player 2's information set.

\gloss{Subtree}\label{subtreegloss}
A subtree is a decision node together with the collection of nodes that
follow it in the tree.  Note that any subtree itself has a tree structure.

In Figure~\ref{fig_samp1}, each decision node defines a subtree consisting of itself
and its followers.   

\gloss{Terminal Node}\label{termnodegloss}
A terminal node is a node which has no other node following it.  Terminal 
nodes represent points at which the extensive form game ends, and typically 
have outcomes attached to them. 

In Figure~\ref{fig_samp1}, there are six terminal nodes, each followed by
a pair of numbers representing the payoffs to each player when that
terminal node is reached.

\gloss{Topological Tree}\label{toptreegloss}
A topological tree (also referred to as a Game tree) is simply a
collection of nodes which are connected together by branches in a way that
looks like a tree:  every node has at most one predecessor (parent), and exactly 
one node (the root node) has no predecessors.  
The Game tree represents a sequence of choices in the
chronological order that they occur.

\end{helpglossary} %

\end{document}

