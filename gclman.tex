%
% FILE: gclman.tex -- source for GCL manual
%
% $Id$
%
\documentstyle[widetext,chicagob]{article}
\renewcommand{\baselinestretch}{.9}
\newcommand{\bd}{\begin{description}}
\newcommand{\ed}{\end{description}}
\begin{document}

\title{GAMBIT COMMAND LANGUAGE\thanks{This is part of the Gambit
Project, which was funded in part by National Science Foundation
grants SBR-9308637 to the California Institute of Technology and
SBR-9308862 to the University of Minnesota.  We are grateful to Rob
Weber for help in compiling this manual and testing parts of the
language.}}

\author{Richard D. McKelvey\\California Institute of
Technology
\and
Andrew McLennan\\University of
Minnesota
\and 
Theodore Turocy\\California Institute of Technology
}

\date{\today\\ \jobname.tex}

\maketitle

\tableofcontents

\section{General Overview}

This document describes the GAMBIT Command Language (GCL).  GAMBIT is
a computer program that allows one to build, manipulate, and solve
finite extensive and normal form games. The GCL provides a method of
directing the operation of GAMBIT that is analagous to that of a high
level general purpose programming language.

The general purpose of the GCL is to provide a simple, but powerful
and flexible language by which one can perform complicated or
repetitive operations and procedures on games in extensive or normal
form.  The language has facilities for building and editing an
extensive or normal form game, converting back and forth between the
extensive and normal form representations, and then solving the
resulting game for various game theoretic equilibria of interest. Flow
control statements allow for repetitive operations (such as
investigating games as one changes various parameters) or conditional
operations.

\section{Installation}

\subsection{Unix}

\subsection{MS-DOS, Microsoft Windows}


\section{Getting Started}

There are small differences in the versions of the GCL between different
operating systems and platforms; however, the use and interface is
essentially the same in the main.  This manual is written for all platforms,
and any special notes pertaining to some particular platforms will be mentioned
at the appropriate time.

There are two ``modes'' in which you will use the GCL: interactive and batch.
First, we discuss batch mode.

Typing 'gcl [CR]' at your shell prompt (either in Unix or MS-DOS) will start
the GCL in batch mode.  After some preliminaries, you will get the GCL
prompt:
\begin{verbatim}
GCL1:
\end{verbatim}

which indicates that the program is ready to accept your input.  The prompt
number labels your input lines for later reference.

A GCL program consists of a series of statements, each ending in a
semi-colon or carriage return. A carriage return will not be
interpreted as the end of a statement, however, if there are unmatched
parentheses or brackets.  For example, the following two lines are
interpreted as one statement:

\begin{verbatim}
Plus[x->1, 
y->2]
\end{verbatim}

If both lines are followed by a carriage return, the statement will
generate the line of output:

\begin{verbatim}
3
\end{verbatim}

The function \verb+Plus+ is defined to return the sum of two integers,
so the command language evaluates the function and returns the value of
$1 + 2$, which is 3.  A more complete description of functions follows in
a later section.

[NOTE: There will be better batch mode support soon.]

The user has the option of creating a text file containing a series of
commands which can then be run together, as a program, by the
language.  This is referred to as ``batch mode''.
The output of such a program can be directed either to the
screen or to a file specified by the user.  A text file can be read by
the language as a program by placing an input file command {\tt "<"}
followed by the name of the file, following the command which enters
the command language.  By using an output file command {\tt ">"}
followed by the name of a file, the user can direct the output of the
command language to an external text file.  For example, at the Unix prompt:

\begin{verbatim}
> gcl < input.gcl > output.out
\end{verbatim}

\section{Language Description}

\subsection{Grammar and syntax}

The central paradigm of the GCL is that we can think of every legal
statement as a function call, which operates on some set of parameters and
returns a value.  Many commonly used functions have a more convenient
``short form'' format for additional terseness and readability.

Each function is identified by its \verb+FunctionName+, which is a
unique \verb+Name+.  A \verb+Name+ is any sequence of upper or lower
case letters or numbers beginning with a letter.  The set of all
built-in functions is listed in the Function Reference section of this
manual.

A statement in the GCL has the
following syntax
\begin{verbatim}
FunctionName[ArgumentList]
\end{verbatim}
where \verb+FunctionName+ is a unique name, and \verb+ArgumentList+ is
a sequence of arguments, separated by commas, each of which takes one
of the following two forms:
\begin{verbatim}
ArgumentName->argval
\end{verbatim}
or
\begin{verbatim}
ArgumentName<->argref
\end{verbatim}

Each function, when originally created, is given a list of formal
argument names, and each formal argument is assigned a data type,
which is one of the allowable data types listed in the section below
on data types.  For each of the built in functions, the formal
arguments and their data types are listed in the entry for that
function in the Function Reference section.  In the above syntax,
\verb+ArgumentName+, must be one of the list of possible formal
arguments for the given function, and \verb+argval+ or \verb+argref+
is any function that evaluates to correct data type for the
corresponding argument. 

The arguments to a function can be either optional or required.
Required arguments must be specified in \verb+ArgumentList+ when the
function is called.  Each optional argument has a default value, which
is also determined when the function is created.  Optional arguments
need only be specified when the default is to be changed.

The arguments to a function can be specified in any order, as long as
the above syntax for the \verb+ArgumentList+ is used.  However, the
argument list for a function can also be called without specifying the 
prefix \verb+ArgumentName->+ or \verb+ArgumentName<->+ as long as the 
arguments are specified in the correct order.  In fact a mixture is 
allowed in which the prefix is specified for some arguments and not
specified for others as long as all of the arguments which do not
include the prefix preceed those that do.  In this case, it is assumed
that those that do not include a prefix are in the correct order.  

Arguments can be passed in one of two ways, either by value or by
reference.  To pass an argument by value, the syntax is
\verb+ArgumentName->argval+.  If an argument is passed by value, then the
function whose name is specified in \verb+argval+ cannot be called
or changed by the function \verb+FunctionName+ unless it is in some
other way made visible to the function (specifically, if it is called
by reference in a different argument to the same function, or if it is a
funcion of global scope -- described below).  To pass by reference,
the syntax is \verb+ArgumentName<->argref+.  If an argument is passed
by reference, then the function whose name is specified in
\verb+argref+ can be called and changed by the function
\verb+FunctionName+.  

Finally, a function with no required arguments can be called by just
specifying the function name.  That function is then called with
default values for all optional arguments (if there are any.)

Summarizing, a rough description of the formal grammar of the GCL
follows:

\begin{verbatim}
Name: [A-Za-z]([A-Za-z0-9_])

Statement: FunctionName
        | FunctionName[ArgumentList]

FunctionName: Name

ArgumentList: Argument
        | Argument, Argumentlist

Argument: ArgumentName -> Statement
        | ArgumentName <-> Statement
        | Statement

ArgumentName: Name
\end{verbatim}

\subsection{Variables}

The simplest kind of function is a $variable$.  A variable is a
constant function with no arguments.  A variable stores one object of
the corresponding data type, and returns it when called.  To create a
user defined variable, or to give a new value to an existing variable,
one can use the built-in function,
\verb+Assign+, which has the syntax:
\begin{verbatim}
Assign[x<->T, y->T]
\end{verbatim}

If the \verb+x+ parameter is a previously undefined variable name, a new
variable of the type is created and the value of the expression passed
to \verb+y+ is ``assigned'' to \verb+x+.  If \verb+Assign+ is called with
an argument \verb+varname+ which is a currently visible function, it results
in an error if the data type which \verb+y+ evaluates to is different that the
current data type of \verb+x+.  On the other hand, if the
data types match, then \verb+varname+ is ``modified:'' \verb+varname+
is first deleted from the list of visible functions and then
\verb+Assign+ is executed.  

The Assign function is used so frequently, it has the following short form
representation:
\begin{verbatim}
name := expression
\end{verbatim}

\subsection{User defined functions}

As GCL programs become more and more complex, frequently there are complicated
operations which must be performed repeatedly.  The command language therefore
supports user-defined functions, which allow for defining sections of code
which may be called later.

A new function can be created using the function
\verb+NewFunction+.  For example, one might define a function to compute
the absolute value of an INTEGER as such:

\begin{verbatim}
NewFunction[Abs[n->INTEGER], INTEGER,
                 If[n > 0, n, If[n < 0, -n, 0]]]
\end{verbatim}

\noindent After defining the \verb+Abs+ function, it may be called in exactly
the same way a system-supplied predefined function may.

Parameter type matching rules apply to user defined functions in exactly the
same way as to predefined functions.  From the function's point of view,
the parameter list is a list of variables on which assignments are
automatically done at the beginning of the function execution.  So, taking
the \verb+Abs+ example above, in executing the call

\begin{verbatim}
Abs[42]
\end{verbatim}

\noindent an assignment \verb+n := 42+ is implicitly performed before the
body of the function is executed.

It is also possible to pass variables by ``reference'' to a user-defined
function in the same way as a predefined function.  In this case, the
function's ``local'' variable is stored in the same physical location in the
computer, and modifying the value locally also takes effect on the variable
passed to the function.  For example, it might
be useful instead to define \verb+Abs+ as:

\begin{verbatim}
NewFunction[Abs[n<->INTEGER], INTEGER,
		 If[n > 0, n, If[n < 0, n := -n, 0]]]
\end{verbatim}

\noindent in which case the function would still return the absolute value
of \verb+n+, but also modify the variable passed to \verb+n+ to be the
absolute value of the input \verb+n+.  So,

\begin{verbatim}
q := -37;
Abs[q]
\end{verbatim}

\noindent would result in the variable \verb+q+ containing the value 37
at the conclusion of execution.

Each function has its own ``scope'', or set of variables.  Within a function
body, the only variables which are visible are those which are declared
in the parameter list of the function (this is \verb+n+ in the \verb+Abs+
example above), and those which are created during the function's execution.
That is, no ``global'' or outside variables may be accessed directly by the
function.  For example, if the user typed in the following:

\begin{verbatim}
i := 4;
NewFunction[FooFunc[x->INTEGER], INTEGER, x * i]
\end{verbatim}

\noindent later execution of the \verb+FooFunc+ would yield an ``undefined
variable i'' error message, since \verb+i+ is never defined within the
function.  If instead \verb+FooFunc+ had been defined as follows:

\begin{verbatim}
NewFunction[FooFunc[x->INTEGER], INTEGER, i := 13, x * i]
\end{verbatim}

\noindent \verb+FooFunc+ would always return 13 times the value of the
parameter \verb+x+, since the value of \verb+i+ inside \verb+FooFunc+ is
always 13, regardless of the value of \verb+i+ outside of the function.


\subsection{Data types}

Like most high-level languages, GCL has a data typing system which
permits only one type of data to be returned by a given function.
However, GCL is a bit unusual in that it has implicit typing of
functions.  That is, a function's return type is determined by its
first use in the program.  For a user defined function, this is in the
NewFunction command that first creates the function.  For a user
defined variable, this will be in the \verb+Assign+ function that
first creates the variable.

The built-in data types for the GCL are the following:  

\medskip

\begin{tabular}{lp{4in}}
BOOL & Evaluates to either True or False. \\ 

INTEGER & Evaluates to an integer of arbitrary size \\ 

RATIONAL & Evaluates to a rational number of arbitrary precision\\ 

FLOAT 	& Evaluates to a floating point number \\

TEXT	& String of arbitrary length enclosed in quotation marks\\ 

INPUT & Input stream \\

OUTPUT & Output stream \\

EFG	& Game in extensive form\\

NFG	& Game in normal form\\

NODE	& A node in an extensive form\\

INFOSET & An information set in an extensive form \\

EFPLAYER & A player in an extensive form game \\

ACTION & An action in an extensive form game \\

OUTCOME & An outcome in an extensive form game \\

NFPLAYER & A player in a normal form game \\

STRATEGY & A strategy in a normal form game\\

MIXED	& Profile of mixed strategies\\

BEHAV	& Profile of behavioral strategies\\
\end{tabular}

\medskip

\noindent
Additionally, variables of any type can be contained in lists of arbitrary
length.
Variables of type LIST are defined similarly to the above variables,
except that the expression to the right of the {\tt ":="} symbol
contains a vector of variables of the same type enclosed in curly braces
and separated by commas.  For example:

\begin{verbatim}
LIST(BOOL) := { BOOL1, BOOL2, . . . . , BOOLn }
\end{verbatim}

The elements of a list are indexed sequentially, with the first element
defined to have index 1.  The predefined function \verb+Subscript+ is
provided to extract an element from a list:

\begin{verbatim}
Subscript[l->LIST(T), i->INTEGER] =: T
\end{verbatim}

\noindent which evaluates to a single variable of the same type as the list.
The shorthand for the \verb+Subscript+ function is \verb+l[[i]]+.

In the same manner, lists of any type may be created and manipulated.
Lists may be nested to any depth; however, the type of all the elements
of the list and any sublists must always be the same.  Hence

\begin{verbatim}
{ 3, { 4, 6 }, 23 }
\end{verbatim}

is a legal list, since all the elements are of type INTEGER, but

\begin{verbatim}
{ 3, { 4.2, 5.7 }, 23 }
\end{verbatim}

is not, since the second element is a LIST(FLOAT).

\subsection{Boolean and numeric data types}

\subsubsection{Boolean}

The BOOL data type is used for logical operations

\begin{itemize}
\item
Variables:

There are two built-in BOOL variables, \verb+True+  and \verb+False+, which
may not be modified: 

\bd
\item
\verb+True+ returns ``True.''

\item
\verb+False+ returns ``False.''
\ed

\item
Operators:

Let \verb+a+ and \verb+b+ be statements evaluating to BOOL. Then  

\bd
\item
\verb+And[a,b]+ is logical ``and.''  It returns \verb+True+ if
both \verb+a+ and \verb+b+ evaluate to \verb+True+, and otherwise
returns \verb+False+.  It has the short form \verb+a && b+, or
\verb+a AND b+. 

\item
\verb+Or[a,b]+ is logical ``or.''  It returns \verb+True+ if
at least one of \verb+a+ and \verb+b+ evaluate to \verb+True+, and otherwise
returns \verb+False+. It has the short form \verb+a || b+, 
or \verb+a OR b+

\item
\verb+Not[a]+ is logical negation.  It returns \verb+True+ if \verb+a+
evaluates to \verb+False+, and \verb+False+ if \verb+a+ is evaluates
to \verb+True+.  It has the short form \verb+!a+ or \verb+NOT a+.
\ed 
\end{itemize}

\subsubsection{Integers}
\begin{itemize}
\item
The INTEGER type may contain arbitrarily large (positive or negative)
values.  It is not limited by the default integer size on any particular
computer.

\item
Operators:
Let \verb+a+ and \verb+b+ be statements evaluating to INTEGER. Then

\bd
\item
\verb+Plus[a,b]+ returns the sum of the values returned by
\verb+a+ and \verb+b+.  It has the short form \verb&a + b&.  Also,
any integer $n>1$ can be viewed as short form for \verb+Plus[n-1,1]+  
 
\item
\verb+Negate[a]+ returns the integer which when added to \verb+a+ yields
0.  It has the short form \verb&-a&.  
 
\item
\verb+Minus[a,b]+ returns the sum of the values returned by
\verb+a+ and \verb+Negate[b]+.  It has the short form \verb&a - b&.  

\item
\verb+Times[a,b]+ returns the product of the values returned by
\verb+a+ and \verb+b+.  It has the short form \verb+a * b+.

\item
\verb+Divide[a,b]+ returns the largest integer which, when
multiplied by \verb+b+ is less than or equal to \verb+a+.  It has the
short form \verb+a DIV b+.

\item
\verb+Modulus[a,b]+ returns the remainder of \verb+a+ divided by
\verb+b+.  This is the integer which, when added to
\verb+Times[b,Divide[a,b]]+, yields \verb+a+.  It has the short form 
\verb+a MOD b+.
\item

\ed
\item
Relational operators take INTEGER arguments and
return a BOOL
\bd
\item
\verb+Greater[a,b]+ returns \verb+True+ if \verb+a+$>$\verb+b+, and
returns \verb+False+ otherwise.  It has the short form \verb+a > b+.

\item
\verb+Less[a,b]+ returns \verb+True+ if \verb+a+$<$\verb+b+, and
returns \verb+False+ otherwise.  It has the short form \verb+a < b+.

\item
\verb+GreaterEqual[a,b]+ returns \verb+True+ if \verb+a+$>=$\verb+b+, and
returns \verb+False+ otherwise.  It has the short form \verb+a >= b+.

\item
\verb+LessEqual[a,b]+ returns \verb+True+ if \verb+a+$<=$\verb+b+, and
returns \verb+False+ otherwise.  It has the short form \verb+a <= b+.

\item
\verb+Equal[a,b]+ returns \verb+True+ if \verb+a+$=$\verb+b+, and
returns \verb+False+ otherwise.  It has the short form \verb+a = b+.

\item
\verb+NotEqual[a,b]+ returns \verb+True+ if \verb+a+$!=$\verb+b+, and
returns \verb+False+ otherwise.  It has the short form \verb+a != b+.
\ed
\end{itemize}

\subsubsection{Rationals}

A RATIONAL returns a pair of integers, called the numerator and the
denominator, where the denominator is
non-zero.  A statement evaluating to rational is represented in the
form \verb+a/b+, where \verb+a+ is the numerator, and \verb+b+ is the
denominator.  

\begin{itemize}
\item
Operators for rationals can all be defined in terms of the operators
for integers.  Let \verb+a+ and \verb+b+ be statements evaluating to
RATIONAL. Say \verb+a+ evaluates to \verb+w/x+ and \verb+b+ evaluates
to \verb+y/z+

\bd
\item
\verb+Plus[a,b]+ returns \verb&w * z + y * x / x * z&. It has the
short form \verb&a + b&.  
 
\item
\verb+Negate[a]+ returns \verb+-w / x+.  It has the short form \verb&-a&.  
 
\item
\verb+Minus[a,b]+ returns \verb+w * z - y * x / x * z+.  
It has the short form \verb&a - b&.  

\item
\verb+Times[a,b]+ returns \verb+w * y / x * z+.  It has the short form 
\verb+a * b+.

\item
\verb+Divide[a,b]+ returns \verb+w * z / x * y+. It is only defined
when \verb+b+ does not evaluate to \verb+0+.  It has the
short form \verb+a / b+.
\item 
\verb+Power[a,b]+ is defined only when \verb+b+ is an INTEGER.  It 
returns \verb+w ^ b / x ^ b+.  It has the short form \verb+a ^ b+.

\ed
\item
Relational operators for RATIONAL can also be defined in terms of
relational operators for INTEGER.  
\bd
\item
\verb+Greater[a,b]+ returns \verb+w * z +$>$\verb+ x * y+.
It has the short form \verb+a > b+.

\item
\verb+Less[a,b]+ returns \verb+w * z +$<$\verb+ x * y+.
It has the short form \verb+a < b+.

\item
\verb+GreaterEqual[a,b]+ returns \verb+w * z +$>=$\verb+ x * y+.
It has the short form \verb+a >= b+.

\item
\verb+LessEqual[a,b]+ returns \verb+w * z +$<=$\verb+ x * y+.
It has the short form \verb+a <= b+.

\item
\verb+Equal[a,b]+ returns \verb+w * z +$=$\verb+ x * y+.
It has the short form \verb+a = b+.

\item
\verb+NotEqual[a,b]+ returns \verb+w * z +$!=$\verb+ x * y+.
It has the short form \verb+a = b+.
\ed
\end{itemize}

\subsubsection{Floats}

A FLOAT returns a pair of integers, in the form \verb+a E b+, where
\verb+a+ is called the mantissa, and \verb+b+ is called the exponent.
The mantissa and exponent have restricted range, as follows:  
\verb+| a | < FloatPrecis+, and \verb+| b - Mag[a] | < FloatMag+, where
\verb+Mag[a]+ is the magnitude of \verb+a+ -- the largest power of ten 
that is less than or equal to \verb+a+.  

It is perhaps more natural to think of a FLOAT as consisting of the
RATIONAL mantissa, \verb+a / 10^Mag[a]+, together with an integer
exponent, \verb+e = | b - Mag[a] |+.  This is the engineering format
in which floating point numbers are frequently expressed.  In this
case, the rational mantissa is written in decimal notation, with one
digit, followed by a decimal point, and followed by the other
significant digits.

From the point of view of speed, floating point numbers have nice
characteristics.  Floating point numbers are tailored to fit in a
fixed amount of storage space.  Hence floating point operations can be
encoded in hardware, and be done in a few clock cycles, in contrast to
the software implementation of arithmetic operations for arbitrary
precision integers or rationals.  

However, from the point of view of accuracy, the FLOAT data type has a
number of undesirable properties which mean that it should be avoided
whenever exact answers are required.  Unlike the INTEGER and RATIONAL
data types, the FLOAT data type does not satisfy the basic axioms of
an ordered field: The set of floating point numbers is not closed
under addition or multiplication.  Neither addition or multiplication
is associative (it is not always the case that
\verb&(a+b)+c = a+(b+c)& or 
\verb+(a*b)*c = a*(b*c)+).  Nor does multiplication
distribute over addition (it is not always True that 
\verb&a*(b+c) = a*b+a*c&.  Further, not all floating point numbers have
multiplicative inverses.  This means that we can have such
properties as \verb+b*(a/b) != a+ (if \verb+a = 1+ and
\verb+b = 3+, for example), or \verb&a+b = a& and yet not \verb+ b = 0+.  In
addition to these problems, the result of a floating poing operation
can be machine dependent, since \verb+FloatPrecis+ and 
\verb+FloatMag+ can differ from machine to machine.

\begin{itemize}
\item
Operators for floats can all be defined in terms of the operators for
integers.  Let \verb+a+ and \verb+b+ be statements
evaluating to FLOAT. Say \verb+a+ evaluates to \verb+wEx+ and
\verb+b+ evaluates to \verb+yEz+.  For any integer \verb+i+, let 
\verb+Insig[i]+ be the number of insignificant digits of \verb+x+.  In
other words, \verb+Insig[i]+ is the smallest integer such that 
\verb+| x DIV 10 ^ m | +$<=$\verb+ FloatPrecis+.  In all of the following
definitions, we write \verb+n = FloatMag+.

\bd
\item
\verb+Plus[a,b]+ returns \verb&(x DIV 10^m)E(m-n)&, where 
\verb&x = a*10^(b+n) + c*10^(d+n)&, 
and \verb+m = Insig[x]+.  
It has the short form \verb&a + b&.
 
\item
\verb+Negate[a]+ returns \verb+-w E x+.  It has the short form \verb&-a&.  
 
\item
\verb+Minus[a,b]+ returns \verb+Plus[a, -b]+
It has the short form \verb&a - b&.  

\item
\verb+Times[a,b]+ returns \verb&(x DIV 10^m)E(m-n)&, where 
\verb&x = a*10^(b+n) * c*10^(d+n)&, 
and \verb+m = Insig[x]+.  
It has the short form \verb&a * b&.
 
\item
\verb+Divide[a,b]+ returns \verb&(x DIV 10^m)E(m-2*n)&, where 
\verb&x = a*10^(b+2*n) DIV c*10^(d+n)&, 
and \verb+m = Insig[x]+.  
It has the short form \verb&a * b&.
 
\ed
\item
Relational operators for FLOAT can also be defined in terms of
relational operators for INTEGER.  
\bd
\item
\verb+Greater[a,b]+ returns \verb&w*10^(x+n) > y*10^(z+n)&.
It has the short form \verb+a > b+.

\item 
\verb+Less[a,b]+ returns \verb&w*10^(x+n) < y*10^(z+n)&.
It has the short form \verb+a < b+.

\item
\verb+GreaterEqual[a,b]+ returns \verb&w*10^(x+n) >= y*10^(z+n)&.
It has the short form \verb+a >= b+.

\item 
\verb+LessEqual[a,b]+ returns \verb&w*10^(x+n) <= y*10^(z+n)&.
It has the short form \verb+a <= b+.

\item 
\verb+Equal[a,b]+ returns \verb&w*10^(x+n) = y*10^(z+n)&.
It has the short form \verb+a = b+.

\item 
\verb+NotEqual[a,b]+ returns \verb&w*10^(x+n) != y*10^(z+n)&.
It has the short form \verb+a != b+.

\ed
\end{itemize}

\subsection{Extensive forms}

\subsubsection{Creating a new extensive form}

\subsubsection{Modifying an extensive form}

\subsubsection{Saving and loading extensive forms}

\subsection{Normal forms}

\subsubsection{Creating a new normal form}

Creating a game in normal form is accomplished by the predefined function
\verb+NewNfg+, which takes as its only required parameter a list of integers
specifying the number of strategies available to each player in the game.
Players are numbered starting at 1, and the $i$th element of the list
corresponds to the number of strategies for the $i$th player in the game.
Upon creation, all entries of the normal form are zero.

Additionally, normal forms may be created based upon an existing extensive
form game.  The predefined function \verb+EfgToNfg+ creates a normal form
which is the reduced normal form of an extensive form, and \verb+EfgToAfg+
produces the agent normal form for an extensive form.

\subsubsection{Modifying a normal form}

[Basically this stuff doesn't exist yet, but will soon. -- TED]

\subsubsection{Saving and loading normal forms}

The Gambit system reads and writes normal form games in a special ASCII
format.  By convention, files containing a normal form game have the extension
\verb+.nfg+.

To write a normal form game to a file, use the function \verb+WriteNfg+.
To read a normal form game from a file, use \verb+ReadNfg+.

\subsection{Order of precedence}

When functions are written in their canonical forms, no ambiguity
arises in the order of evaluation: In order to evaluate a function,
all arguments must be evaluated first.  Arguments are evaluated in the
natural order.  For built in functions, this is the default order
given in the Function Reference section.  For user defined functions,
it is the order specified when the function is created.  If another
function is encountered as an argument to a function, it is evaluated
at that time.  This leads to a recursive structure of evaluation,
which stops only when an argument being evaluated is a constant
function (i. e., a function with no arguments).

When short form forms are used, then ambiguity can arise in the
intended order of evaluation.  For example, the statement \verb&a + b * c&,
could be meant as \verb+Plus[a,Times[b,c]]+ or as
\verb+Times[Plus[a,b],c]+.  In order to resolve such ambiguities, all
functions that have a short form representation are given an order of
precedence.  When a statement is parsed by the GCL, it is first
scanned from left to right, replacing each short form expression at the
top level of precedence with its canonical form.  Then it is scanned
again replacing each short form expression at the second level of
precedence with its canonical form, and so on, until all short form
expressions have been eliminated.  

The order of precedence for built-in functions is as follows:
\bd
\item
\verb+()+
\item
\verb+:=+
\item
\verb+||+
\item
\verb+&&+
\item
\verb+NOT+
\item
\verb+= != < <= > >=+
\item
\verb&+ -&
\item
\verb+* / DIV MOD+
\item
\verb&(unary) + -&
\item
\verb+[[ ]]+
\ed

Thus, the statement \verb&a + b * c& would become 
\begin{verbatim}
Plus[a,b*c]
Plus[a,Times[b,c]]
\end{verbatim}

\noindent
On the other hand, the statement \verb&( a + b ) * c& would become 
\begin{verbatim}
Paren[a+b]*c
Paren[Plus[a,b]]*c
Times[Paren[Plus[a,b]],c]
\end{verbatim}

\noindent
which, since \verb+Paren+ is just the identity mapping, is equivalent to
\begin{verbatim}
Times[Plus[a,b],c]
\end{verbatim}

\subsection{Flow Control}

The GCL contains three functions which allow flow control within a
program.  Flow control functions differ from other functions in that
when they are evaluated, the list of visible functions does not
change.

\medskip

\begin{tabular}{lp{4in}}
If[BOOL, & \\
\{series of commands\}] & Evaluates the BOOL expression
once. If it evaluates to True, the series of commands are performed.
Each command must be followed by a semi-colon or carriage return.  If
the BOOL evaluates to FALSE, none of the commands are performed.\\
If[BOOL, & \\
\{series of commands\}, & \\
\{series of commands\}] & Evaluates the BOOL expression
once. If it evaluates to True, the first series of commands are
performed.  Each command must be followed by a semi- colon or carriage
return.  If the BOOL evaluates to FALSE, the second series of
commands are performed. \\
While[BOOL, &\\
\{series of commands\}] & Evaluates the BOOL
expression or variable.  As long as it evaluates to True, the series
of commands are performed.  Each command must be followed by a
semi-colon or carriage return.  If the BOOL evaluates to FALSE, the
loop ends without performing the commands.\\
NOTE: & The While command functions as a
loop.  That is, it continues to perform the series of commands as long
as the expression evaluates to True.  The If statement, however, only
performs the series of commands once if the expression evaluates to
True.\\
For[\{expressions\}, & \\
\{expression\}, & \\
\{expressions\}, & \\
\{statements\}] & Evaluates the first set of expressions as initialization;
afterwards, so long as the second parameter (which must evaluate to a 
BOOL value) remains True, executes the statement body followed by the
third parameter expressions in a fashion similar to the While command.
\end{tabular}

\section{Function Reference, by Category}

This section contains a list of functions, organized by
category.  For a description of each function, and the arguments,
refer to the following, alphabetically organized, section.

\noindent In this section, an asterisk (*) before the function name indicates
that the function has been implemented in the current binary, and a
hash mark (\#) indicates the function is implemented as a user-defined
function.

\protect \large \begin{verbatim}
*Assign[x<->T, y->T] =: T
\end{verbatim} \normalsize

\protect \large \begin{verbatim}
*Include[file->TEXT]
\end{verbatim}\normalsize

\protect \large \begin{verbatim}
*NewFunction[name[argument-list], body]
\end{verbatim}\normalsize

\protect \large \begin{verbatim}
*Quit
\end{verbatim}\normalsize

\protect \large \begin{verbatim}
*UnAssign[x<->T] =: T
\end{verbatim}\normalsize

\medskip
\subsection{Flow Control}

\protect \large \begin{verbatim}
*For[start, test, incr, body]
\end{verbatim}\normalsize


\protect \large \begin{verbatim}
*If[condition, t, f]
\end{verbatim} \normalsize


\protect \large \begin{verbatim}
*While[test, statements]
\end{verbatim} \normalsize


\medskip
\subsection{Lists}

\protect \large \begin{verbatim}
*Contains[list->LIST(T), x->T] =: BOOL
\end{verbatim}\normalsize

\protect \large \begin{verbatim}
*Length[list->LIST(T)] =: INTEGER
\end{verbatim}\normalsize

\protect \large \begin{verbatim}
*NthElement[list->LIST(T),n->INTEGER] =: T 
\end{verbatim}\normalsize

\protect \large \begin{verbatim}
*Plus[x->LIST(T), y->LIST(T)] =: LIST(T)
\end{verbatim}\normalsize

\protect \large \begin{verbatim}
*Remove[list->LIST(T), n->INTEGER] =: LIST(T)
\end{verbatim}\normalsize


\medskip
\subsection{String Functions}

\protect \large \begin{verbatim}
*Length[text->TEXT] =: INTEGER
\end{verbatim}\normalsize

\protect \large \begin{verbatim}
*NthChar[text->TEXT, n->INTEGER] =: TEXT
\end{verbatim}\normalsize

\protect \large \begin{verbatim}
*Plus[x->TEXT, y->TEXT] =: TEXT
\end{verbatim} \normalsize



\medskip
\subsection{Input and Output}


\protect \large \begin{verbatim}
*Input[file->TEXT] =: INPUT
\end{verbatim}\normalsize

\protect \large \begin{verbatim}
*LoadEfg[file->TEXT] =: EFG
\end{verbatim}\normalsize

\protect \large \begin{verbatim}
*LoadNfg[file->TEXT] =: NFG
\end{verbatim}\normalsize

\protect \large \begin{verbatim}
*Output[file->TEXT] =: OUTPUT
\end{verbatim}\normalsize

\protect \large \begin{verbatim}
Read[input->INPUT] =: [FLOAT,INTEGER,RATIONAL,TEXT,LIST]
\end{verbatim}\normalsize

\protect \large \begin{verbatim}
*ReadEfg[input->INPUT] =: EFG
\end{verbatim}\normalsize

\protect \large \begin{verbatim}
*ReadNfg[input->INPUT] =: NFG
\end{verbatim}\normalsize

\protect \large \begin{verbatim}
*SaveEfg[file->TEXT,overWrite->BOOL] =: BOOL
\end{verbatim}\normalsize

\protect \large \begin{verbatim}
*SaveNfg[file->TEXT,overWrite->BOOL] =: BOOL
\end{verbatim}\normalsize

\protect \large \begin{verbatim}
SetFormat[output->OUTPUT,width->INTEGER, precis->INTEGER] =: OUTPUT
\end{verbatim}\normalsize


\protect \large \begin{verbatim}
Write[output->OUTPUT, x->BOOL] =: OUTPUT 
Write[output->OUTPUT, x->INTEGER] =: OUTPUT
Write[output->OUTPUT, x->FLOAT] =: OUTPUT
Write[output->OUTPUT, x->RATIONAL] =: OUTPUT
Write[output->OUTPUT, x->TEXT, {quoted->BOOL}] =: OUTPUT
\end{verbatim}\normalsize


\protect \large \begin{verbatim}
*WriteEfg[{output->OUTPUT}, {efg->EFG}] =: OUTPUT
\end{verbatim}\normalsize


\protect \large \begin{verbatim}
*WriteNfg[{output->OUTPUT}, {nfg->NFG}, {sset->INTEGER}] =: OUTPUT
\end{verbatim}\normalsize



\medskip
\subsection{Logic}

\protect \large \begin{verbatim}
*And[x->BOOL, y->BOOL] =: BOOL
\end{verbatim} \normalsize


\protect \large \begin{verbatim}
*Equal[x->T, y->T] =: BOOL
\end{verbatim}\normalsize


\protect \large \begin{verbatim}
*Not[x->BOOL] =: BOOL
\end{verbatim}\normalsize


\protect \large \begin{verbatim}
*NotEqual[x->T, y->T] =: BOOL
\end{verbatim}\normalsize


\protect \large \begin{verbatim}
*Or[x->BOOL, y->BOOL] =: BOOL
\end{verbatim} \normalsize


\medskip
\subsection{Arithmetic}


\protect \large \begin{verbatim}
*Divide[x->T, y->T] =: T
\end{verbatim} \normalsize


\protect \large \begin{verbatim}
*Divide[x->INTEGER, y->INTEGER] =: INTEGER
\end{verbatim} \normalsize


\protect \large \begin{verbatim}
*Exp[x->FLOAT] =: FLOAT
\end{verbatim} \normalsize


\protect \large \begin{verbatim}
*Greater[x->T, y->T] =: BOOL
\end{verbatim}\normalsize


\protect \large \begin{verbatim}
*GreaterEqual[x->T, y->T] =: BOOL
\end{verbatim}\normalsize


\protect \large \begin{verbatim}
*Less[x->T, y->T] =: BOOL
\end{verbatim}\normalsize


\protect \large \begin{verbatim}
*LessEqual[x->T, y->T] =: BOOL
\end{verbatim}\normalsize


\protect \large \begin{verbatim}
*Log[x->FLOAT] =: FLOAT
\end{verbatim} \normalsize


\protect \large \begin{verbatim}
*Minus[x->T, y->T] =: T
\end{verbatim} \normalsize


\protect \large \begin{verbatim}
*Modulus[x->INTEGER, y->INTEGER] =: INTEGER
\end{verbatim}\normalsize

\protect \large \begin{verbatim}
*Paren[x->T] =: T
\end{verbatim}\normalsize

\protect \large \begin{verbatim}
*Plus[x->T, y->T] =: T
\end{verbatim} \normalsize

\protect \large \begin{verbatim}
*Times[x->T, y->T] =: T
\end{verbatim} \normalsize

\medskip
\subsection{Extensive Form Manipulation}

\protect \large \begin{verbatim}
*AppendAction[infoset->INFOSET] =: INFOSET
\end{verbatim}\normalsize

\protect \large \begin{verbatim} 
*AppendNode[node->NODE, infoset->INFOSET] =: NODE (first child)
\end{verbatim}\normalsize

\protect \large \begin{verbatim} 
*AttachOutcome[node->NODE, outcome->OUTCOME] =: OUTCOME
\end{verbatim}\normalsize

\protect \large \begin{verbatim}
*CopyTree[from->NODE, to->NODE] =: NODE (from)
\end{verbatim}\normalsize

\protect \large \begin{verbatim}
*DeleteAction[infoset->INFOSET, action->ACTION] =: INFOSET  (of action)
\end{verbatim}\normalsize

\protect \large \begin{verbatim}
*DeleteNode[node->NODE, keep->NODE] =: NODE (parent ?)  
\end{verbatim}\normalsize

\protect \large \begin{verbatim}
*DeleteOutcome[outc->OUTCOME] =: BOOL
\end{verbatim}\normalsize

\protect \large \begin{verbatim}
*DeleteTree[node->NODE] =: NODE
\end{verbatim}\normalsize

\protect \large \begin{verbatim}
*DetachOutcome[node->NODE] =: NODE
\end{verbatim}\normalsize

\protect \large \begin{verbatim}
*InsertAction[infoset->INFOSET, at->ACTION] =: INFOSET 
\end{verbatim}\normalsize

\protect \large \begin{verbatim}
*InsertNode[node->NODE, infoset->INFOSET] =: NODE
\end{verbatim}\normalsize

\protect \large \begin{verbatim}
JoinInfoset[node->NODE, infoset->INFOSET] =: NODE
\end{verbatim}\normalsize

\protect \large \begin{verbatim}
*LeaveInfoset[node->NODE] =: INFOSET
\end{verbatim}\normalsize


\protect \large \begin{verbatim}
*MergeInfosets[infoset1->INFOSET,
               infoset2->INFOSET] =: INFOSET (infoset1)
MergeInfosets[n->LIST(NODE)] =: INFOSET  (infoset of n[[1]])
\end{verbatim}\normalsize


\protect \large \begin{verbatim}
*MoveTree[from->NODE, to->NODE] =: NODE
\end{verbatim}\normalsize

\protect \large \begin{verbatim} 
*NewEfg[{rational->BOOL}, {players->LIST(TEXT)}] =: EFG
\end{verbatim}\normalsize


\protect \large \begin{verbatim} 
*NewInfoset[player->EFPLAYER, actions->LIST(TEXT),
            {name->TEXT}] =: INFOSET
*NewInfoset[player->EFPLAYER, actions->INTEGER,
            {name->TEXT}] =: INFOSET
\end{verbatim}\normalsize

\protect \large \begin{verbatim} 
*NewOutcome[{efg->EFG}, {name->TEXT}] =: OUTCOME
NewOutcome[payoffs->LIST(T), {efg->EFG}, {name->TEXT}] =: OUTCOME
\end{verbatim}\normalsize

\protect \large \begin{verbatim}
*NewPlayer[efg->EFG, {name->TEXT}] =: EFPLAYER 
\end{verbatim}\normalsize

\protect \large \begin{verbatim}
Reveal[infoset->INFOSET, who->LIST[EFPLAYER]],
       {what->LIST(ACTIONS)}] =: INFOSET
\end{verbatim}\normalsize

\protect \large \begin{verbatim} 
*SetChanceProbs[infoset->INFOSET, probs->LIST(T)] =: INFOSET
\end{verbatim}\normalsize

\protect \large \begin{verbatim}
*SetName[x<->ACTION, name->TEXT] =: ACTION
*SetName[x<->EFG, name->TEXT] =: EFG
*SetName[x<->INFOSET, name->TEXT] =: INFOSET
*SetName[x<->NFG, name->TEXT] =: NFG
*SetName[x<->NODE, name->TEXT] =: NODE
*SetName[x<->OUTCOME, name->TEXT] =: OUTCOME
*SetName[x<->PLAYER, name->TEXT] =: PLAYER
\end{verbatim}\normalsize

\protect \large \begin{verbatim}
SetPayoff[outcome->OUTCOME, player->PLAYER, payoff->T] =: OUTCOME
\end{verbatim}\normalsize

\protect \large \begin{verbatim}
*SetPayoff[outcome->OUTCOME, {payoff->LIST(T)}] =: OUTCOME
\end{verbatim}\normalsize

\medskip
\subsection{Normal Form Manipulation}

\protect \large \begin{verbatim}
AddStrategy[support->SUPPORT,list->LIST(STRATEGY)] =: SUPPORT
\end{verbatim}\normalsize

\noindent {\bf Note}: type param not implemented yet...
\protect \large \begin{verbatim}
*NewNfg[dim->LIST(INTEGER), {type->DATATYPE}] =: NFG
\end{verbatim}\normalsize

\protect \large \begin{verbatim}
NewSupport[nfg->NFG] =: SUPPORT
\end{verbatim}\normalsize

\protect \large \begin{verbatim}
CompressNfg[nfg->NFG,{support->SUPPORT}] =: NFG
\end{verbatim}\normalsize

\protect \large \begin{verbatim}
RandomNfg[nfg->NFG, {random->BOOL}, {seed->INTEGER}] =: NFG
\end{verbatim}\normalsize

\protect \large \begin{verbatim}
RandomEfg[efg->EFG, {random->BOOL}, {seed->INTEGER}] =: EFG
\end{verbatim}\normalsize

\protect \large \begin{verbatim}
RemoveStrategy[support->SUPPORT,list->LIST(STRATEGY)] =: SUPPORT
\end{verbatim}\normalsize

\protect \large \begin{verbatim}
SetPayoffNfg[list->LIST(STRATEGY), player->PLAYER, 
	payoff->T] =: LIST(STRATEGY)
\end{verbatim}\normalsize

\protect \large \begin{verbatim}
SetPayoffNf[list->LIST(INTEGER), player->PLAYER, 
	payoff->T] =: LIST(INTEGER)
\end{verbatim}\normalsize

\protect \large \begin{verbatim}
SetPayoffNfg[list->LIST(STRATEGY), payoffs->LIST(T)] =: LIST(STRATEGY)
\end{verbatim}\normalsize

\protect \large \begin{verbatim}
SetPayoffNfg[list->LIST(INTEGER), payoffs->LIST(T)] =: LIST(INTEGER)
\end{verbatim}\normalsize

\medskip
\subsection{Conversions}


\protect \large \begin{verbatim}
Afg[efg->EFG,{time<->FLOAT}] =: NFG
\end{verbatim}\normalsize


\protect \large \begin{verbatim}
Efg[nfg->NFG,{time<->FLOAT}] =: EFG
\end{verbatim}\normalsize


\protect \large \begin{verbatim}
*Float[x->INTEGER] =: FLOAT
*Float[x->FLOAT] =: FLOAT
*Float[x->RATIONAL] =: FLOAT
\end{verbatim} \normalsize

\protect \large \begin{verbatim}
*Nfg[{efg->EFG}, {time<->FLOAT}] =: NFG
\end{verbatim}\normalsize

\protect \large \begin{verbatim}
*Rational[x->INTEGER] =: RATIONAL
*Rational[x->FLOAT] =: RATIONAL
*Rational[x->RATIONAL] =: RATIONAL
\end{verbatim} \normalsize

\protect \large \begin{verbatim}
*Text[x->INTEGER] =: TEXT
*Text[x->FLOAT] =: TEXT
*Text[x->RATIONAL] =: TEXT
*Text[x->TEXT] =: TEXT
\end{verbatim} \normalsize



\medskip
\subsection{Solution Modules}

\protect \large \begin{verbatim}
*Behav[mixed->MIXED] =: BEHAV
\end{verbatim}\normalsize

\protect \large \begin{verbatim}
Behav[mixed->LIST(MIXED)] =: LIST(BEHAV)
\end{verbatim}\normalsize

\protect \large \begin{verbatim}
*ElimDom[nfg<->NFG, {start->SUPPORT}, {strong->BOOL}, 
	{mixed->BOOL}, {players->LIST(NFPLAYERS)}, 
	{time<->FLOAT}] =: SUPPORT
ElimAllDom[nfg<->NFG, {maxIter->INTEGER}, {strong->BOOL}, 
	{mixed->BOOL},	{players->LIST(NFPLAYERS)},
        {time<->FLOAT}] =: LIST(SUPPORT)
\end{verbatim} \normalsize

\protect \large \begin{verbatim}
*EnumMixedSolve[nfg->NFG, {stopAfter->INTEGER},
      {nPivots<->INTEGER}, {time<->FLOAT}] =: LIST(MIXED)
\end{verbatim}\normalsize

\protect \large \begin{verbatim}
*EnumPureSolve[nfg->NFG, {stopAfter->INTEGER}, 
          {time<->FLOAT}] =: LIST(MIXED) 
\end{verbatim}\normalsize

\protect \large \begin{verbatim}
*LCPSolve[efg<->EFG, {stopAfter->INTEGER},
          {nPivots<->INTEGER}, {time<->FLOAT}] =: LIST(BEHAV)
\end{verbatim}\normalsize

\protect \large \begin{verbatim}
*LCPSolve[nfg->NFG, {stopAfter->INTEGER},
          {nPivots<->INTEGER}, {time<->FLOAT}] =: LIST(MIXED)
\end{verbatim}\normalsize

\protect \large \begin{verbatim}
*LiapSolve[efg->EFG, {time<->FLOAT}, {nEvals<->INTEGER},
         {stopAfter->INTEGER}, {nTries->INTEGER}] =: LIST(BEHAV)
\end{verbatim}\normalsize

\protect \large \begin{verbatim}
*LiapSolve[nfg->NFG, {time<->FLOAT}, {nEvals<->INTEGER},
         {stopAfter->INTEGER}, {nTries->INTEGER}] =: LIST(MIXED)
\end{verbatim}\normalsize

\protect \large \begin{verbatim}
*LPSolve[nfg<->NFG, {nPivots<->INTEGER},
          {time<->FLOAT}] =: LIST(MIXED)
\end{verbatim}\normalsize
 
\protect \large \begin{verbatim}
LqreSolve[efg->EFG(FLOAT),
         {pxifile->OUTPUT}, {time<->FLOAT},
         {nEvals<->INTEGER}, {nIters<->INTEGER},
         {fullGraph->BOOL},
         {minLam->FLOAT}, {maxLam->FLOAT}, 
         {delLam->FLOAT}, {powLam->INTEGER}, 
         {start->BEHAV}] =: LIST(BEHAV)
\end{verbatim}\normalsize

\protect \large \begin{verbatim}
LqreSolve[nfg->NFG(FLOAT),
         {pxifile->OUTPUT}, {time<->FLOAT},
         {nEvals<->INTEGER}, {nIters<->INTEGER},
         {fullGraph->BOOL},
         {minLam->FLOAT}, {maxLam->FLOAT}, 
         {delLam->FLOAT}, {powLam->INTEGER}, 
         {start->MIXED}] =: LIST(MIXED)
\end{verbatim}\normalsize

\protect \large \begin{verbatim}
*LqreGridSolve[nfg->NFG, {pxifile->OUTPUT},
           {minLam->T}, {maxLam->T}, 
           {delLam->T}, {powLam->INTEGER}, 
           {delp->T}, {tol->T},
           {nEvals<->INTEGER}, {time<->FLOAT}] =: LIST(MIXED)
\end{verbatim}\normalsize

\protect \large \begin{verbatim}
SetOptions[alg->TEXT, param->TEXT, value->T] =: T
\end{verbatim}\normalsize

\protect \large \begin{verbatim}
*SimpDivSolve[nfg->NFG,{stopAfter->INTEGER}, 
         {nRestarts->INTEGER}, {leashLength->INTEGER},
         {nEvals<->INTEGER}, {time<->FLOAT}] =: LIST(MIXED)
\end{verbatim}\normalsize

\medskip
\subsection{Getting Information}

\protect \large \begin{verbatim} 
*Actions[infoset->INFOSET] =: LIST(ACTION)
\end{verbatim}\normalsize

\protect \large \begin{verbatim} 
*Centroid[efg->EFG] =: BEHAV
*Centroid[nfg->NFG,support->SUPPORT] =: MIXED
\end{verbatim}\normalsize

\protect \large \begin{verbatim}
*Chance[efg->EFG] =: EFPLAYER
\end{verbatim}\normalsize

\protect \large \begin{verbatim}
*ChanceProbs[infoset->INFOSET] =: LIST(T)
\end{verbatim}\normalsize

\protect \large \begin{verbatim}
*Infoset[node->NODE] =: INFOSET
\end{verbatim}\normalsize

\protect \large \begin{verbatim}
Infosets[player->EFPLAYER] =: LIST(INFOSET)
\end{verbatim}\normalsize

\protect \large \begin{verbatim}
*IsPredecessor[node->NODE, of->NODE] =: BOOL
\end{verbatim}\normalsize

\protect \large \begin{verbatim}
*IsRoot[node->NODE] =: BOOL
\end{verbatim}\normalsize

\protect \large \begin{verbatim}
*IsSuccessor[node->NODE, from->NODE] =: BOOL
\end{verbatim}\normalsize

\protect \large \begin{verbatim} 
LastAction[node->NODE] =: ACTION
\end{verbatim}\normalsize

\protect \large \begin{verbatim}
*Name[x->ACTION] =: TEXT
*Name[x->EFG] =: TEXT
*Name[x->INFOSET] =: TEXT
*Name[x->NFG] =: TEXT
*Name[x->NODE] =: TEXT
*Name[x->OUTCOME] =: TEXT
*Name[x->PLAYER] =: TEXT
Name[x->STRATEGY] =: TEXT
\end{verbatim}\normalsize

\protect \large \begin{verbatim}
*NextSibling[node->NODE] =: NODE
\end{verbatim}\normalsize

\protect \large \begin{verbatim} 
#Nodes[{efg->EFG}] =: LIST(NODE)
\end{verbatim}\normalsize

\protect \large \begin{verbatim} 
#NonterminalNodes[{efg->EFG}] =: LIST(NODE)
\end{verbatim}\normalsize

\protect \large \begin{verbatim} 
*NthChild[node->NODE, n->INTEGER] =: NODE
\end{verbatim}\normalsize

\protect \large \begin{verbatim}
*NumActions[infoset->INFOSET] =: INTEGER
\end{verbatim}\normalsize

\protect \large \begin{verbatim}
*NumChildren[node->NODE] := INTEGER
\end{verbatim}\normalsize

\protect \large \begin{verbatim}
*NumInfosets[player->EFPLAYER] =: INTEGER
\end{verbatim}\normalsize

\protect \large \begin{verbatim}
*NumMembers[infoset->INFOSET] =: INTEGER
\end{verbatim}\normalsize

\protect \large \begin{verbatim}
#NumNodes[{efg->EFG}] =: INTEGER
\end{verbatim}\normalsize

\protect \large \begin{verbatim}
*NumOutcomes[{efg->EFG}] =: INTEGER
\end{verbatim} \normalsize

\protect \large \begin{verbatim}
*NumPlayers[{efg->EFG}] =: INTEGER
\end{verbatim} \normalsize

\protect \large \begin{verbatim}
NumStrats[player->NFPLAYER,{support->SUPPORT}] =: INTEGER
\end{verbatim}\normalsize

\protect \large \begin{verbatim}
*Outcome[node->NODE] =: OUTCOME
\end{verbatim}\normalsize

\protect \large \begin{verbatim}
*Outcomes[{efg->EFG}] =: LIST(OUTCOME)
\end{verbatim}\normalsize

\protect \large \begin{verbatim}
*Parent[node->NODE] =: NODE
\end{verbatim}\normalsize

\protect \large \begin{verbatim}
*Payoff[outcome->OUTCOME] =: LIST(T)
\end{verbatim}\normalsize

\protect \large \begin{verbatim}
Payoff[strategy->BEHAV] =: LIST(T)
\end{verbatim}\normalsize

\protect \large \begin{verbatim}
Payoff[strategy->MIXED] =: LIST(T)
\end{verbatim}\normalsize

\protect \large \begin{verbatim}
*Player[infoset->INFOSET] =: EFPLAYER
*Player[node->NODE] =: EFPLAYER
\end{verbatim}\normalsize

\protect \large \begin{verbatim}
*Players[efg->EFG] =: LIST(EFPLAYER)
\end{verbatim} \normalsize

\protect \large \begin{verbatim}
*Players[nfg->NFG] =: LIST(NFPLAYER)
\end{verbatim} \normalsize

\protect \large \begin{verbatim}
*PriorSibling[node->NODE] =: NODE
\end{verbatim}\normalsize

\protect \large \begin{verbatim}
RealizProbs[strategy->BEHAV] =: LIST(T)
\end{verbatim}\normalsize

\protect \large \begin{verbatim}
*RootNode[efg->EFG] =: NODE
\end{verbatim}\normalsize

\protect \large \begin{verbatim} 
#TerminalNodes[efg->EFG] =: LIST(NODE)
\end{verbatim}\normalsize


\medskip
\subsection{Timing}


\protect \large \begin{verbatim}
*ElapsedTime[] =: FLOAT
\end{verbatim}\normalsize


\protect \large \begin{verbatim}
*IsWatchRunning[] =: BOOL
\end{verbatim}\normalsize


\protect \large \begin{verbatim}
*StartWatch[] =: FLOAT
\end{verbatim}\normalsize


\protect \large \begin{verbatim}
*StopWatch[] =: FLOAT
\end{verbatim}\normalsize


\section{Function Reference, Alphabetical}

The following is a list of procedures with the operations that they
perform in the Gambit Command Language:

Variable types in all capital letters indicate the type of the
required parameter or return value of the procedure.

The symbol {\tt "=:"} to the right of a procedure specifies the type
of the return value (if any exists) produced by the procedure.

Paremeters enclosed in brackets {\tt "\{ \}"} represent optional
parameters which are not necessary for the procedure to function.

\begin{itemize}

%--A--

\item
\protect \large \begin{verbatim} 
*Actions[infoset->INFOSET] =: LIST(ACTION)
\end{verbatim}\normalsize

\item
\protect \large \begin{verbatim}
AddStrategy[support->SUPPORT,list->LIST(STRATEGY)] =: SUPPORT
\end{verbatim}\normalsize

\item
\protect \large \begin{verbatim}
Afg[efg->EFG,{time<->FLOAT}] =: NFG
\end{verbatim}\normalsize

\bd
\item
[Description:] Converts a game in extensive form to the same
game in the corresponding agent normal form.
\item
[Return value:] The new agent normal form game.
\item 
[Required parameters:]\hfil\null

\bd
\item
[efg:] The game in extensive form to be converted to agent normal
form.
\ed

\item
[Optional parameters:]\hfil\null
	
\bd
\item
[time:] Elapsed time for the operation.
\ed
\ed

\item 
\protect \large \begin{verbatim}
*And[x->BOOL, y->BOOL] =: BOOL
\end{verbatim} \normalsize
\bd
\item
[Short forms:] \verb+x && y+, \verb+x AND y+.
\item
[Description:] Logical And.
\item
[Return value:] Returns the \verb+True+ if both \verb+x+ and \verb+y+ are
\verb+True+, else returns \verb+False+.  
\item
[Required parameters:]\hfil\null
\bd
\item
[x:] First argument.  
\item
[y:] Second argument
\ed
\item
[Optional parameters:] None.
\ed


\item
\protect \large \begin{verbatim}
*AppendAction[infoset->INFOSET] =: INFOSET
\end{verbatim}\normalsize

\bd
\item
[Description:] Adds an action into the information set, as the last
action. 
\item
[Return value:] Returns the information set. [Ted -- better to return action]
\item
[Required parameters:]\hfil\null

\bd
\item
[infoset:] The information set which is to have a new
action appended to it.
\ed

\item
[Optional parameters:] None.\hfill\null
\ed


\item
\protect \large \begin{verbatim} 
*AppendNode[node->NODE, infoset->INFOSET] =: NODE (first child)
\end{verbatim}\normalsize

\bd
\item
[Description:] Places \verb+node+ into the information set
\verb+infoset+.  The result is \verb+node+ becomes a decision node,
with number of branches equal to the number of actions in \verb+infoset+.

\item
[Return value:] Returns the first child of the source node.
\item
[Required parameters:]\hfil\null
\par
\bd
\item
[node:] The terminal node to which branches are to be added.
\item
[infoset:] The information set to which the node will now belong to.
\ed
\ed

\item 
\protect \large \begin{verbatim}
*Assign[x<->T, y->T] =: T
\end{verbatim} \normalsize
for all types {\tt T}.
\bd
\item
[Short form:] \verb+name := expression+
\item
[Description:] Creates a variable (a function with no arguments) with name
given by \verb+name+ and return data type the same as that of
\verb+expression+
\item
[Return value:] Returns the value assigned.
\item
[Required parameters:]\hfil\null
\bd
\item
[name:] The name of the variable to be created.  This must be a different
name from any previously existing function.  
\item
[expression:] Returns the value to be stored by the variable
\ed
\item
[Optional parameters:] None.
\ed

\item
\protect \large \begin{verbatim} 
*AttachOutcome[node->NODE, outcome->OUTCOME] =: OUTCOME
\end{verbatim}\normalsize

\bd
\item
[Description:] Attaches the indicated outcome to the specified node.
\item
[Return value:] The outcome that is attached to the node. 
\item
[Required parameters:]\hfil\null

\bd
\item
[node:] The node to which the outcome is to be attached.  Note that the
node need not be terminal in order to have an outcome attached.
\item
[outcome:] The outcome to be attached.  
\ed

\item
[Optional parameters:] None.
\ed

%--B--

\item
\protect \large \begin{verbatim}
*Behav[mixed->MIXED] =: BEHAV
\end{verbatim}\normalsize
\protect \large \begin{verbatim}
Behav[mixed->LIST(MIXED)] =: LIST(BEHAV)
\end{verbatim}\normalsize

\bd
\item
[Description:] Maps a specified mixed profile to a behavioral
profile on the associated extensive form game.
\item
[Return value:] Returns the behavioral profile.
\item
[Required parameters:]\hfil\null

\bd
\item
[mixed:] The mixed profile which is to be projected to a behavioral
profile.
\ed

\item
[Optional parameters:] None.\hfill\null
\ed


%--C--

\item
\protect \large \begin{verbatim} 
*Centroid[efg->EFG] =: BEHAV
*Centroid[nfg->NFG,support->SUPPORT] =: MIXED
\end{verbatim}\normalsize

\item
\protect \large \begin{verbatim}
*Chance[efg->EFG] =: EFPLAYER
\end{verbatim}\normalsize

\item
\protect \large \begin{verbatim}
*ChanceProbs[infoset->INFOSET] =: LIST(T)
\end{verbatim}\normalsize

\bd
\item
[Description:] Finds the vector of probabilities for the actions at
the specified information set.  This is only meaninful for 
information sets belonging to the player representing chance.
\item
[Return value:] Returns the vector of action probabilities for the
information set.
\item
[Required parameters:]\hfil\null
	
\bd
\item
[infoset:] The information set for which the action probabilities are to
be found.
\ed

\item
[Optional parameters:] None.
\ed

\item
\protect \large \begin{verbatim}
CompressNfg[nfg->NFG,{support->SUPPORT}] =: NFG
\end{verbatim}\normalsize

\bd
\item
[Description:] Creates a new normal form game with support a subset of
the support for the original game.  
\item
[Return value:] The new normal form game.
\item
[Required parameters:]\hfil\null
	
\bd
\item
[nfg:] The game in normal form from which the new normal form game
is to be created.
\ed

\item  
[Optional parameters:]\hfill\null
\bd
\item
[support:] The support which is to become the full profile of strategy
sets for the reduced game.   
\ed
\ed

\item
\protect \large \begin{verbatim}
*Contains[list->LIST(T), x->T] =: BOOL
\end{verbatim}\normalsize

For all types T.

\bd
\item[Description:] Finds whether the specified value of type T is
contained in the indicated list of type T.
\item[Return value:] True when the specified value is contained in the
list.

\item[Required parameters:]\hfil\null
	
\bd
\item	
[x:] The value of type T to be searched for.
\item
[list:] The list of type T which is to be searched in.
\ed

\item
[Optional parameters:] None.
\ed


\item
\protect \large \begin{verbatim}
*CopyTree[from->NODE, to->NODE] =: NODE
\end{verbatim}\normalsize

\bd
\item
[Description:] Copies the subtree rooted at the specified node to the
indicated node.
\item
[Return value:] Returns the source node $from$.
\item
[Required parameters:]\hfil\null

\bd
\item	
[from:] The node at which the subtree to be copied is rooted.
\item
[to:] The node to which the subtree is to be copied.
\ed

\item
[Optional parameters:] None.
\ed

%--D--

\item
\protect \large \begin{verbatim}
*DeleteAction[infoset->INFOSET, action->ACTION] =: INFOSET  (of action)
\end{verbatim}\normalsize

\bd
\item
[Description:] Deletes the indicated action from the
information set containing the specified node.
\item
[Return value:] The infoset of the action.  [Required parameters:]\hfil\null
	
\bd
\item
[infoset:] The information set from which the action is to be deleted.
\item [action:] The action to be deleted.
\ed

\item
[Optional parameters:] None.
\ed

\item
\protect \large \begin{verbatim}
*DeleteNode[node->NODE, keep->NODE] =: NODE (parent ?)  
\end{verbatim}\normalsize

\bd
\item   
[Description:] Deletes the indicated node from the tree.  The
specified child of the deleted node is kept and replaces the node in
its spot in the tree.  The subtrees associated with all other children
of the deleted node are deleted.
\item
[Return value:] The node which replaces the deleted node.
\item
[Required parameters:]\hfil\null
	
\bd
\item
[node:] The node to be deleted.
\item
[keep:] The child of the deleted node which is to be kept.
\ed

\item
[Optional parameters:] None.
\ed

\item
\protect \large \begin{verbatim}
*DeleteOutcome[outc->OUTCOME] =: BOOL
\end{verbatim}\normalsize

\bd
\item
[Description:] Removes the specified outcome from the given game in
extensive form.  All nodes with the specified outcome attached are
reset to having no outcome attached.  If the outcome does not exist in
the given game, no action is taken.  The outcomes remaining in the
game are re-numbered, so that the number of the outcome removed is
given to the outcome which previously had the next highest number, the
number previously corresponding to this outcome is assigned to the
outcome which previously had the next highest number, and so on.
\item  
[Return value:] Returns \verb+True+ on success, else \verb+False+.  
\item
[Required parameters:]\hfil\null
	
\bd
\item
[outc:] The outcome to be removed.  
\ed
\ed

\item
\protect \large \begin{verbatim}
*DeleteTree[node->NODE] =: NODE
\end{verbatim}\normalsize

\bd
\item
[Description:] Deletes the subtree rooted at the specified node.  This
node then becomes a terminal node.
\item
[Return value:] The indicated node which is now a terminal node.
\item
[Required parameters:]\hfil\null

\bd
\item  
[node:] The node where the subtree to be deleted is rooted.
\ed

\item
[Optional parameters:] None.
\ed

\item
\protect \large \begin{verbatim}
*DetachOutcome[node->NODE] =: NODE
\end{verbatim}\normalsize

\bd
\item
[Description:] Sets the outcome attached to the specified node to the
null outcome.
\item   
[Return value:] The indicated node.
\item
[Required parameters:]\hfil\null

\bd
\item	
[node:] The node whose outcome is to be detached.
\ed

\item
[Optional parameters:] None.
\ed

\item 
\protect \large \begin{verbatim}
*Divide[x->T, y->T] =: T
\end{verbatim} \normalsize

For T = FLOAT, RATIONAL  
\bd
\item
[Short form:] \verb%x / y%
\item
[Description:] Divides $y$ into $x$.
\item
[Return value:] Returns the value of $x / y$  
\item
[Required parameters:]\hfil\null
\bd
\item
[x:] First argument.  
\item
[y:] Second argument
\ed
\item
[Optional parameters:] None.
\ed

\item 
\protect \large \begin{verbatim}
*Divide[x->INTEGER, y->INTEGER] =: INTEGER
\end{verbatim} \normalsize

\bd
\item
[Short form:] \verb%x DIV y%
\item
[Description:] Divides $y$ into $x$, truncating the result.
\item
[Return value:] Returns the value of $x DIV y$  
\item
[Required parameters:]\hfil\null
\bd
\item
[x:] First argument.  
\item
[y:] Second argument
\ed
\item
[Optional parameters:] None.
\item
[See also:] \verb+Modulus+.
\ed

\bd
\item
[Description:] Determines the number of strategies which dominate the
specified strategy for the indicated player in the given game in
normal form.
\item
[Return value:] Returns the number of strategies which dominate the
specified strategy.  If the strategy is undominated, returns a value
of zero.
\item
[Required parameters:]\hfil\null

\bd
\item
[N:] The game in normal form in which the specified strategy of the
indicated player is to be tested for domination.
\item
[pl:] The number corresponding to the player for whom the specified
strategy is to be tested for domination.
\item
[st:] The number corresponding to the strategy belonging to the
indicated player which is to be tested for domination.
\ed

\item
[Optional parameters:] None.
\ed


%--E--

\item
\protect \large \begin{verbatim}
Efg[nfg->NFG,{time<->FLOAT}] =: EFG
\end{verbatim}\normalsize

\item

\protect \large \begin{verbatim}
*ElapsedTime[] =: FLOAT
\end{verbatim}\normalsize

\bd
\item
[Description:] Calculates the time transpired since the last time
\verb+StartWatch[]+ was used if the system stopwatch is currently running.
Otherwise, calculates the time elapsed between the last
\verb+StartWatch[]+ - \verb+StopWatch[]+ match.
\item
[Return value:] Returns the time value given by the system stopwatch
\item
[Required parameters:] None.
\item   
[Optional parameters:] None.
\ed

\item
\protect \large \begin{verbatim}
*ElimDom[nfg<->NFG, {start->SUPPORT}, {strong->BOOL}, 
	{mixed->BOOL}, {players->LIST(NFPLAYERS)}, 
	{time<->FLOAT}] =: SUPPORT
ElimAllDom[nfg<->NFG, {maxIter->INTEGER}, {strong->BOOL}, 
	{mixed->BOOL},	{players->LIST(NFPLAYERS)},
        {time<->FLOAT}] =: LIST(SUPPORT)
\end{verbatim} \normalsize

\bd
\item
[Description:] Computes all dominated strategies for the given list of
players in the normal form game with the given support.  
\item  
[Return value:] Returns a new support with the undominated strategies.

\item
[Required parameters:]\hfill\null 
\bd
\item
[nfg:] The game in normal form for which dominated strategies are to
be eliminated.
\ed	
\item
[Optional parameters:]\hfil\null

\bd
\item
[start:] The initial support.  If unspecified, it defaults to the full
strategy set for each player in the game.  
\item
[strong:] Determines the type of dominance used.  The default value is
False, which signifies that weak dominance is used.  If `strong` is
specified to be True, strong dominance is used.
\item
[mixed:] Determines if mixed strategies are used in determining if a
given pure strategy is dominated.  The default value is False, meaning
that only pure strategies are used. If \verb+strong+ is specified to
be True, mixed strategies are used.
\item
[players:] A list of players for whom dominant strategies are
computed.  Default is all players in the game.  
\item
[time:] Returns the elapsed time for the operation.
\item
[maxIter:] The number of iterations to be performed by the procedure.  If
not specified, it has a default value of zero which means that all
dominated strategies are iteratively deleted.
\ed
\ed

\item
\protect \large \begin{verbatim}
*EnumMixedSolve[nfg->NFG, {stopAfter->INTEGER},
      {nPivots<->INTEGER}, {time<->FLOAT}] =: LIST(MIXED)
\end{verbatim}\normalsize

\bd
\item
[Description:] Finds all Nash equilibria (pure and mixed) for a two
person game.  More precisely, it finds the set of extreme points of
the components of the set of Nash equilibria.  The procedure is to
enumerate the set of complementary basic feasible solutions. (See, eg
\cite[1964]{Man:64}.)
\item
[Return value:] The list of equilibria found.
\item
[Required parameters:]
\bd
\item
[nfg:] The game in normal form for which the solutions are to be found.  
\ed
\item
[Optional parameters:]\hfil\null

\bd
\item
[stopAfter:] Specifies the maximum number of equilibria to find.  If
not specified, the default value is zero, which means that all
equilibria are found.  To check if there is a unique Nash equilibrium,
one could set this parameter to 2.
\item
[nPivots:] Number of pivots. 
\item
[time:] Returns the elapsed time for the operation.
\item
\ed
\ed

\item
\protect \large \begin{verbatim}
*EnumPureSolve[nfg->NFG, {stopAfter->INTEGER}, 
          {time<->FLOAT}] =: LIST(MIXED) 
\end{verbatim}\normalsize

\bd
\item
[Description:] Finds the pure strategy Nash equilibria in the
specified normal form game.  
\item
[Return value:] The equilibria
found.  
\item
[Required parameters:]\hfil\null
	
\bd
\item
[nfg:] The game in normal form for which the weak pure strategy Nash
equilibria are to be found.
\ed

\item
[Optional parameters:]\hfil\null
	
\bd
\item
[stopAfter:] Allows the user to set the maximum number of Nash equilibria to
find.  Has a default value of 1.
\item
[time:] Returns the elapsed time for the operation.
\ed
\ed

\item
\protect \large \begin{verbatim}
*Equal[x->T, y->T] =: BOOL
\end{verbatim}\normalsize

For all types T.

\bd
\item
[Short form:] \verb+x = y+.
\item
[Description:] Equality check for two objects.
\item
[Return value:] Returns the value of $x = y$.
\item
[Required parameters:]\hfil\null
	
\bd
\item
[x:] First argument.
\item
[y:] Second argument.
\ed

\item
[Optional parameters:] None.

\ed

\item
\protect \large \begin{verbatim}
*Exp[x->FLOAT] =: FLOAT
\end{verbatim} \normalsize


%--F--

\item
\protect \large \begin{verbatim}
*Float[x->INTEGER] =: FLOAT
*Float[x->FLOAT] =: FLOAT
*Float[x->RATIONAL] =: FLOAT
\end{verbatim} \normalsize


\item
\protect \large \begin{verbatim}
For[start, test, incr, body]
\end{verbatim}\normalsize

\bd
\item
[Description:] Executes $start$, then repeatedly evaluates $body$ and $incr$
until $test$ fails to give $True$.
\item
[See also:] \verb+While+.
\ed

%--G--

\item
\protect \large \begin{verbatim}
GetOutcome[E->EFG, outc->INTEGER] =: LIST(RATIONAL)
\end{verbatim}\normalsize

\bd
\item
[Description:] Finds the vector of payoffs to the players for the
outcome with the indicated number in the specified game in
extensive form.
\item
[Return value:] The vector of payoffs corresponding to the specified
outcome.  If the outcome is not defined, the procedure returns the
empty list.
\item
[Required parameters:]\hfil\null

\bd
\item
[E:] The game in extensive form for which the outcome is to be
found.
\item
[outc:] The number assigned to the outcome for which the payoff vector
will be returned.
\ed

\item
[Optional paramteters:] None.
\ed

\item
\protect \large \begin{verbatim}
*GreaterEqual[x->T, y->T] =: BOOL
\end{verbatim}\normalsize

For T = INTEGER, FLOAT, RATIONAL, or TEXT.

\bd
\item
[Short form:] \verb+x >= y+
\item
[Description:] Yields $True$ if $x$ is greater than or equal to $y$
\item
[Return value:] Returns the value of $x \geq y$.
\item
[Required parameters:]\hfil\null

\bd
\item
[x:] First argument
\item
[y:] Second argument
\ed

\item
[Optional parameters:] None.
\ed

\item
\protect \large \begin{verbatim}
*Greater[x->T, y->T] =: BOOL
\end{verbatim}\normalsize

For T = INTEGER, FLOAT, RATIONAL, or TEXT.

\bd
\item
[Short form:] \verb+x > y+
\item
[Description:] Yields $True$ if $x$ is greater than $y$
\item
[Return value:] Returns the value of $x > y$.
\item
[Required parameters:]\hfil\null

\bd
\item
[x:] First argument
\item
[y:] Second argument
\ed
\item
[Optional parameters:] None.
\ed

%--H--

%--I--

\item 
\protect \large \begin{verbatim}
If[condition, t, f]
\end{verbatim} \normalsize
  
\bd
\item
[Description:] If antecedent is True, evaluates $t$.  If
antecedent is false, evaluates $f$. 
\item
[Return value:] If antecedent is True, returns value of $t$.  If
antecedent is false, returns value of $f$.  
\item
[Required parameters:]\hfil\null
\bd
\item
[condition:] The boolean guard of the expression.
\item
[t:] Statements to be evaluated if antecedent is True. 
\item
[f:] Statements to be evaluated if antecedent is false. 
\ed
\ed


\item
\protect \large \begin{verbatim}
*Include[file->TEXT]
\end{verbatim}\normalsize

\bd
\item
[Description:] Includes the commands in the specified file as commands
in the running GCL program.  The program treats the command as if it
were replaced by the contents of the file.
\item
[Return value:] None.
\item
[Required parameters:]\hfil\null
	
\bd
\item
[file:] Any expression which evaluates to a text string which is a
valid filename.
\ed

\item
[Optional parameters:] None.\hfil\null
\ed

\item
\protect \large \begin{verbatim}
*Infoset[node->NODE] =: INFOSET
\end{verbatim}\normalsize

\protect \large \begin{verbatim}
Infosets[player->EFPLAYER] =: LIST(INFOSET)
\end{verbatim}\normalsize

\item
\protect \large \begin{verbatim}
*Input[file->TEXT] =: INPUT
\end{verbatim}\normalsize

\item
\protect \large \begin{verbatim}
*InsertAction[infoset->INFOSET, at->ACTION] =: INFOSET 
\end{verbatim}\normalsize

\bd
\item
[Description:] Inserts a new action into the information set before [Ted
-- or after?] the indicated actions.  If no action is specified, it is
inserted as the last action.
\item
[Return value:] Returns the information set. [Ted -- better to return action]
\item
[Required parameters:]\hfil\null

\bd
\item
[infoset:] The information set which is to have a new
action inserted into it.
\ed

\item
[Optional parameters:]\hfill\null
\bd
\item
[at:] The number which will correspond to the first inserted action.
\ed
\ed

\item
\protect \large \begin{verbatim}
*InsertNode[node->NODE, infoset->INFOSET] =: NODE
\end{verbatim}\normalsize

\bd
\item
[Description:] Inserts a new node belonging to the indicated
information set at the location of the specified node and moves that
node to the end of the first branch of the new node.
\item
[Return value:] Returns the inserted node.
\item
[Required parameters:]\hfil\null

\bd
\item
[node:] The location where the new node is to be inserted.  The node
currently at this location is then moved to the end of the first
branch of the new node.
\item
[infoset:] The information set to which the new node is to be added.  
\ed

\item
[Optional parameters:] None.
\ed

\item
\protect \large \begin{verbatim}
*IsPredecessor[node->NODE, of->NODE] =: BOOL
\end{verbatim}\normalsize

\bd
\item
[Description:] Finds whether or not the first node indicated is a
predecessor of the second node given.
\item
[Return value:] Returns True only if the first node indicated is a
predecessor of the second node given.
\item
[Required parameters:]\hfil\null

\bd
\item
[node:] The node which, when it is the predecessor of the other node,
will result in the return value being True.
\item
[of:] The node which, when it is preceeded by the other node, will
result in the return value being True.
\ed

\item
[Optional parameters:] None.
\ed

\item
\protect \large \begin{verbatim}
*IsRoot[node->NODE] =: BOOL
\end{verbatim}\normalsize

\item
\protect \large \begin{verbatim}
*IsSuccessor[node->NODE, from->NODE] =: BOOL
\end{verbatim}\normalsize
\bd
\item
[Description:] Finds whether or not the first node indicated is a
successor of the second node given.
\item
[Return value:] Returns True only if the first node indicated is a
successor of the second node given.
\item
[Required parameters:]\hfil\null
	
\bd
\item
[node:] The node which, when it is a successor of the other node, will
result in the return value being True.
\item
[of:] The node which, when it is succeeded by the other node, will
result in the return value being True.
\ed

\item
[Optional parameters:] None.
\ed

\item
\protect \large \begin{verbatim}
*IsWatchRunning[] =: BOOL
\end{verbatim}\normalsize

\bd
\item
[Description:] Finds whether or not the system stopwatch is currently
running.
\item
[Return value:] True only if the system stopwatch is currently
running.  Returns False otherwise.
\item
[Required parameters:] None.
\item
[Optional parameters:] None.
\ed

%--J--

\item
\protect \large \begin{verbatim}
JoinInfoset[node->NODE, infoset->INFOSET] =: NODE
\end{verbatim}\normalsize

\bd
\item
[Description:] Removes the fspecified node from the information set to
which it currently belongs and places it in the specified information
set.
\item
[Return value:] Returns the node $n$.
\item
[Required parameters:]\hfil\null
	  
\bd
\item
[node:] The node which is to be removed from its current information set
and placed in the information set belonging to the second node given.
\item
[to:] The node belonging to the information set to which the first
node specified will be transfered.
\ed

\item
[Optional parameters:] None.
\ed

%--K--

%--L--

\item
\protect \large \begin{verbatim} 
LastAction[node->NODE] =: ACTION
\end{verbatim}\normalsize

\item
\protect \large \begin{verbatim}
*LCPSolve[efg<->EFG, {stopAfter->INTEGER},
          {nPivots<->INTEGER}, {time<->FLOAT}] =: LIST(BEHAV)
\end{verbatim}\normalsize

\protect \large \begin{verbatim}
*LCPSolve[nfg->NFG, {stopAfter->INTEGER},
          {nPivots<->INTEGER}, {time<->FLOAT}] =: LIST(MIXED)
\end{verbatim}\normalsize
\bd
\item
[Description:] For a normal form game, the game is set up as a linear
complementarity problem, and solved via the Lemke-Howson Algorithm.
(See, eg., Searches for equilibria of the specified normal form game
specified using the Lemke-Howson algorithm (see
\cite[1964]{LemHow:64}) using lexicographic rule for insuring
termination, as developed in \cite[1971]{Eav:71}.

For an extensive form game, this algorithm implements Koller, Megiddo
and von Stengel's {\em sequence form} (\cite[1994]{KolMegSte:94}. The
sequence form is a formulation of the set of Nash equilibria as the
solution to a non linear complementarity problem in variables that
correspond to ``sequences'' in the extensive form, which they show can
be solved by Lemke's algorithm for Linear complementarity problems.
The method has nice properties in terms of its computational
complexity, as it only grows linearly in the size of the extensive
form game.  

\item
[Return value:] The list of equilibria found.
\item
[Required parameters:]\hfil\null

\bd
\item
[nfg:] The game in normal form for which the solution is to be
searched.
\item
[efg:] The game in normal form for which the solution is to be
searched.
\ed

\item
[Optional parameters:]\hfil\null

\bd
\item
[stopAfter:] Specifies the maximum number of equilibria to find.  If
not specified, the default value is zero, which means that all
equilibria reachable by the algorithm are to be found.
\item
[nPivots:] The total number of pivots. 
\item
[time:] Returns the elapsed time for the operation.
\ed
\ed

\item
\protect \large \begin{verbatim}
*LeaveInfoset[node->NODE] =: INFOSET
\end{verbatim}\normalsize

\bd
\item
[Description:] Removes the specified node from the information set to
which it currently belongs and creates and places it in a new
singleton information set belonging to the same player.
\item
[Return value:] The new information set.
\item
[Required parameters:]\hfil\null
	  
\bd
\item
[node:] The node which is to be removed from the information which it
currently belongs to.
\ed

\item
[Optional parameters:] None.
\ed

\item
\protect \large \begin{verbatim}
*Length[text->TEXT] =: INTEGER
\end{verbatim}\normalsize

\item
\protect \large \begin{verbatim}
*Length[list->LIST(T)] =: INTEGER
\end{verbatim}\normalsize
For all types T.

\bd
\item
[Description:] Finds the number of elements in the specified list of
type T.
\item
[Return value:] The number of elements found in the list.
\item
[Required parameters:]\hfil\null

\bd
\item
[list:] The list of type T for which the number of elements is to be
found.
\ed

\item
[Optional parameters:] None.
\ed

\item
\protect \large \begin{verbatim}
*LessEqual[x->T, y->T] =: BOOL
\end{verbatim}\normalsize

For T = INTEGER, FLOAT, RATIONAL, or TEXT.

\bd
\item
[Short form:] \verb+x <= y+
\item
[Description:] Yields $True$ if $x$ is less than or equal to $y$
\item
[Return value:] Returns the value of $x \leq y$.
\item
[Required parameters:]\hfil\null

\bd
\item
[x:] First argument
\item
[y:] Second argument
\ed

\item
[Optional parameters:] None.
\ed

\item
\protect \large \begin{verbatim}
*Less[x->T, y->T] =: BOOL
\end{verbatim}\normalsize

For T = INTEGER, FLOAT, RATIONAL, or TEXT.

\bd
\item
[Short form:] \verb+x < y+
\item
[Description:] Yields $True$ if $x$ is less than $y$
\item
[Return value:] Returns the value of $x < y$.
\item
[Required parameters:]\hfil\null

\bd
\item
[x:] First argument
\item
[y:] Second argument
\ed
\item
[Optional parameters:] None.
\ed

\item
\protect \large \begin{verbatim}
*LiapSolve[efg->EFG, {time<->FLOAT}, {nEvals<->INTEGER},
         {stopAfter->INTEGER}, {nTries->INTEGER}, 
	{start->BEHAV}] =: LIST(BEHAV)
\end{verbatim}\normalsize

\protect \large \begin{verbatim}
*LiapSolve[nfg->NFG, {time<->FLOAT}, {nEvals<->INTEGER},
         {stopAfter->INTEGER}, {nTries->INTEGER},
	{start->MIXED}] =: LIST(MIXED)
\end{verbatim}\normalsize


\bd
\item
[Description:] Finds Nash equilibria via the Lyapunov function method
described in \cite[1991]{McK:91}.  Works on either the
extensive or normal form.  This algorithm casts the problem as a
function minimization problem by use of a Lyapunov function for Nash
equilibria.  This is a continuously differentiable non negative
function whose zeros coincide with the set of Nash equilibria of the
game.  A standard descent algorithm is used to find a constrained
local minimum of the function from any given starting location.  Since
a local minimum need not be a global minimum (with value 0,) the
algorithm is not guaranteed to find a Nash equilibrium from any fixed
starting point.  The algorithm thus incorporates the capability of
restarting.  The algorithm starts from the initial starting point
determined by the parameter 'start'.  If a Nash equilibrium is not
found, it will keep searching from new randomly chosen starting points
until a Nash equilibrium has been found or the maximum number of tries
(parameter 'ntries') is exceeded, whichever comes first.
\item
[Return value:] Returns the list of solutions found.
\item
[Required parameters:] Exactly one of the following \hfil\null

\bd
\item
[nfg:] The game in normal form for which the Liapunov solution is to
be found.
\item
[efg:] The game in extensive form for which the Liapunov solution is
to be found.
\ed

\item
[Optional parameters:]\hfil\null

\bd
\item
[start:] Sets the starting profile for the descent algorithm.  The
default is the centroid.
\item
[ntries:] Sets the maximum number of attempts at finding each
equilibrium. Default is 10
\item
[stopAfter:] Sets the number of equilibria to find.  Has a default
value of 1.  [maxitsOpt:] Sets the maximum number of iterations to the
n-dimensional optimization routine.  Default is 200.
\item
[maxits1D:] Sets the maximum number of iterations in the
1-dimensional line search.  Default is 100.
\item
[maxitsND:] Sets the maximum number of iterations in the
n-dimensional optimization.  
\item
[tolND:] Sets the tolerance for the n-dimensional optimization
routine.  Default is 1.0e-10.
\item
[tol1D:] Sets the tolerance for the 1-dimensional line search.
Default is 2.0e-10.
\item
[time:] Returns the elapsed time for the operation.
\ed
\ed

\item
\protect \large \begin{verbatim}
*LoadEfg[file->TEXT] =: EFG
\end{verbatim}\normalsize

\bd
\item
[Description:] Loads an extensive form game (in standard format) from
an external file.  
\item
[Return value:] The game in extensive form read from the file.
\item

[Required parameters:

\bd
\item
[file:] The full path name of the file from which the extensive form game
is to be read. \ed

\item
[Optional parameters:] None.\hfil\null
\ed

\item
\protect \large \begin{verbatim}
*LoadNfg[file->TEXT] =: NFG
\end{verbatim}\normalsize

\bd
\item
[Description:] Loads a normal form game (in standard format) from
an external file.  
\item
[Return value:] The normal form game read from the file.
\item

[Required parameters:

\bd
\item
[file:] The full path name of the file from which the normal form game
is to be read. \ed

\item
[Optional parameters:] None.\hfil\null
\ed

\item
\protect \large \begin{verbatim}
*Log[x->FLOAT] =: FLOAT
\end{verbatim} \normalsize

\item
\protect \large \begin{verbatim}
*LPSolve[nfg<->NFG, {nPivots<->INTEGER},
          {time<->FLOAT}] =: LIST(MIXED)
\end{verbatim}\normalsize

\bd
\item
[Description:] Finds the minmimax solution (a Nash equilibrium) for a
two person constant-sum game, by solving it as a Linear Program.
\item
[Return value:] The equilibrium found.  Returns an empty set if the
game is not two person, constant sum. 
\item
[Required parameters:]
\bd
\item
[nfg:] The game in normal form for which the solutions are to be found.  
\ed

\item
[Optional parameters:]\hfil\null

\bd
\item[nPivots:] Number of pivots. 
\item[time:] Elapsed time for the operation.
\ed
\ed

\item
\protect \large \begin{verbatim}
LqreSolve[efg->EFG(FLOAT),
         {pxifile->OUTPUT}, {time<->FLOAT},
         {nEvals<->INTEGER}, {nIters<->INTEGER},
         {fullGraph->BOOL},
         {minLam->FLOAT}, {maxLam->FLOAT}, 
         {delLam->FLOAT}, {powLam->INTEGER}, 
         {start->BEHAV}] =: LIST(BEHAV)
\end{verbatim}\normalsize

\protect \large \begin{verbatim}
LqreSolve[nfg->NFG(FLOAT),
         {pxifile->OUTPUT}, {time<->FLOAT},
         {nEvals<->INTEGER}, {nIters<->INTEGER},
         {fullGraph->BOOL},
         {minLam->FLOAT}, {maxLam->FLOAT}, 
         {delLam->FLOAT}, {powLam->INTEGER}, 
         {start->MIXED}] =: LIST(MIXED)
\end{verbatim}\normalsize

\bd
\item
[Description:] Computes a branch of the logistic quantal response
equilibrium correspondence (as described in \cite[1995]{McKPal:95a}
for normal form games, and in \cite[1995b]{McKPal:95b} for extensive
form games.  The branch is computed for values of $\lambda$ between
$\underline{\lambda}$ and $\bar{\lambda}.$ The algorithm starts at
$\lambda_0 = \underline{\lambda}$ if $\delta>0,$ or $\lambda_0 =
\bar{\lambda}$ if $\delta<0$. It then increments according to the
formula 
$$
\lambda_{t+1} = \lambda_t +\delta \lambda_t^a,
$$ where $\underline\lambda,$ $\bar\lambda,$ $\delta,$ and $a$ are
parameters described below. In the computation for the first value of
$\lambda_0$, the algorithm begins its search for a solution at the
starting point determined by the parameter ''start.''  At each
successive value of $\lambda_t,$ the algorithm begins it's search at
the point found in step $t - 1.$ 

The {\em principal branch} of the logistic quantal response
equilibrium mapping is the branch that is connected to the centroid of
the game at $\lambda = 0$.  This branch can be computed by setting
$\lambda_0$ sufficiently small.  As discussed in McKelvey and Palfrey,
for generic normal form games, the principal branch of the quantal
response equilibrium correspondence converges to a unique selection
from the set of Nash equilibria as $\lambda$ goes to infinity.
Similarly, for generic extensive form games, the principal branch of
the quantal response mapping converges to a unique selection from the
set of sequential equilibria as $\lambda$ goes to infinity.  Hence, in
extensive form games, this algorithm can be used to compute an
approximation to a sequential equilibrium.
\item
[Return value:] The List of Mixed (or Behavior) profiles found -- one
profile for each value of $\lambda_t$.  If \verb+fullGraph+ is
\verb+False+, then only the last value is returned.  
\item 
[Required parameters:]\hfil\null Exactly one of the following

\bd
\item
[nfg:] The game in normal form for which the quantal response
equilibrium solution is to be
found.
\item	
[efg:] The game in extensive form for which the quantal response
equilibrium solution is to be found.
\ed

\item
[Optional parameters:]\hfil\null

\bd
\item
[pxifile:] Can be used to generate an output file compatible for input
to pxi, a program for graphical viewing and display of the output.  
\item
[time:] Returns the elapsed time for the operation.
\item
[nEvals:] Number of function evaluations.
\item
[nIters:] Number of iterations of the the function minimization.
\item
[minLam:] Sets $\underline\lambda,$ the minimum value of $\lambda.$
Default is $\underline\lambda = 0.01$.
\item
[maxLam:] Sets $\bar\lambda,$ the maximum value of $\lambda.$  Default is
$\bar\lambda = 30.0.$
\item
[delLam:]  Constant, $\delta,$ used in incrementing.   Default is
$\delta = .01.$
\item
[powLam:] Exponent, $a,$ used in incrementing $\lambda.$  Setting $a = 0$
corresponds to linear incrementing, and $a = 1$ results in geometric
incrementing.  Default is $a = 1$.
\item
[start:] Sets the starting point of the search for the initial value of
$\lambda.$  Default is the centroid, where all strategies are chosen
with equal probability.  
\item
[maxitsN:] Sets the maximum number of iterations to the
n-dimensional optimization routine.  Default is 200.
\item
[maxits1:] Sets the maximum number of iterations in the
1-dimensional line search.  Default is 100.
\item
[tolN:] Sets the tolerance for the n-dimensional optimization
routine.  Default is 1.0e-10.
\item
[tol1:] Sets the tolerance for the 1-dimensional line search.
Default is 2.0e-10.
\item
\ed
\ed

\item
\protect \large \begin{verbatim}
*LqreGridSolve[nfg->NFG, {pxifile->OUTPUT},
           {minLam->T}, {maxLam->T}, 
           {delLam->T}, {powLam->INTEGER}, 
           {delp->T}, {tol->T},
           {nEvals<->INTEGER}, {time<->FLOAT}] =: LIST(MIXED)
\end{verbatim}\normalsize

\bd
\item
[Description:] Performs a grid search to compute the complete logistic
quantal response correspondence for a (small) two-person normal form
game.  The correspondence is computed for values of $\lambda$ between
$\underline{\lambda}$ and $\bar{\lambda}.$ Points are evaluated in
terms of the value of an objective function, that measures the
distance between the original point, and the best response to the best
response (under the logistic best response function.)  Points that are
close (within 'tol') to being fixed points are kept, others are
discarded.  Values of the probabilities are evaluated on a grid of
mesh 'delp.'

\item
[Return value:] The List of Mixed (or Behavior) profiles found -- one
profile for each value of $\lambda_t$. 

\item
[Required parameters:]\hfil\null
	
\bd
\item  
[nfg:] The two person game in normal form which is to be operated on.
\ed

\item
[Optional parameters:]  See LqreSolve[] for additional parameters.

\bd
\item
[delp:] Grid size for search over probability space.  Default is XXXX .
\item
[tol:] The tolerance on the objective function.  Values of $p$ for
which the objective function is less this value are kept.  
\ed
\ed

%--M--

\item
\protect \large \begin{verbatim}
*MergeInfosets[infoset1->INFOSET,
               infoset2->INFOSET] =: INFOSET (infoset1)
MergeInfosets[n->LIST(NODE)] =: INFOSET  (infoset of n[[1]])
\end{verbatim}\normalsize

\bd
\item
[Description:] Moves all the nodes from the source information set,
$infoset1,$ into the target information set, $infoset2,$ and deletes
the source information set. 
\item
[Return value:] Returns the remaining information set $infoset1$.
\item
[Required parameters:]\hfil\null

\bd
\item
[infoset1:] The target information set.  
\item
[infoset2:] The source information set. 
\ed

\item
[Optional parameters:] None.
\ed

\item 
\protect \large \begin{verbatim}
*Minus[x->T, y->T] =: T
\end{verbatim} \normalsize

For T = INTEGER, FLOAT, RATIONAL  
\bd
\item
[Short form:] \verb%x - y%
\item
[Description:] Subtracts $y$ from $x$.
\item
[Return value:] Returns the value of $x - y$  
\item
[Required parameters:]\hfil\null
\bd
\item
[x:] First argument.  
\item
[y:] Second argument
\ed
\item
[Optional parameters:] None.
\ed

\item
\protect \large \begin{verbatim}
*Modulus[x->INTEGER, y->INTEGER] =: INTEGER
\end{verbatim}\normalsize

\bd
\item
[Short form:] \verb+x MOD y+
\item
[Description:] Computes the remainder from dividing \verb+y+ into \verb+x+.
\item
[Return value:] Returns the remainder of $x / y$.
\item
[Required parameters:]\hfil\null

\bd
\item
[x:] The dividend
\item
[y:] The divisor
\ed

\item
[Optional parameters:] None.
\item
[See also:] \verb+Divide+.
\ed


\item
\protect \large \begin{verbatim}
*MoveTree[from->NODE, to->NODE] =: NODE
\end{verbatim}\normalsize

\bd
\item
[Description:] Moves the subtree rooted at the first node specified to
the second given node.
\item
[Return value:] Returns the destination node $to$.
\item
[Required parameters:]\hfil\null
	  
\bd
\item
[from:] The node at which the subtree to be moved is currently rooted.
\item
[to:] The node to which the subtree is to be moved.
\ed

\item
[Optional parameters:] None.
\ed

%--N--

\item
\protect \large \begin{verbatim}
*Name[x->ACTION] =: TEXT
*Name[x->EFG] =: TEXT
*Name[x->INFOSET] =: TEXT
*Name[x->NFG] =: TEXT
*Name[x->NODE] =: TEXT
*Name[x->OUTCOME] =: TEXT
*Name[x->PLAYER] =: TEXT
Name[x->STRATEGY] =: TEXT
\end{verbatim}\normalsize

\bd
\item
[Description:] Finds the name of the specified object.
\item
[Return value:] The name of the object.
\item
[Required parameters:]\hfil\null

\bd
\item
[x:] The object whose name is to be found.
\ed

\item
[Optional parameters:] None.
\ed


\item
\protect \large \begin{verbatim} 
*NewInfoset[player->EFPLAYER, actions->LIST(TEXT),
            {name->TEXT}] =: INFOSET
*NewInfoset[player->EFPLAYER, actions->INTEGER,
            {name->TEXT}] =: INFOSET
\end{verbatim}\normalsize


\item   
\protect \large \begin{verbatim}
*NewFunction[name[argument-list], body]
\end{verbatim}\normalsize

\bd
\item
[Description:] Creates a new function.  The return data type of the
function is the return type of \verb+body+ on first evaluation.  It is
a run time error to ever return a different data type.  A function is
visible only after it is created.  Functions created within the body
of a function are only visible to the body, and to functions created
within the body.
\item
[Return value:] 1 if success, 0 if failure.  
\item
[Required parameters:]
\bd
\item[name:] The name of the newly created name.  Must be different
than all function names in current scope.  
\ed
\item
[Optional parameters:]\hfil\null
\ed

\item
\protect \large \begin{verbatim} 
*NewEfg[{rational->BOOL}, {players->LIST(TEXT)}] =: EFG
\end{verbatim}\normalsize

\bd
\item
[Description:] Allocates and returns a new extensive form game.
\item
[Return value:] The newly allocated game in extensive form.
\item
[Required parameters:] None.\hfil\null
\item
[Optional parameters:]\hfil\null

\bd
\item
[players:] A list of players for the game.  If
unspecified, the new game has only the chance player.
\item
[rational:] Specifies the type (Rational $=$ True, or Float $=$ False)
of the new extensive form game.
\item
[type:] Specifies the type (Float or Rational) of the new extensive
form game.
\ed
\ed

\item

\noindent {\bf Note}: type param not implemented yet...
\protect \large \begin{verbatim}
*NewNfg[dim->LIST(INTEGER), {type->DATATYPE}] =: NFG
\end{verbatim}\normalsize

\bd
\item
[Description:] Constructs a new game in normal form with the
specified dimensionality.  The \verb+seed+ parameter allows the creation
of a random normal form game with entries between 0 and 1.

\item
[Return value:] Returns the new game in normal form.
\item
[Required parameters:]\hfil\null
	
\bd
\item
[dim:] The desired dimensionality of the new game.
\ed

\item
[Optional parameters:]\hfil\null
	
\bd
\item
[type:] Specifies the type (Float or Rational) of the new normal form
game.  This parameter defaults to \verb+Float+.
\ed

\ed

\item
\protect \large \begin{verbatim} 
*NewOutcome[{efg->EFG}, {name->TEXT}] =: OUTCOME
NewOutcome[payoffs->LIST(T), {efg->EFG}, {name->TEXT}] =: OUTCOME
\end{verbatim}\normalsize

\bd
\item
[Description:] Creates a new outcome within the specified game in
extensive form. 
\item
[Return value:] The newly created outcome. 
\item 
[Required parameters:]\hfil\null

\bd
\item
[efg:] The game in extensive form for which a new outcome is to be
created.
\ed

\item
[Optional parameters:]\hfil\null

\bd
\item
[name:] Allows the user to specify a name for the new outcome
created.  If not specified, the new outcome is assigned the lowest
positive number for which no otcome is defined.  If the number
specified already exists as an outcome, no action is taken and the
procedure returns a value of zero.  (Ted -- is there a default name?)
\ed
\ed

\item
\protect \large \begin{verbatim}
*NewPlayer[efg->EFG, {name->TEXT}] =: EFPLAYER 
\end{verbatim}\normalsize

\item
\protect \large \begin{verbatim}
*NextSibling[node->NODE] =: NODE
\end{verbatim}\normalsize

\bd
\item
[Description:] Finds the next sibling of the specified node.
\item
[Return value:] The node which is the next sibling of the specified
node.
\item
[Required parameters:]\hfil\null

\bd
\item
[n:] The node for which the next sibling is to be found.
\ed

\item
[Optional parameters:] None.
\ed


\item
\protect \large \begin{verbatim}
NewSupport[nfg->NFG] =: SUPPORT
\end{verbatim}\normalsize

\item
\protect \large \begin{verbatim}
*Nfg[{efg->EFG}, {time<->FLOAT}] =: NFG
\end{verbatim}\normalsize

\bd
\item
[Description:] Converts a game in extensive form to the same
game as a reduced normal form game.
\item
[Return value:] The new normal form game.
\item
[Required parameters:]\hfil\null
	
\bd
\item
[E:] The game in extensive form to be converted to normal form.
\ed

\item
[Optional parameters:]\hfil\null
	
\bd
\item
[time:] Returns the elapsed time for the operation.
\ed
\ed

\item
\protect \large \begin{verbatim} 
#Nodes[{efg->EFG}] =: LIST(NODE)
\end{verbatim}\normalsize

\item
\protect \large \begin{verbatim} 
#NonterminalNodes[{efg->EFG}] =: LIST(NODE)
\end{verbatim}\normalsize

\item
\protect \large \begin{verbatim}
*Not[x->BOOL] =: BOOL
\end{verbatim}\normalsize

\bd
\item
[Short form:] \verb+NOT x+
\item
[Description:] Logical negation.
\item
[Return value:] Returns the boolean opposite of $x$.  
\item
[Required parameters:]\hfil\null
\bd
\item
[x:] The value to be negated.
\ed
\item
[Optional parameters:] None.
\ed

\item
\protect \large \begin{verbatim}
*NotEqual[x->T, y->T] =: BOOL
\end{verbatim}\normalsize

For all types T.

\bd
\item
[Short form:] \verb+x != y+.
\item
[Description:] Inequality check for two objects.
\item
[Return value:] Returns the value of $x != y$.
\item
[Required parameters:]\hfil\null
	
\bd
\item
[x:] First argument.
\item
[y:] Second argument.
\ed

\item
[Optional parameters:] None.

\ed

\item
\protect \large \begin{verbatim}
*NthChar[text->TEXT, n->INTEGER] =: TEXT
\end{verbatim}\normalsize

\item
\protect \large \begin{verbatim} 
*NthChild[node->NODE, n->INTEGER] =: NODE
\end{verbatim}\normalsize

\bd
\item
[Description:] Finds the node which is the indicated child of the
specified node.  
\item
[Short form:] node\#n.
\item
[Return value:] Returns the node which is the indicated child of the
specified node.
\item
[Required parameters:]\hfil\null
	  
\bd
\item
[node:] The node for which the indicated child is to be found.
\item
[n:] The branch number corresponding to the child of the specified node
which is to be found.
\ed

\item 
[Optional parameters:] None.
\ed

\item
\protect \large \begin{verbatim}
*NthElement[list->LIST(T),n->INTEGER] =: T 
\end{verbatim}\normalsize

\item
\protect \large \begin{verbatim}
*NumActions[infoset->INFOSET] =: INTEGER
\end{verbatim}\normalsize

\item
\protect \large \begin{verbatim}
*NumChildren[node->NODE] := INTEGER
\end{verbatim}\normalsize

\bd
\item
[Description:] Finds the number of children of the specified node.
\item
[Return value:] Returns the number of children of the specified node.
\item
[Required parameters:]\hfil\null

\bd
\item
[node:] The node whose children are to be counted.
\ed

\item
[Optional parameters:] None.
\ed

\item
\protect \large \begin{verbatim}
*NumInfosets[player->EFPLAYER] =: INTEGER
\end{verbatim}\normalsize

\bd
\item
[Description:] Finds the number of information sets belonging to the
specified player in a given extensive form game.
\item
[Return value:] Returns the number of information sets belonging to
the specified player.  If the indicated player is not defined in the
game, the return value is equal to zero.
\item
[Required parameters:]\hfil\null

\bd
\item
[player:] The player for whom the information sets are to be counted.
\item
 [Optional parameters:] None.
\ed
\ed

\item
\protect \large \begin{verbatim}
*NumMembers[infoset->INFOSET] =: INTEGER
\end{verbatim}\normalsize

\item
\protect \large \begin{verbatim}
#NumNodes[efg->EFG] =: INTEGER
\end{verbatim}\normalsize

\bd
\item
[Description:] Finds the number of nodes (including terminal nodes) in
the specified game in extensive form.
\item
[Return value:] Returns the number of nodes found.
\item
[Required parameters:]\hfil\null

\bd
\item
[efg:] The game in extensive form in which the number of nodes are to
be counted.
\ed

\item
[Optional parameters:] None.
\ed

\item
\protect \large \begin{verbatim}
*NumOutcomes[efg->EFG] =: INTEGER
\end{verbatim} \normalsize

\bd
\item
[Description:] Finds the number of outcomes in the specified game
in extensive form.
\item
[Return value:] Returns the number of outcomes found.
\item
[Required parameters:] \hfil\null

\bd
\item
[efg:] The game in extensive form in which the number of outcomes is
to be found.
\ed

\item
[Optional parameters:] None.
\ed

\item
\protect \large \begin{verbatim}
*NumPlayers[efg->EFG] =: INTEGER
\end{verbatim} \normalsize

\bd
\item
[Description:] Finds the number of players in the specified game in
extensive form.
\item
[Return value:] Returns the number of players found.
\item
[Required parameters:]\hfil\null

\bd
\item
[efg:] The game in extensive form in which the number of players is
to be found.
\ed

\item   
[Optional parameters:] None.
\ed

\item
\protect \large \begin{verbatim}
NumStrats[player->NFPLAYER,{support->SUPPORT}] =: INTEGER
\end{verbatim}\normalsize

\bd

\item
[Description:] Finds the number of strategies for the specified player
in the normal form game with given supports.
\item
[Return value:] Returns the number of strategies found.
\item
[Required parameters:]\hfil\null

\bd
\item
[player:] The number corresponding to the player in the given game in
normal form for whom the number of strategies is to be found.
\ed

\item
[Optional parameters:]\hfil\null
	
\bd
\item  
[support:] The support for the normal form game.  Default is all
strategies for all players.  
\ed
\ed

%--O--

\item 
\protect \large \begin{verbatim}
*Or[x->BOOL, y->BOOL] =: BOOL
\end{verbatim} \normalsize
  
\bd
\item
[Short form:] \verb+x || y+, \verb+x OR y+.
\item
[Description:] Logical Or.
\item
[Return value:] Returns the maximum of $x$ and $y$.  
\item
[Required parameters:]\hfil\null
\bd
\item
[x:] First argument.  
\item
[y:] Second argument
\ed
\item
[Optional parameters:] None.
\ed

\item
\protect \large \begin{verbatim}
*Outcome[node->NODE] =: OUTCOME
\end{verbatim}\normalsize

\item
\protect \large \begin{verbatim}
*Outcomes[{efg->EFG}] =: LIST(OUTCOME)
\end{verbatim}\normalsize



\item
\protect \large \begin{verbatim}
OutcomeOf[n->NODE] =: INTEGER
\end{verbatim}\normalsize

\bd

\item
[Description:] Finds the number of the outcome associated with the
specified node.
\item
[Return value:] Returns the number corresponding to the outcome
attached to the specified node.  If the node does not have an outcome
attached, a value of zero is returned.
\item
[Required parameters:]\hfil\null
	
\bd
\item
[n:] The node for which the attached outcome is to be found.
\ed

\item
[Optional parameters:] None.
\ed

\item
\protect \large \begin{verbatim}
*Output[file->TEXT] =: OUTPUT
\end{verbatim}\normalsize


%--P--

\item
\protect \large \begin{verbatim}
*Paren[x->T] =: T
\end{verbatim}\normalsize

for all types T.

\bd
\item
[Short form:] \verb+(x)+
\item
[Description:] The identity mapping.  Used to control the order of evaluation
of expressions.
\item
[Return value:] The value of the expression \verb+x+.
\item
[Required parameters:]\hfil\null
\bd
\item
[x:] The expression to be evaluated.
\ed
\item
[Optional parameters:] None.
\ed

\item
\protect \large \begin{verbatim}
*Parent[node->NODE] =: NODE
\end{verbatim}\normalsize

\item
\protect \large \begin{verbatim}
ParentOf[n->NODE] =: NODE
\end{verbatim}\normalsize

\bd
\item
[Description:] Finds the node which is the parent of the specified
node.
\item
[Return value:] Returns the node which is the parent of the specified
node.
\item
[Required parameters:]\hfil\null
	
\bd
\item
[n:] The node for which the parent is to be found.
\ed

\item
[Optional parameters:] None.
\ed

\item
\protect \large \begin{verbatim}
*Payoff[outcome->OUTCOME] =: LIST(T)
\end{verbatim}\normalsize

\item
\protect \large \begin{verbatim}
Payoff[strategy->BEHAV] =: LIST(T)
\end{verbatim}\normalsize

\item
\protect \large \begin{verbatim}
Payoff[strategy->MIXED] =: LIST(T)
\end{verbatim}\normalsize

\item
\protect \large \begin{verbatim}
*Player[infoset->INFOSET] =: EFPLAYER
\end{verbatim}\normalsize

\bd
\item
[Description:] Finds the player who makes the decision at the
specified information set.
\item
[Return value:] Returns the number corresponding to the player found.
\item
[Required parameters:]\hfil\null

\bd
\item
[infoset:] The information set for which the player making the
decision is to be found.
\ed

\item
[Optional parameters:] None.\hfil\null
\ed

\item
\protect \large \begin{verbatim}
*Player[node->NODE] =: EFPLAYER
\end{verbatim}\normalsize

\bd
\item
[Description:] Finds the player who makes the decision at the
specified node.
\item
[Return value:] Returns the number corresponding to the player found.
\item
[Required parameters:]\hfil\null

\bd
\item
[node:] The node for which the player making the decision is to be found.
\ed

\item
[Optional parameters:] None.
\ed

\item
\protect \large \begin{verbatim}
*Players[efg->EFG] =: LIST(EFPLAYER)
\end{verbatim} \normalsize

\bd
\item
[Description:] Finds the players of the given extensive form game. 
\item
[Return value:]  A list of players in the game
\item
[Required parameters:]
\bd
\item
[efg:] The extensive form game for which to find the players.  
\ed
\item
[Optional parameters:]
\ed

\item
\protect \large \begin{verbatim}
*Players[nfg->NFG] =: LIST(NFPLAYER)
\end{verbatim} \normalsize

\bd
\item
[Description:] Finds the players of the given normal form game. 
\item
[Return value:]  A list of players in the game
\item
[Required parameters:]
\bd
\item
[nfg:] The normalform game for which to find the players.  
\ed
\item
[Optional parameters:]
\ed

\item 
\protect \large \begin{verbatim}
*Plus[x->T, y->T] =: T
\end{verbatim} \normalsize

For T = INTEGER, FLOAT, RATIONAL  
\bd
\item
[Short form:] \verb%x + y%
\item
[Description:] Adds $x$ and $y$.
\item
[Return value:] Returns the value of $x + y$  
\item
[Required parameters:]\hfil\null
\bd
\item
[x:] First argument.  
\item
[y:] Second argument
\ed
\item
[Optional parameters:] None.
\ed

\item
\protect \large \begin{verbatim}
*Plus[x->LIST(T), y->LIST(T)] =: LIST(T)
\end{verbatim}\normalsize

\item 
\protect \large \begin{verbatim}
*Plus[x->TEXT, y->TEXT] =: TEXT
\end{verbatim} \normalsize

\bd
\item
[Short form:] \verb%x + y%
\item
[Description:] Concatenates $x$ and $y$.
\item
[Return value:] Returns a TEXT object consisting of $x$ followed by $y$  
\item
[Required parameters:]\hfil\null
\bd
\item
[x:] First argument.  
\item
[y:] Second argument
\ed
\item
[Optional parameters:] None.
\ed

\item
\protect \large \begin{verbatim}
*PriorSibling[node->NODE] =: NODE
\end{verbatim}\normalsize

\bd
\item
[Description:] Finds the prior sibling of the specified node.
\item
[Return value:] The node which is the prior sibling of the specified
node.
\item
[Required parameters:]\hfil\null

\bd
\item
[node:] The node for which the prior sibling is to be found.
\ed

\item
[Optional parameters:] None.
\ed


%--Q--


\item
\protect \large \begin{verbatim}
Quit
\end{verbatim}\normalsize

\bd

\item
[Description:] Exits the command language.
\item
[Return value:] None.
\item
[Required parameters:] None.
\item
[Optional parameters:] None.
\ed

%--R--

\item
\protect \large \begin{verbatim}
RandomNfg[nfg->NFG, {random->BOOL}, {seed->INTEGER}] =: NFG
\end{verbatim}\normalsize
\protect \large \begin{verbatim}
RandomEfg[efg->EFG, {random->BOOL}, {seed->INTEGER}] =: EFG
\end{verbatim}\normalsize

\item
\protect \large \begin{verbatim}
*Rational[x->INTEGER] =: RATIONAL
*Rational[x->FLOAT] =: RATIONAL
*Rational[x->RATIONAL] =: RATIONAL
\end{verbatim} \normalsize


\item
\protect \large \begin{verbatim}
Read[input->INPUT] =: [FLOAT,INTEGER,RATIONAL,TEXT,LIST]
\end{verbatim}\normalsize

\item
\protect \large \begin{verbatim}
*ReadEfg[input->INPUT] =: EFG
\end{verbatim}\normalsize

\bd
\item
[Description:] Reads in an extensive form game from an input stream.  
\item
[Return value:] The game in extensive form read from the stream.
\item

[Required parameters:

\bd
\item
[input:] The input stream from which the extensive form game is to be
read. 
\ed

\item
[Optional parameters:] None.\hfil\null
\ed

\item
\protect \large \begin{verbatim}
*ReadNfg[input->INPUT] =: NFG
\end{verbatim}\normalsize
\bd
\item
[Description:] Reads a normal form game from an input stream.  
\item
[Return value:] The game in normal form read from the stream.
\item
[Required parameters:]\hfil\null

\bd
\item
[file:] The input stream from which the normal form game is to be
read.  
\ed

\item
[Optional parameters:]\hfil\null
\bd

\item
[type:] Indicates whether the file is to be read as a floating- point
normal form or as a rational normal form.  If not given, the default
value is floating-point.
\ed
\ed

\item
\protect \large \begin{verbatim}
RealizProbs[strategy->BEHAV] =: LIST(T)
\end{verbatim}\normalsize

\item
\protect \large \begin{verbatim}
*Remove[list->LIST(T), n->INTEGER] =: LIST(T)
\end{verbatim}\normalsize

\item
\protect \large \begin{verbatim}
RemoveStrategy[support->SUPPORT,list->LIST(STRATEGY)] =: SUPPORT
\end{verbatim}\normalsize

\item
\protect \large \begin{verbatim}
Reveal[infoset->INFOSET, who->LIST[EFPLAYER]], 
	{what->LIST(ACTIONS)}] =: INFOSET
\end{verbatim}\normalsize

\bd
\item
[Description:] Reveals the list of actions at the given information
set to the indicated players, refining their information partitions at
all information sets in the extensive form game accordingly.  The list
of actions is considered as a set Thus, it is indicated to the
spceified players whether the set of actions in \verb+what+ or its
complement has occurred.  If \verb+what+ is not specified, then all of
the actions in the information set are individually revealed to
the players specified.  
\item
[Return value:] Returns the information set \verb+infoset+.
\item
[Required parameters:]\hfil\null
	
\bd
\item
[infoset:] The information set whose actions are to be revealed.  
\ed

\item
[Optional parameters:]\hfil\null
\bd
\item
[who:] The list of players to whom the actions are to be revealed.
Default is all players. 
\item
[what:] The list of actions to be revealed.  
\ed

\ed


\item
\protect \large \begin{verbatim}
*RootNode[efg->EFG] =: NODE
\end{verbatim}\normalsize

\bd
\item
[Description:] Finds the root node of specified game in the given
game in extensive form.  
\item
[Return value:] Returns the root node of the game.
\item
[Required parameters:]\hfil\null
	
\bd
\item
[efg:] The game in extensive form in which the root node is to be
found.
\ed

\item
[Optional parameters:]\hfil\null

\ed

%--S--

\item
\protect \large \begin{verbatim}
*SaveEfg[file->TEXT,overWrite->BOOL] =: BOOL
\end{verbatim}\normalsize

\bd
\item
[Description:] Saves an extensive form game (in standard format) to an
external file.  
\item
[Return value:] \verb+True+ on success, otherwise \verb+False+
\item

[Required parameters:

\bd
\item
[file:] The full path name of the file to which the extensive form game
is to be saved. \ed

\item
[Optional parameters:]\hfil\null
\bd
\item
[overWrite:] If \verb+True+, will overwrite the file.  If \verb+False+, will
not overwrite an existing file.  
\ed
\ed

\item
\protect \large \begin{verbatim}
*SaveNfg[file->TEXT,overWrite->BOOL] =: BOOL
\end{verbatim}\normalsize

\bd
\item
[Description:] Saves normal form game (in standard format) to an
external file.  
\item
[Return value:] \verb+True+ on success, otherwise \verb+False+
\item

[Required parameters:

\bd
\item
[file:] The full path name of the file to which the normal form game
is to be saved. \ed

\item
[Optional parameters:]\hfil\null
\bd
\item
[overWrite:] If \verb+True+, will overwrite the file.  If \verb+False+, will
not overwrite an existing file.  
\ed
\ed

\item

\item
\protect \large \begin{verbatim} 
*SetChanceProbs[infoset->INFOSET, probs->LIST(T)] =: INFOSET
\end{verbatim}\normalsize

\bd
\item
[Description:] Sets the action probabilities to the specified values
for the given node.  This is only meaninful for nodes in information
sets belonging to the chance player.
\item
[Return value:] Returns the information set specified.
\item
[Required parameters:]\hfil\null
	
\bd
\item
[infoset:] The information set for chance at which the actions will be
assigned the specified probabilities.
\item
[probs:] The vector of values which represent the probabilites which
will be assigned to the actions at the indicated node.
\ed

\item
[Optional parameters:] None.
\ed

\item
\protect \large \begin{verbatim}
SetFormat[output->OUTPUT,width->INTEGER, precis->INTEGER] =: OUTPUT
\end{verbatim}\normalsize


\item
\protect \large \begin{verbatim}
*SetName[x<->ACTION, name->TEXT] =: ACTION
*SetName[x<->EFG, name->TEXT] =: EFG
*SetName[x<->INFOSET, name->TEXT] =: INFOSET
*SetName[x<->NFG, name->TEXT] =: NFG
*SetName[x<->NODE, name->TEXT] =: NODE
*SetName[x<->OUTCOME, name->TEXT] =: OUTCOME
*SetName[x<->PLAYER, name->TEXT] =: PLAYER
\end{verbatim}\normalsize

\item
\protect \large \begin{verbatim}
SetOptions[alg->TEXT, param->TEXT, value->T] =: T
\end{verbatim}\normalsize

\item
\protect \large \begin{verbatim}
*SetPayoff[outcome->OUTCOME, player->PLAYER, payoff->T] =: OUTCOME
\end{verbatim}\normalsize

\bd
\item
[Description:] Sets the payoff  of the indicated outcome for the given
player in extensive form, to the value specified.
\item
[Return value:] Returns the outcome.
\item
[Required parameters:]\hfil\null
	
\bd
\item
[outcome:] The extensive form game in which the given outcome is to be
assigned values.
\ed

\item
[Optional parameters:] None.
\bd
\item
[payoff:] The payoff which is to be assigned.
\item
[player:] The player whose payoff is to be set.  
\ed
\ed

\item
\protect \large \begin{verbatim}
*SetPayoff[outcome->OUTCOME, {payoff->LIST(T)}] =: OUTCOME
\end{verbatim}\normalsize

\bd
\item
[Description:] Sets the payoffs  of the indicated outcome in the given
game in extensive form, to the values in the specified vector.
\item
[Return value:] Returns the outcome.
\item
[Required parameters:]\hfil\null
	
\bd
\item
[outcome:] The extensive form game in which the given outcome is to be
assigned values.
\ed

\item
[Optional parameters:] None.
\bd
\item
[payoffs:] The vector of payoffs which are to be assigned to the
outcome.  The length of `value` must match the number of players
defined in the game.
\ed
\ed

\item
\protect \large \begin{verbatim}
SetPayoffNfg[list->LIST(STRATEGY), player->PLAYER, 
	payoff->T] =: LIST(STRATEGY)
\end{verbatim}\normalsize

\item
\protect \large \begin{verbatim}
SetPayoffNf[list->LIST(INTEGER), player->PLAYER, 
	payoff->T] =: LIST(INTEGER)
\end{verbatim}\normalsize

\item
\protect \large \begin{verbatim}
SetPayoffNfg[list->LIST(STRATEGY), payoffs->LIST(T)] =: LIST(STRATEGY)
\end{verbatim}\normalsize

\item
\protect \large \begin{verbatim}
SetPayoffNfg[list->LIST(INTEGER), payoffs->LIST(T)] =: LIST(INTEGER)
\end{verbatim}\normalsize



\item
\protect \large \begin{verbatim}
*SimpDivSolve[nfg->NFG,{stopAfter->INTEGER}, 
         {nRestarts->INTEGER}, {leashLength->INTEGER},
         {nEvals<->INTEGER}, {time<->FLOAT}] =: LIST(MIXED)
\end{verbatim}\normalsize

\bd
\item
[Description:] Computes a Nash equilibrium to a normal form game based
on a simplicial subdivision algorithm.  The algorithm implemented is
that of \cite{VTH:1987}.  The 
algorithm is a simplicial subdivision algorithm which can start at any
point in the simplex.  The algorithm starts with a given grid size,
follows a path of almost completely labeled subsimplexes, and
converges to a completely labeled sub-simplex that approximates the
solution.  Additional accuracy is obtained by refining the grid size
and restarting from the previously found point.  The idea is that by
restarting at a close approximation to the solution, each successive
increase in accuracy will yield a short path, and hence be quick.

In its pure form, the algorithm is guaranteed to find at least one
mixed strategy equilibrium to any n-person game.  Experience shows
that the algorithm works well, and is acceptably fast for many
moderate size problems.  But in some examples it can be quite slow.
The reason for this is that sometimes after restarting with a refined
grid size, even though the starting point is a good approximation to
the solution, the algorithm will go to the boundary of the simplex
before converging back to a point that is close to the original
starting point.  When this occurs, each halving of the grid size will
take twice as long to converge as the previous grid size.  If a high
degree of accuracy is required, or if the normal form is large, this
can result in the algorithm taking a long time to converge.

In order to combat the above difficulty, a parameter 'leash' has been
added to the algorithm which places a limit on the distance which the
algorithm can wander from the restart point. (Setting this parameter
to 0 results in no limit, and gives the pure form of the algorithm.)
With this parameter set to a non trivial value, the algorithm is no
longer guaranteed to converge, and setting small values of the
parameter will sometimes yield lack of convergence.  However,
experience shows that values of the parameter on the order of 10
generally do not destroy convergence, and yield much faster
convergence.

\item
[Return value:] The list of equilibria found.
\item
[Required parameters:]\hfil\null

\bd
\item
[nfg:] The game in normal form on which the operation will be
performed.
\ed

\item
[Optional parameters:]\hfil\null
	
\bd
\item
[stopAfter:] Maximum number of equilibria to find. Default is 1.  
\item 
[nRestarts:] Number of restarts.  At each restart the mesh of the
triangulation is halved.  So this parameter determines the final mesh
by the formula ${1/2}^{ndivs}$.
\item
[leashLength:] Sets the leashlength. Default is 0, which results in no
constraint, or no leash.  
\item
[time:] Returns the elapsed time for the operation.
\ed
\ed

\item

\protect \large \begin{verbatim}
*StartWatch[] =: FLOAT
\end{verbatim}\normalsize

\bd
\item
[Description:] Starts the system stopwatch.
\item
[Return value:] The return value is uninteresting.
\item
[Required parameters:] None.
\item
[Optional parameters:] None.
\ed

\item
\protect \large \begin{verbatim}
*StopWatch[] =: FLOAT
\end{verbatim}\normalsize

\bd
\item
[Description:] Stops the system stopwatch.
\item
[Return value:] The return value is uninteresting.
\item
[Required parameters:] None.
\item
[Optional parameters:] None.
\ed

%--T--

\item
\protect \large \begin{verbatim} 
#TerminalNodes[efg->EFG] =: LIST(NODE)
\end{verbatim}\normalsize

\item
\protect \large \begin{verbatim}
*Text[x->INTEGER] =: TEXT
*Text[x->FLOAT] =: TEXT
*Text[x->RATIONAL] =: TEXT
*Text[x->TEXT] =: TEXT
\end{verbatim} \normalsize

\item 
\protect \large \begin{verbatim}
*Times[x->T, y->T] =: T
\end{verbatim} \normalsize
  
\bd
\item
[Short form:] \verb+x * y+
\item
[Description:] Multiplies $x$ and $y$.
\item
[Return value:] Returns the value of $x * y$  
\item
[Required parameters:]\hfil\null
\bd
\item
[x:] First argument.  
\item
[y:] Second argument
\ed
\item
[Optional parameters:] None.
\ed

%--U--

\protect \large \begin{verbatim}
*UnAssign[x<->T] =: T
\end{verbatim}\normalsize


%--V--

%--W--

\item 
\protect \large \begin{verbatim}
While[test, statements]
\end{verbatim} \normalsize
  
\bd
\item
[Description:] If test is True, evaluates the statement list consisting of
\verb+do+ followed by itself.  If test fails, evaluates the empty
statement list.
\item
[Return value:] If test fails, returns the last value evaluated from the
statement body, or NULL if the loop never executes.

\item
[Required parameters:]\hfil\null
\bd
\item
[antecedent:] Antecedent.  
\item
[then:] Statement list to be evaluated if antecedent is True. 
\item
[else:] Statement list to be evaluated if antecedent is false. 
\ed
\item
[See also:] \verb+For+.
\ed

\item
\protect \large \begin{verbatim}
Write[output->OUTPUT, x->BEHAV] =: OUTPUT 
Write[output->OUTPUT, x->BOOL] =: OUTPUT 
Write[output->OUTPUT, x->EFG] =: OUTPUT
Write[output->OUTPUT, x->INTEGER] =: OUTPUT
Write[output->OUTPUT, x->FLOAT] =: OUTPUT
Write[output->OUTPUT, x->MIXED] =: OUTPUT
Write[output->OUTPUT, x->NFG] =: OUTPUT
Write[output->OUTPUT, x->RATIONAL] =: OUTPUT
Write[output->OUTPUT, x->TEXT, {quoted->BOOL}] =: OUTPUT
\end{verbatim}\normalsize

\bd
\item
[Description:] Does a formatted write of the given object to an output
stream. 
\item
[Return value:] Returns OUTPUT stream output
\item
[Required parameters:]\hfil\null

\bd
\item
[output:] The output stream to which the information is written.  
\item
[x:] The object which is to be written.  
\ed
\ed

%--X--

%--Y--

%--Z--

\end{itemize}
\bibliographystyle{chicagob}
\addcontentsline{toc}{chapter}{Bibliography}
\bibliography{gambit}
\end{document}







 




















