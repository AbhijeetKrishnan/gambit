Gambit Command Language


General Overview

The general purpose of the language is to allow the user to perform a series of
operations and procedures on games in extensive or normal form through a set of
commands.  The games can be randomly generated by a procedure within the
program, read into the program from external files, or created in extensive
form using the language.  Games created or contained in the extensive form can
be transformed into the normal form using a procedure within the program.

The language consists of a series of statements, each ending in a semi-colon
or carriage return.  A carriage return will not be interpreted as the end of a
statement, however, if there are unmatched parentheses or brackets.  For
example, the following two lines are interpreted as one statement:

	Output[1,
	2]

If both lines are followed by a carriage return, the statement will generate
the line of output: 

	1, 2

The user has the option of creating a text file containing a series of commands
which can then be run together, as a program, by the language.  The output of 
such a program can be directed either to the screen or to a file specified by 
the user.  A text file can be read by the language as a program by placing an 
input file command "<" followed by the name of the file, following the command 
which enters the command language.  By using an output file command ">" 
followed by the name of a previously non-existent file, the user can direct the
output of the command language to an external text file.  For example:

	> gcl_sun < input.gcl > output.out


Output

Output can be produced by a program or command using the following command:

	Output[{expression1}, {expression2}, . . . ];

The command produces a line of output containing the values or strings 
specified by the expressions.  An expression can be a variable of any type or 
can evaluate to any value or string which can be contained in a variable.  The 
brackets "{ }" are not a part of the language and are used only to enclose
expressions which are also not a part of the language.


Variables

Variables can be defined and given a value using the ":=" symbol where the name
of the variable is to the left of the symbol and to the right there is an
expression which evaluates to a value of the type desired for the variable.
A variable can be of the following types:

	BOOLEAN		Boolean (evaluates to constants:  True or False)
	INTEGER		Evaluates to an integer of arbitrary size
	RATIONAL	Evaluates to a rational number of arbitrary precision
	TEXT		String of arbitrary length enclosed in quotation marks
	EXTFORM		Game in extensive form (created or read)
	NORFORM		Game in normal form (created or read)
	NODE		A node (may only be a subvariable of an extensive form
			problem)
	DATATYPE	Defines precision (constants:  Float or Rational)

Additionally, variables of type BOOLEAN, INTEGER, RATIONAL, and TEXT can be 
contained in lists of arbitrary length.  Lists are defined similarly to the 
above variables, except that the expression to the right of the ":=" symbol
contains a vector of variables of the same type enclosed in brackets and
separated by commas.  For example:

	BOOLLIST := { BOOLEAN1, BOOLEAN2, . . . . , BOOLEANn }

The elements of a list are indexed sequentially.  Therefore, the ith element of
a list is defined as:

	BOOLLIST[[i]]	which evaluates to a single variable of type BOOLEAN

In the same manner, list variables of type INTLIST, RATLIST, and TEXTLIST can
be created and accessed for variables of type INTEGER, RATIONAL, and TEXT, 
respectively.


Flow Control

Gambit Command Language also contains mechanisms which allow flow control 
within a program in one of three manners.

While[BOOLEAN, {series of commands}];	Evaluates the BOOLEAN expression or
					variable.  As long as it evaluates to
					TRUE, the series of commands are 
					performed.  Each command must be 
					followed by a semi-colon.  If the
					BOOLEAN evaluates to FALSE, the loop
					ends without performing the commands.

If[BOOLEAN, {series of commands}];	Evaluates the BOOLEAN expression once.
					If it evaluates to TRUE, the series of
					commands are performed.  Each command
					must be followed by a semi-colon.  If
					the BOOLEAN evaluates to FALSE, none
					of the commands are performed.

	NOTE:	The While command functions as a loop.  That is, it continues
		to perform the series of commands as long as the expression
		evaluates to TRUE.  The If statement, however, only performs
		the series of commands once if the expression evaluates to
		TRUE.


Procedures

The following is a list of procedures with the operations that they perform
in the Gambit Command Language:

Variable types in all capital letters indicate the type of the required
parameter or return value of the procedure.

The symbol "=:" to the right of a procedure specifies the type of the return
value (if any exists) produced by the procedure.

Paremeters enclosed in brackets "{ }" represent optional parameters which
are not necessary for the procedure to function.

----------

AllLemke[N->NORFORM] =: NORFORM

   -	Description:  Searches for all solutions of the specified normal form
	problem using the Lemke-Howson algorithm.
   -	Return value:  Returns the same problem in 
	normal form.
   -	Required parameters:
	  *  N:  The problem normal form for which solutions are to be found.
   -	Optional parameters:  None.

----------

AppendNode[n->NODE, pl->INTEGER, br->INTEGER] =: NODE

   -	Description:  Appends the number of branches specified to the indicated
	terminal node and places the node in a new information set for the
	given player.
   -	Return value:  Returns the node to which new branches have been 
	appended.
   -	Required parameters:
	  *  n:  The terminal node to which branches are to be added.
	  *  pl:  The player whose information set the node will now belong to.
	     If the player specified is not already defined in the game, the 
	     player is created automatically. 
	  *  br:  The number of branches to be appended to the node.
   -	Optional parameters:  None.

----------

AttachOutcome[n->NODE, outc->INTEGER] =: NODE

   -	Description:  Attaches the indicated outcome to the specified node.  
   -	Return value:  Returns the node to which the outcome is attached.
   -	Required parameters:
	  *  n:  The node to which the outcome is to be attached.  Note that
	     the node need not be terminal in order to have an outcome 
	     attached.
	  *  outc:  The number of the outcome to be attached.  If no outcome 
	     with the indicated number exists, no action is taken.  
   -	Optional parameters:  None.

----------

Car[l->BOOLLIST] =: BOOLEAN
Car[l->INTLIST] =: INTEGER
Car[l->RATLIST] =: RATIONAL
Car[l->TEXTLIST] =: TEXT

   -	Description:  Finds the first element of the specified list.
   -	Return value:  The first element of the list.
   -	Required parameters:
	  *  l:  The list on which the procedure is to be performed.
   -	Optional parameters:  None.

----------

Cdr[l->BOOLLIST] =: BOOLLIST
Cdr[l->INTLIST] =: INTLIST
Cdr[l->RATLIST] =: RATLIST
Cdr[l->TEXTLIST] =: TEXTLIST

   -	Description:  Deletes the first element of the specified list.
   -	Return value:  The list minus the first element.
   -	Required parameters:
	  *  l:  The list on which the procedure is to be performed.
   -	Optional parameters:  None.

----------

Contains[e->BOOLEAN, l->BOOLLIST] =: BOOLEAN
Contains[e->INTEGER, l->INTLIST] =: BOOLEAN
Contains[e->RATIONAL, l->RATLIST] =: BOOLEAN
Contains[e->TEXT, l->TEXTLIST] =: BOOLEAN

   -	Description:  Finds whether the specified value is contained in the 
	indicated list.
   -	Return value:  True when the specified value is contained in the list.
   -	Required parameters:
	  *  e:  The value to be searched for.
	  *  l:  The list which is to be searched in.
   -	Optional parameters:  None.

----------

CopyTree[n->NODE, to->NODE] =: NODE

   -	Description:  Copies the subtree rooted at the specified node to the 
	indicated node.
   -	Return value:  XXXX.
   -	Required parameters:
	  *  n:  The node at which the subtree to be copied is rooted.
	  *  to:  The node to which the subtree is to be copied.
   -	Optional parameters:  None.

----------

DeleteAction[n->NODE, br->INTEGER] =: NODE

   -	Description:  Deletes the indicated action number from the information
	set containing the specified node.
   -	Return value:  The given node.
   -	Required parameters:
	  *  n:  The node belonging to the information set from which the 
	     action is to be deleted.
	  *  br:  The number of the branch to be deleted.
   -	Optional parameters:  None.

----------

DeleteNode[n->NODE, keep->INTEGER] =: NODE

   -	Description:  Deletes the indicated node from the tree.  The specified
	child of the deleted node is kept and replaces the node in its spot in
	the tree.  The subtrees associated with all other children of the 
	deleted node are deleted.
   -	Return value:  The node which replaces the deleted node.
   -	Required parameters:
	  *  n:  The node to be deleted.
	  *  keep:  The child of the deleted node which is to be kept.
   -	Optional parameters:  None.

----------

DeleteTree[n->NODE] =: NODE

   -	Description:  Deletes the subtree rooted at the specified node.  This 
	node then becomes a terminal node.
   -	Return value:  The indicated node which is now a terminal node.
   -	Required parameters:
	  *  n:  The node where the subtree to be deleted is rooted.
   -	Optional parameters:  None.

----------

DetachOutcome[n->NODE] =: NODE

   -	Description:  Sets the outcome attached to the specified node to the 
	null outcome.
   -	Return value:  The indicated node.
   -	Required parameters:
	  *  n:  The node whose outcome is to be detached.
   -	Optional parameters:  None.

----------

ElimAllDom[N->NORFORM, {strong->BOOLEAN}, {time<->RATIONAL}, {output->TEXT}] 
	=: INTEGER

   -	Description:  Iteratively eliminates dominated strategies for all 
	players.
   -	Return value:  The number of the new strategy set.  The original full
	strategy set is numbered 1.  Each iteration creates a new strategy set
	with an integer value one greater than that of the previous strategy 
	set.  Therefore, if no dominated strategies are found, the return value
	is equal to 1.
   -	Required parameters:
	  *  N:  The problem in normal form for which dominated strategies are
	     to be eliminated.
   -	Optional parameters:
	  *  strong:  Determines the type of dominance used.  The default value
	     is False, which signifies that weak dominance is used.  If 
	     `strong` is specified to be True, strong dominance is used.  
	  *  time:  Returns the elapsed time for the operation.
	  *  output:  Redirects the output of the procedure to a specified 
	     file.  Directing output to ``/dev/null'' under Unix turns output
	     off.  By default, it directs output to the screen.

----------

ElimDom[N->NORFORM, {pl->INTLIST}, {strong->BOOLEAN}, {time<->RATIONAL},
	{output->TEXT}] =: INTEGER

   -	Description:  Performs one iteration of eliminating dominated 
	strategies for the list of players specified.  
   -	Return value:  The number of the new strategy set.  The original full
	strategy set is numbered 1.  Each iteration creates a new strategy set
	with an integer value one greater than that of the previous strategy 
	set.  Therefore, if no dominated strategies are found, the return value
	is equal to 1.
   -	Required paremeters:
	  *  N:  The problem in normal form for which dominated strategies are
	     to be eliminated.
   -	Optional parameters:
	  *  pl:  Specifies the list of players for which the procedure is to
	     be performed.  If unspecified, it defaults to all players. 
 	  *  strong:  Determines the type of dominance used.  The default value
	     is False, which signifies that weak dominance is used.  If 
	     `strong` is specified to be True, strong dominance is used.  
	  *  time:  Returns the elapsed time for the operation.
	  *  output:  Redirects the output of the procedure to a specified 
	     file.  Directing output to ``/dev/null'' under Unix turns output
	     off.  By default, it directs output to the screen.

----------

EfgToNfg[E->EXTFORM, {time<->RATIONAL}] =: NORFORM

	Converts a problem in extensive form to the same problem as a normal
form game and returns the normal form game.  The optional parameter 'time'
returns the elapsed time for the operation.

----------

GetActionName[n->NODE, br->INTEGER] =: TEXT

	Returns the name assigned to the action corresponding to the branch
indicated by 'br` of the node specified by 'n'.

----------

GetInfosetName[n->NODE] =: TEXT

	Returns the name of the information set to which the node specified by
'n' belongs.

----------

GetNextSibling[n->NODE] =: NODE

	Returns the next sibling of the node specified by 'n'.

----------

GetOutcome[E->EXTFORM, outc->INTEGER] =: RATLIST

	Returns the vector of payoffs to the players for the outcome whose
number is indicated by 'outc' in the problem in extensive form given by 'E'.
If the outcome is not defined, the procedure returns the empty list.

---------

GetOutcomeName[E->EXTFORM, outc->INTEGER] =: TEXT

	Returns the name of the outcome specified by 'outc' in the problem in
extensive form indicated by 'E'.  If the outcome is not defined, an error is
given.  By default, the name of an outcome is the empty string.

----------

GetPlayerName[E->EXTFORM, pl->INTEGER] =: TEXT

	Returns the name of the player indicated by 'pl' in the problem in 
extensive form given by 'E'.

----------

GetPriorSibling[n->NODE] =: NODE

	Returns the prior sibling of the node specified by 'n'.

----------

GetTitle[E->EXTFORM] =: TEXT

	Returns the title of the problem in extensive form specified by 'E'.

----------

Gobit[N->NORFORM, {nequilib->INTEGER}, {type->INTEGER},
	{maxitsDFP->INTEGER}, {maxitsBrent->INTEGER}, {minLam->RATIONAL},
	{maxLam->RATIONAL}, {delLam->RATIONAL}, {tolDFP->RATIONAL},
	{tolBrent}->RATIONAL}, {time<->RATIONAL}, {output->TEXT}] =: INTEGER

	Computes the Gobit solution for the specified normal form game.  The
optional parameter `nequilib` sets the number of equilibria to find and has a
default value of 1.  The optional parameter 'type'  XXXX.  The optional 
parameter 'maxitsDFP' XXXX and has a default value of XXXX.  The optional
parameter 'maxitsBrent` sets the maximum number of iterations in Brent and has
a default value of XXXX.  The optional parameters `minLam` and `maxLam` set the
minimum and maximum values of Lambda and have default values of .01 and 30,
respectively.  The optional parameter `delLam` specifies the change in the 
value of Lambda and has a default value of `1.01`.  The optional parameter
`tolDFP` XXXX and has a default value of 1.0E-10.  The optional parameter
`tolBrent` XXXX and has a default value of XXXX.  The optional parameter `time`
returns the elapsed time for the operation.  The optional parameter `output`
can be used to redirect the output of this procedure to a file.  Directing output to ``/dev/null'' under Unix turns output off.  By default, it
directs output to the screen.

----------

HasOutcome[E->EXTFORM, outc->INTEGER] =: BOOLEAN

	Returns True exactly when the problem in extensive form specified by 
'E' contains an outcome with the number given by 'outc'.  Returns false 
otherwise.

----------

IsWatchRunning[] =: BOOLEAN

	Returns "True" if the system stopwatch is currently running.

----------

Lemke[N->NORFORM] =: NORFORM

	Searches for one solution of the normal form game specified using the
Lemke-Howson algorithm.  Returns the same problem in normal form.	

----------

Length[l->BOOLLIST] =: INTEGER
Length[l->INTLIST] =: INTEGER
Length[l->RATLIST] =: INTEGER
Length[l->TEXTLIST] =: INTEGER

	Returns the number of elements in the specified list.

----------

MergeInfosets[n->NODE, to->NODE] =: NODE

	Moves all the nodes from the information set to which the node
specified by 'n' belongs to the information to which the node indicated by 'to'
belongs.

----------

NewEfg[] =: EXTFORM

	Allocates and returns a new extensive form game.

----------

NewOutcome[E->EXTFORM, {outc->INTEGER}] =: INTEGER

	Creates and assigns a number to a new outcome within the specified 
problem in extensive form.  Returns the number of the newly created outcome.
The optional parameter 'outc' allows the user to specify the number given to 
the new outcome created.  If `outc` is not specified, the new outcome is
assigned the lowest positive number for which no game is defined.  If the 
number specified by `outc` already exists as an outcome, no action is taken and
the procedure returns a value of zero.

----------

PureNash[N->NORFORM, {num->INTEGER}, {time<->RATIONAL}, {output->TEXT}]
	 =: INTEGER

	Finds the weak pure strategy Nash equilibria in the specified normal 
form game.  Returns the number of equilibria found.  The optional parameter 
'num' allows the user to set the maximum number of Nash equilibria to find and 
has a default value of 1.  The optional parameter 'time' returns the elapsed 
time for the operation.  The optional parameter `output` can be used to
redirect the output of this procedure to a file.  Directing output to ``/dev/
null'' under Unix turns output off.  By default, it directs output to the
screen.

----------

Quit

	Exits the command language.

----------

RandomNfg[dim->INTLIST, {seed->INTEGER}] =: NORFORM

	Constructs a new problem in normal form of the specified
dimensionality.  If a value is specified for the optional parameter 'seed',
the random number generator is reseeded with the given integer.

----------

ReadEfgFile[file->TEXT] =: EXTFORM

	Creates a new problem whose contents are the extensive form game
stored in the given file.

----------

ReadNfgFile[file->TEXT, {type->DATATYPE}] =: NORFORM

	Creates a new problem whose contents are the normal form game stored
in the given file.  The file is read as a floating-point normal form unless
otherwise specified by the user.

----------

RootNode[E->EXTFORM, {game->INTEGER}] =: NODE

	Returns, as a variable of type NODE, the root node of game number 
`game` in the specified problem.  If 'game' is not specified, the first game is
used by default.

----------

Run[TEXT]

	Includes the commands in the file specified by the string TEXT as
commands in the running GCL program.  TEXT may be any legal expression
evaluating to type TEXT.  The program treats the command as if it were
replaced by the contents of the file.

----------

SetActionName[n->NODE, br->INTEGER, name->TEXT] =: NODE

	Sets the name of the branch number specified by 'br' of the indicated 
node to the name given.

----------

SetInfosetName[n->NODE, name->TEXT] =: NODE

	Assigns the name given by 'name' to the information set to which the 
node indicated by 'n' belongs.

----------

SetOutcome[E->EXTFORM, outc->INTEGER, value->INTLIST] =: EXTFORM
SetOutcome[E->EXTFORM, outc->INTEGER, value->RATLIST] =: EXTFORM

	Sets the value of the outcome indicated by `outc` in the give problem 
in extensive form, to the vector specified by the parameter 'value'.  The 
length of `value` must match the number of players defined in the problem.  If
no outcome with the given number exists, then it is created.  Returns the 
modified problem in extensive form.


SetOutcome[E->EXTFORM, outc->INTEGER, pl->INTEGER, value->INTEGER] =: EXTFORM
SetOutcome[E->EXTFORM, outc->INTEGER, pl->INTEGER, value->RATIONAL] =: EXTFORM

	Sets the payoff to the player indicated by 'pl' of the outcome whose 
number is indicated by `outc` in the given problem in extensive form to the
given value.  If no outcome with the given number exists, no action is 
performed. 

----------

SetOutcomeName[E->EXTFORM, outc->INTEGER, name->TEXT] =: EXTFORM

	Assigns the outcome with the specified number the title given by 
'name'.  Returns the original extensive form problem with the outcome name 
assigned.

----------

SetPlayerName[E->EXTFORM, pl->INTEGER, name->TEXT] =: EXTFORM

	Assigns the player with the specified number the namer given by 'name'.
Returns the original extensive form problem with the player name assigned.

----------

SetTitle[E->EXTFORM, title->TEXT] =: EXTFORM

	Assigns the title specified by `title` to the problem in extensive form
given by 'E'.  Returns the problem with the title assigned.

----------

StartWatch[] =: INTEGER

	Starts the system stopwatch.  The return value is uninteresting.

----------

StopWatch[] =: INTEGER

	Stops the system stopwatch.  The return value is uninteresting.

----------

WriteEfgFile[E->EXTFORM, file->TEXT] =: EXTFORM

	Writes the given extensive form to the specified file.  Returns the
same extensive form.

----------

WriteNfgFile[N->NORFORM, file->TEXT] =: NORFORM

	Writes the given normal form to the specified file.  Returns the
same normal form.

----------





























